\usepackage{graphicx}
\graphicspath{{recursos/}}

% cor das caixas
\usepackage{xcolor}
\definecolor{verde}{RGB}{22,159,37}

% pacote de configuração
\usepackage[
    nome = Python,
    cor  = verde,
    logo = logo.pdf,
    link = https://www.python.org/
]{pacotes/tutorial}

\newcommand{\python}{\software}
\newcommand{\novonome}[2]{
    \newcommand{#1}{%
        \texttt{#2}
    }
}
\novonome{\matplotlib}{Matplotlib}
\novonome{\numpy}{NumPy}
\novonome{\scipy}{SciPy}
\novonome{\pandas}{Pandas}

% pacotes extras
\usepackage{caption, subcaption, pdfpages, float}
\usepackage{circuitikz, graphics, wrapfig, pgf}

% pacotes para importar código
\usepackage{caption, booktabs}
\usepackage[section, newfloat]{minted}
\definecolor{sepia}{RGB}{252,246,226}
\setminted{
    bgcolor = sepia,
    style   = pastie,
    frame   = leftline,
    autogobble,
    samepage,
    python3,
}
\setmintedinline{
    bgcolor={}
}

% ambientes de códigos de Python
\newmintedfile[pyinclude]{python}{}
\newmintinline[pyline]{python}{}
\newcommand{\pyref}[2]{\href{#1}{\texttt{#2}}}

\SetupFloatingEnvironment{listing}{name=Código}

\newcommand{\novopynome}[2]{
    \newcommand{#1}{%
        \pyline{#2}
    }
}
\novopynome{\dataframe}{DataFrame}
\novopynome{\pyplot}{pyplot}

% começa a seção no `0`
\setcounter{section}{-1}
