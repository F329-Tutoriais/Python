Curvas de nível servem para representar dados tridimensionais em um plano. As três variáveis para esse tipo de gráfico são $x$ e $y$ independentes e $z = f(x,y)$.

Serão usados, como exemplo, dados semelhantes aos do experimento sobre potencial elétrico entre barras de cobre em uma solução condutiva. Nesse caso, as variáveis independentes são as distâncias $x$ e $y$ no plano e a variável dependente é o potencial $V$ de cada ponto.

\begin{table}[H]
    \centering
    \begin{tabular}{ccc}
\toprule
Posição $X$ [\si{\cm}] & Posição $Y$ [\si{\cm}] & Tensão [\si{\volt}] \\
\midrule
       $-12.0 \pm 0.5$ &         $-7.0 \pm 0.5$ &   $0.330 \pm 0.241$ \\
        $-8.0 \pm 0.5$ &         $-7.3 \pm 0.5$ &   $0.329 \pm 0.240$ \\
        $-4.0 \pm 0.5$ &         $-7.3 \pm 0.5$ &   $0.330 \pm 0.241$ \\
         $0.0 \pm 0.5$ &         $-7.4 \pm 0.5$ &   $0.328 \pm 0.240$ \\
         $4.0 \pm 0.5$ &         $-7.4 \pm 0.5$ &   $0.327 \pm 0.239$ \\
         $8.0 \pm 0.5$ &         $-7.3 \pm 0.5$ &   $0.328 \pm 0.240$ \\
        $12.0 \pm 0.5$ &         $-7.4 \pm 0.5$ &   $0.332 \pm 0.242$ \\
       $-12.0 \pm 0.5$ &         $-4.6 \pm 0.5$ &   $0.661 \pm 0.473$ \\
        $-8.0 \pm 0.5$ &         $-4.5 \pm 0.5$ &   $0.661 \pm 0.473$ \\
        $-4.0 \pm 0.5$ &         $-4.6 \pm 0.5$ &   $0.663 \pm 0.474$ \\
\multicolumn{3}{c}{\dots} \\
\bottomrule
\end{tabular}

    \caption{Primeiros 10 pontos coletados.}
    \label{tab:contorno:dados}
\end{table}


\subsection{Curvas por Malha Triangular}
    \begin{listing}[H]
        \caption{Desenho das curvas de nível}
        \label{code:contorno:base}

        \pyinclude[firstline=28, lastline=34]{recursos/contorno/contorno.py}
    \end{listing}

    A forma mais abrangente de se desenhar as curvas de nível é formando uma \href{https://www.wikiwand.com/pt/Malha_triangular}{malha triangular} dos pontos, onde cada ponto $(x, y)$ é tratado vétice de um triângulo e a váriavel dependente é usada para montar planos com os vértices de cada triângulo. Por mais que o processo pareça um pouco complicado, isso é feito com apenas uma função, \pyref{https://matplotlib.org/3.1.0/api/_as_gen/matplotlib.pyplot.tricontour.html}{tricontour}.

    \begin{figure}[H]
        \centering
        %% Creator: Matplotlib, PGF backend
%%
%% To include the figure in your LaTeX document, write
%%   \input{<filename>.pgf}
%%
%% Make sure the required packages are loaded in your preamble
%%   \usepackage{pgf}
%%
%% Figures using additional raster images can only be included by \input if
%% they are in the same directory as the main LaTeX file. For loading figures
%% from other directories you can use the `import` package
%%   \usepackage{import}
%% and then include the figures with
%%   \import{<path to file>}{<filename>.pgf}
%%
%% Matplotlib used the following preamble
%%   
%%       \usepackage[portuguese]{babel}
%%       \usepackage[T1]{fontenc}
%%       \usepackage[utf8]{inputenc}
%%   \usepackage{fontspec}
%%
\begingroup%
\makeatletter%
\begin{pgfpicture}%
\pgfpathrectangle{\pgfpointorigin}{\pgfqpoint{3.600000in}{2.800000in}}%
\pgfusepath{use as bounding box, clip}%
\begin{pgfscope}%
\pgfsetbuttcap%
\pgfsetmiterjoin%
\pgfsetlinewidth{0.000000pt}%
\definecolor{currentstroke}{rgb}{0.000000,0.000000,0.000000}%
\pgfsetstrokecolor{currentstroke}%
\pgfsetstrokeopacity{0.000000}%
\pgfsetdash{}{0pt}%
\pgfpathmoveto{\pgfqpoint{0.000000in}{0.000000in}}%
\pgfpathlineto{\pgfqpoint{3.600000in}{0.000000in}}%
\pgfpathlineto{\pgfqpoint{3.600000in}{2.800000in}}%
\pgfpathlineto{\pgfqpoint{0.000000in}{2.800000in}}%
\pgfpathclose%
\pgfusepath{}%
\end{pgfscope}%
\begin{pgfscope}%
\pgfsetbuttcap%
\pgfsetmiterjoin%
\pgfsetlinewidth{0.000000pt}%
\definecolor{currentstroke}{rgb}{0.000000,0.000000,0.000000}%
\pgfsetstrokecolor{currentstroke}%
\pgfsetstrokeopacity{0.000000}%
\pgfsetdash{}{0pt}%
\pgfpathmoveto{\pgfqpoint{0.378629in}{0.330514in}}%
\pgfpathlineto{\pgfqpoint{3.450000in}{0.330514in}}%
\pgfpathlineto{\pgfqpoint{3.450000in}{2.650000in}}%
\pgfpathlineto{\pgfqpoint{0.378629in}{2.650000in}}%
\pgfpathclose%
\pgfusepath{}%
\end{pgfscope}%
\begin{pgfscope}%
\pgfpathrectangle{\pgfqpoint{0.378629in}{0.330514in}}{\pgfqpoint{3.071371in}{2.319486in}}%
\pgfusepath{clip}%
\pgfsetbuttcap%
\pgfsetroundjoin%
\pgfsetlinewidth{0.803000pt}%
\definecolor{currentstroke}{rgb}{0.800000,0.800000,0.800000}%
\pgfsetstrokecolor{currentstroke}%
\pgfsetdash{{2.960000pt}{1.280000pt}}{0.000000pt}%
\pgfpathmoveto{\pgfqpoint{0.634576in}{0.330514in}}%
\pgfpathlineto{\pgfqpoint{0.634576in}{2.650000in}}%
\pgfusepath{stroke}%
\end{pgfscope}%
\begin{pgfscope}%
\definecolor{textcolor}{rgb}{0.150000,0.150000,0.150000}%
\pgfsetstrokecolor{textcolor}%
\pgfsetfillcolor{textcolor}%
\pgftext[x=0.634576in,y=0.252737in,,top]{\color{textcolor}\rmfamily\fontsize{8.330000}{9.996000}\selectfont \(\displaystyle -10\)}%
\end{pgfscope}%
\begin{pgfscope}%
\pgfpathrectangle{\pgfqpoint{0.378629in}{0.330514in}}{\pgfqpoint{3.071371in}{2.319486in}}%
\pgfusepath{clip}%
\pgfsetbuttcap%
\pgfsetroundjoin%
\pgfsetlinewidth{0.803000pt}%
\definecolor{currentstroke}{rgb}{0.800000,0.800000,0.800000}%
\pgfsetstrokecolor{currentstroke}%
\pgfsetdash{{2.960000pt}{1.280000pt}}{0.000000pt}%
\pgfpathmoveto{\pgfqpoint{1.274445in}{0.330514in}}%
\pgfpathlineto{\pgfqpoint{1.274445in}{2.650000in}}%
\pgfusepath{stroke}%
\end{pgfscope}%
\begin{pgfscope}%
\definecolor{textcolor}{rgb}{0.150000,0.150000,0.150000}%
\pgfsetstrokecolor{textcolor}%
\pgfsetfillcolor{textcolor}%
\pgftext[x=1.274445in,y=0.252737in,,top]{\color{textcolor}\rmfamily\fontsize{8.330000}{9.996000}\selectfont \(\displaystyle -5\)}%
\end{pgfscope}%
\begin{pgfscope}%
\pgfpathrectangle{\pgfqpoint{0.378629in}{0.330514in}}{\pgfqpoint{3.071371in}{2.319486in}}%
\pgfusepath{clip}%
\pgfsetbuttcap%
\pgfsetroundjoin%
\pgfsetlinewidth{0.803000pt}%
\definecolor{currentstroke}{rgb}{0.800000,0.800000,0.800000}%
\pgfsetstrokecolor{currentstroke}%
\pgfsetdash{{2.960000pt}{1.280000pt}}{0.000000pt}%
\pgfpathmoveto{\pgfqpoint{1.914314in}{0.330514in}}%
\pgfpathlineto{\pgfqpoint{1.914314in}{2.650000in}}%
\pgfusepath{stroke}%
\end{pgfscope}%
\begin{pgfscope}%
\definecolor{textcolor}{rgb}{0.150000,0.150000,0.150000}%
\pgfsetstrokecolor{textcolor}%
\pgfsetfillcolor{textcolor}%
\pgftext[x=1.914314in,y=0.252737in,,top]{\color{textcolor}\rmfamily\fontsize{8.330000}{9.996000}\selectfont \(\displaystyle 0\)}%
\end{pgfscope}%
\begin{pgfscope}%
\pgfpathrectangle{\pgfqpoint{0.378629in}{0.330514in}}{\pgfqpoint{3.071371in}{2.319486in}}%
\pgfusepath{clip}%
\pgfsetbuttcap%
\pgfsetroundjoin%
\pgfsetlinewidth{0.803000pt}%
\definecolor{currentstroke}{rgb}{0.800000,0.800000,0.800000}%
\pgfsetstrokecolor{currentstroke}%
\pgfsetdash{{2.960000pt}{1.280000pt}}{0.000000pt}%
\pgfpathmoveto{\pgfqpoint{2.554183in}{0.330514in}}%
\pgfpathlineto{\pgfqpoint{2.554183in}{2.650000in}}%
\pgfusepath{stroke}%
\end{pgfscope}%
\begin{pgfscope}%
\definecolor{textcolor}{rgb}{0.150000,0.150000,0.150000}%
\pgfsetstrokecolor{textcolor}%
\pgfsetfillcolor{textcolor}%
\pgftext[x=2.554183in,y=0.252737in,,top]{\color{textcolor}\rmfamily\fontsize{8.330000}{9.996000}\selectfont \(\displaystyle 5\)}%
\end{pgfscope}%
\begin{pgfscope}%
\pgfpathrectangle{\pgfqpoint{0.378629in}{0.330514in}}{\pgfqpoint{3.071371in}{2.319486in}}%
\pgfusepath{clip}%
\pgfsetbuttcap%
\pgfsetroundjoin%
\pgfsetlinewidth{0.803000pt}%
\definecolor{currentstroke}{rgb}{0.800000,0.800000,0.800000}%
\pgfsetstrokecolor{currentstroke}%
\pgfsetdash{{2.960000pt}{1.280000pt}}{0.000000pt}%
\pgfpathmoveto{\pgfqpoint{3.194052in}{0.330514in}}%
\pgfpathlineto{\pgfqpoint{3.194052in}{2.650000in}}%
\pgfusepath{stroke}%
\end{pgfscope}%
\begin{pgfscope}%
\definecolor{textcolor}{rgb}{0.150000,0.150000,0.150000}%
\pgfsetstrokecolor{textcolor}%
\pgfsetfillcolor{textcolor}%
\pgftext[x=3.194052in,y=0.252737in,,top]{\color{textcolor}\rmfamily\fontsize{8.330000}{9.996000}\selectfont \(\displaystyle 10\)}%
\end{pgfscope}%
\begin{pgfscope}%
\pgfpathrectangle{\pgfqpoint{0.378629in}{0.330514in}}{\pgfqpoint{3.071371in}{2.319486in}}%
\pgfusepath{clip}%
\pgfsetbuttcap%
\pgfsetroundjoin%
\pgfsetlinewidth{0.803000pt}%
\definecolor{currentstroke}{rgb}{0.800000,0.800000,0.800000}%
\pgfsetstrokecolor{currentstroke}%
\pgfsetdash{{2.960000pt}{1.280000pt}}{0.000000pt}%
\pgfpathmoveto{\pgfqpoint{0.378629in}{0.552931in}}%
\pgfpathlineto{\pgfqpoint{3.450000in}{0.552931in}}%
\pgfusepath{stroke}%
\end{pgfscope}%
\begin{pgfscope}%
\definecolor{textcolor}{rgb}{0.150000,0.150000,0.150000}%
\pgfsetstrokecolor{textcolor}%
\pgfsetfillcolor{textcolor}%
\pgftext[x=0.150000in,y=0.512785in,left,base]{\color{textcolor}\rmfamily\fontsize{8.330000}{9.996000}\selectfont \(\displaystyle -6\)}%
\end{pgfscope}%
\begin{pgfscope}%
\pgfpathrectangle{\pgfqpoint{0.378629in}{0.330514in}}{\pgfqpoint{3.071371in}{2.319486in}}%
\pgfusepath{clip}%
\pgfsetbuttcap%
\pgfsetroundjoin%
\pgfsetlinewidth{0.803000pt}%
\definecolor{currentstroke}{rgb}{0.800000,0.800000,0.800000}%
\pgfsetstrokecolor{currentstroke}%
\pgfsetdash{{2.960000pt}{1.280000pt}}{0.000000pt}%
\pgfpathmoveto{\pgfqpoint{0.378629in}{0.870669in}}%
\pgfpathlineto{\pgfqpoint{3.450000in}{0.870669in}}%
\pgfusepath{stroke}%
\end{pgfscope}%
\begin{pgfscope}%
\definecolor{textcolor}{rgb}{0.150000,0.150000,0.150000}%
\pgfsetstrokecolor{textcolor}%
\pgfsetfillcolor{textcolor}%
\pgftext[x=0.150000in,y=0.830523in,left,base]{\color{textcolor}\rmfamily\fontsize{8.330000}{9.996000}\selectfont \(\displaystyle -4\)}%
\end{pgfscope}%
\begin{pgfscope}%
\pgfpathrectangle{\pgfqpoint{0.378629in}{0.330514in}}{\pgfqpoint{3.071371in}{2.319486in}}%
\pgfusepath{clip}%
\pgfsetbuttcap%
\pgfsetroundjoin%
\pgfsetlinewidth{0.803000pt}%
\definecolor{currentstroke}{rgb}{0.800000,0.800000,0.800000}%
\pgfsetstrokecolor{currentstroke}%
\pgfsetdash{{2.960000pt}{1.280000pt}}{0.000000pt}%
\pgfpathmoveto{\pgfqpoint{0.378629in}{1.188406in}}%
\pgfpathlineto{\pgfqpoint{3.450000in}{1.188406in}}%
\pgfusepath{stroke}%
\end{pgfscope}%
\begin{pgfscope}%
\definecolor{textcolor}{rgb}{0.150000,0.150000,0.150000}%
\pgfsetstrokecolor{textcolor}%
\pgfsetfillcolor{textcolor}%
\pgftext[x=0.150000in,y=1.148260in,left,base]{\color{textcolor}\rmfamily\fontsize{8.330000}{9.996000}\selectfont \(\displaystyle -2\)}%
\end{pgfscope}%
\begin{pgfscope}%
\pgfpathrectangle{\pgfqpoint{0.378629in}{0.330514in}}{\pgfqpoint{3.071371in}{2.319486in}}%
\pgfusepath{clip}%
\pgfsetbuttcap%
\pgfsetroundjoin%
\pgfsetlinewidth{0.803000pt}%
\definecolor{currentstroke}{rgb}{0.800000,0.800000,0.800000}%
\pgfsetstrokecolor{currentstroke}%
\pgfsetdash{{2.960000pt}{1.280000pt}}{0.000000pt}%
\pgfpathmoveto{\pgfqpoint{0.378629in}{1.506144in}}%
\pgfpathlineto{\pgfqpoint{3.450000in}{1.506144in}}%
\pgfusepath{stroke}%
\end{pgfscope}%
\begin{pgfscope}%
\definecolor{textcolor}{rgb}{0.150000,0.150000,0.150000}%
\pgfsetstrokecolor{textcolor}%
\pgfsetfillcolor{textcolor}%
\pgftext[x=0.241822in,y=1.465998in,left,base]{\color{textcolor}\rmfamily\fontsize{8.330000}{9.996000}\selectfont \(\displaystyle 0\)}%
\end{pgfscope}%
\begin{pgfscope}%
\pgfpathrectangle{\pgfqpoint{0.378629in}{0.330514in}}{\pgfqpoint{3.071371in}{2.319486in}}%
\pgfusepath{clip}%
\pgfsetbuttcap%
\pgfsetroundjoin%
\pgfsetlinewidth{0.803000pt}%
\definecolor{currentstroke}{rgb}{0.800000,0.800000,0.800000}%
\pgfsetstrokecolor{currentstroke}%
\pgfsetdash{{2.960000pt}{1.280000pt}}{0.000000pt}%
\pgfpathmoveto{\pgfqpoint{0.378629in}{1.823882in}}%
\pgfpathlineto{\pgfqpoint{3.450000in}{1.823882in}}%
\pgfusepath{stroke}%
\end{pgfscope}%
\begin{pgfscope}%
\definecolor{textcolor}{rgb}{0.150000,0.150000,0.150000}%
\pgfsetstrokecolor{textcolor}%
\pgfsetfillcolor{textcolor}%
\pgftext[x=0.241822in,y=1.783736in,left,base]{\color{textcolor}\rmfamily\fontsize{8.330000}{9.996000}\selectfont \(\displaystyle 2\)}%
\end{pgfscope}%
\begin{pgfscope}%
\pgfpathrectangle{\pgfqpoint{0.378629in}{0.330514in}}{\pgfqpoint{3.071371in}{2.319486in}}%
\pgfusepath{clip}%
\pgfsetbuttcap%
\pgfsetroundjoin%
\pgfsetlinewidth{0.803000pt}%
\definecolor{currentstroke}{rgb}{0.800000,0.800000,0.800000}%
\pgfsetstrokecolor{currentstroke}%
\pgfsetdash{{2.960000pt}{1.280000pt}}{0.000000pt}%
\pgfpathmoveto{\pgfqpoint{0.378629in}{2.141620in}}%
\pgfpathlineto{\pgfqpoint{3.450000in}{2.141620in}}%
\pgfusepath{stroke}%
\end{pgfscope}%
\begin{pgfscope}%
\definecolor{textcolor}{rgb}{0.150000,0.150000,0.150000}%
\pgfsetstrokecolor{textcolor}%
\pgfsetfillcolor{textcolor}%
\pgftext[x=0.241822in,y=2.101474in,left,base]{\color{textcolor}\rmfamily\fontsize{8.330000}{9.996000}\selectfont \(\displaystyle 4\)}%
\end{pgfscope}%
\begin{pgfscope}%
\pgfpathrectangle{\pgfqpoint{0.378629in}{0.330514in}}{\pgfqpoint{3.071371in}{2.319486in}}%
\pgfusepath{clip}%
\pgfsetbuttcap%
\pgfsetroundjoin%
\pgfsetlinewidth{0.803000pt}%
\definecolor{currentstroke}{rgb}{0.800000,0.800000,0.800000}%
\pgfsetstrokecolor{currentstroke}%
\pgfsetdash{{2.960000pt}{1.280000pt}}{0.000000pt}%
\pgfpathmoveto{\pgfqpoint{0.378629in}{2.459357in}}%
\pgfpathlineto{\pgfqpoint{3.450000in}{2.459357in}}%
\pgfusepath{stroke}%
\end{pgfscope}%
\begin{pgfscope}%
\definecolor{textcolor}{rgb}{0.150000,0.150000,0.150000}%
\pgfsetstrokecolor{textcolor}%
\pgfsetfillcolor{textcolor}%
\pgftext[x=0.241822in,y=2.419211in,left,base]{\color{textcolor}\rmfamily\fontsize{8.330000}{9.996000}\selectfont \(\displaystyle 6\)}%
\end{pgfscope}%
\begin{pgfscope}%
\pgfpathrectangle{\pgfqpoint{0.378629in}{0.330514in}}{\pgfqpoint{3.071371in}{2.319486in}}%
\pgfusepath{clip}%
\pgfsetbuttcap%
\pgfsetroundjoin%
\pgfsetlinewidth{1.405250pt}%
\definecolor{currentstroke}{rgb}{0.940823,0.940823,0.940823}%
\pgfsetstrokecolor{currentstroke}%
\pgfsetdash{}{0pt}%
\pgfpathmoveto{\pgfqpoint{0.378629in}{0.474696in}}%
\pgfpathlineto{\pgfqpoint{0.486885in}{0.478056in}}%
\pgfpathlineto{\pgfqpoint{0.890524in}{0.441531in}}%
\pgfpathlineto{\pgfqpoint{0.999340in}{0.437584in}}%
\pgfpathlineto{\pgfqpoint{1.402419in}{0.436570in}}%
\pgfpathlineto{\pgfqpoint{1.511665in}{0.434554in}}%
\pgfpathlineto{\pgfqpoint{1.914314in}{0.424103in}}%
\pgfpathlineto{\pgfqpoint{2.313314in}{0.425116in}}%
\pgfpathlineto{\pgfqpoint{2.426210in}{0.425116in}}%
\pgfpathlineto{\pgfqpoint{2.826419in}{0.436523in}}%
\pgfpathlineto{\pgfqpoint{2.938105in}{0.436251in}}%
\pgfpathlineto{\pgfqpoint{3.049454in}{0.425884in}}%
\pgfpathlineto{\pgfqpoint{3.450000in}{0.409803in}}%
\pgfusepath{stroke}%
\end{pgfscope}%
\begin{pgfscope}%
\pgfpathrectangle{\pgfqpoint{0.378629in}{0.330514in}}{\pgfqpoint{3.071371in}{2.319486in}}%
\pgfusepath{clip}%
\pgfsetbuttcap%
\pgfsetroundjoin%
\pgfsetlinewidth{1.405250pt}%
\definecolor{currentstroke}{rgb}{0.850119,0.850119,0.850119}%
\pgfsetstrokecolor{currentstroke}%
\pgfsetdash{}{0pt}%
\pgfpathmoveto{\pgfqpoint{0.378629in}{0.705080in}}%
\pgfpathlineto{\pgfqpoint{0.796187in}{0.718039in}}%
\pgfpathlineto{\pgfqpoint{0.890524in}{0.709503in}}%
\pgfpathlineto{\pgfqpoint{1.305864in}{0.694438in}}%
\pgfpathlineto{\pgfqpoint{1.402419in}{0.694195in}}%
\pgfpathlineto{\pgfqpoint{1.823796in}{0.686419in}}%
\pgfpathlineto{\pgfqpoint{1.914314in}{0.684070in}}%
\pgfpathlineto{\pgfqpoint{2.004012in}{0.684298in}}%
\pgfpathlineto{\pgfqpoint{2.426210in}{0.684298in}}%
\pgfpathlineto{\pgfqpoint{2.516179in}{0.686862in}}%
\pgfpathlineto{\pgfqpoint{2.938105in}{0.685833in}}%
\pgfpathlineto{\pgfqpoint{3.358756in}{0.646668in}}%
\pgfpathlineto{\pgfqpoint{3.450000in}{0.643005in}}%
\pgfusepath{stroke}%
\end{pgfscope}%
\begin{pgfscope}%
\pgfpathrectangle{\pgfqpoint{0.378629in}{0.330514in}}{\pgfqpoint{3.071371in}{2.319486in}}%
\pgfusepath{clip}%
\pgfsetbuttcap%
\pgfsetroundjoin%
\pgfsetlinewidth{1.405250pt}%
\definecolor{currentstroke}{rgb}{0.739377,0.739377,0.739377}%
\pgfsetstrokecolor{currentstroke}%
\pgfsetdash{}{0pt}%
\pgfpathmoveto{\pgfqpoint{0.378629in}{0.979087in}}%
\pgfpathlineto{\pgfqpoint{0.678758in}{0.988402in}}%
\pgfpathlineto{\pgfqpoint{0.890524in}{0.980702in}}%
\pgfpathlineto{\pgfqpoint{1.101662in}{0.968158in}}%
\pgfpathlineto{\pgfqpoint{1.402419in}{0.957264in}}%
\pgfpathlineto{\pgfqpoint{1.611138in}{0.937289in}}%
\pgfpathlineto{\pgfqpoint{1.914314in}{0.931467in}}%
\pgfpathlineto{\pgfqpoint{2.213045in}{0.931467in}}%
\pgfpathlineto{\pgfqpoint{2.426210in}{0.944157in}}%
\pgfpathlineto{\pgfqpoint{2.726441in}{0.943394in}}%
\pgfpathlineto{\pgfqpoint{2.938105in}{0.972165in}}%
\pgfpathlineto{\pgfqpoint{3.146709in}{0.960159in}}%
\pgfpathlineto{\pgfqpoint{3.450000in}{0.931920in}}%
\pgfusepath{stroke}%
\end{pgfscope}%
\begin{pgfscope}%
\pgfpathrectangle{\pgfqpoint{0.378629in}{0.330514in}}{\pgfqpoint{3.071371in}{2.319486in}}%
\pgfusepath{clip}%
\pgfsetbuttcap%
\pgfsetroundjoin%
\pgfsetlinewidth{1.405250pt}%
\definecolor{currentstroke}{rgb}{0.586082,0.586082,0.586082}%
\pgfsetstrokecolor{currentstroke}%
\pgfsetdash{}{0pt}%
\pgfpathmoveto{\pgfqpoint{0.378629in}{1.274791in}}%
\pgfpathlineto{\pgfqpoint{0.383186in}{1.274063in}}%
\pgfpathlineto{\pgfqpoint{0.890524in}{1.254088in}}%
\pgfpathlineto{\pgfqpoint{0.892094in}{1.253806in}}%
\pgfpathlineto{\pgfqpoint{1.402419in}{1.224067in}}%
\pgfpathlineto{\pgfqpoint{1.405522in}{1.222972in}}%
\pgfpathlineto{\pgfqpoint{1.914314in}{1.174065in}}%
\pgfpathlineto{\pgfqpoint{2.426210in}{1.204293in}}%
\pgfpathlineto{\pgfqpoint{2.931974in}{1.274310in}}%
\pgfpathlineto{\pgfqpoint{2.938105in}{1.276428in}}%
\pgfpathlineto{\pgfqpoint{3.165614in}{1.260780in}}%
\pgfpathlineto{\pgfqpoint{3.442603in}{1.244837in}}%
\pgfpathlineto{\pgfqpoint{3.450000in}{1.244148in}}%
\pgfusepath{stroke}%
\end{pgfscope}%
\begin{pgfscope}%
\pgfpathrectangle{\pgfqpoint{0.378629in}{0.330514in}}{\pgfqpoint{3.071371in}{2.319486in}}%
\pgfusepath{clip}%
\pgfsetbuttcap%
\pgfsetroundjoin%
\pgfsetlinewidth{1.405250pt}%
\definecolor{currentstroke}{rgb}{0.448443,0.448443,0.448443}%
\pgfsetstrokecolor{currentstroke}%
\pgfsetdash{}{0pt}%
\pgfpathmoveto{\pgfqpoint{0.378629in}{1.738159in}}%
\pgfpathlineto{\pgfqpoint{0.686981in}{1.688914in}}%
\pgfpathlineto{\pgfqpoint{0.890524in}{1.680900in}}%
\pgfpathlineto{\pgfqpoint{1.206140in}{1.624175in}}%
\pgfpathlineto{\pgfqpoint{1.402419in}{1.612737in}}%
\pgfpathlineto{\pgfqpoint{1.715761in}{1.502196in}}%
\pgfpathlineto{\pgfqpoint{1.914314in}{1.483110in}}%
\pgfpathlineto{\pgfqpoint{2.114078in}{1.494907in}}%
\pgfpathlineto{\pgfqpoint{2.426210in}{1.570173in}}%
\pgfpathlineto{\pgfqpoint{2.625450in}{1.597755in}}%
\pgfpathlineto{\pgfqpoint{2.938105in}{1.705804in}}%
\pgfpathlineto{\pgfqpoint{3.141915in}{1.691785in}}%
\pgfpathlineto{\pgfqpoint{3.450000in}{1.695866in}}%
\pgfusepath{stroke}%
\end{pgfscope}%
\begin{pgfscope}%
\pgfpathrectangle{\pgfqpoint{0.378629in}{0.330514in}}{\pgfqpoint{3.071371in}{2.319486in}}%
\pgfusepath{clip}%
\pgfsetbuttcap%
\pgfsetroundjoin%
\pgfsetlinewidth{1.405250pt}%
\definecolor{currentstroke}{rgb}{0.317416,0.317416,0.317416}%
\pgfsetstrokecolor{currentstroke}%
\pgfsetdash{}{0pt}%
\pgfpathmoveto{\pgfqpoint{0.378629in}{2.171524in}}%
\pgfpathlineto{\pgfqpoint{0.483512in}{2.153717in}}%
\pgfpathlineto{\pgfqpoint{0.890524in}{2.089086in}}%
\pgfpathlineto{\pgfqpoint{0.993527in}{2.059570in}}%
\pgfpathlineto{\pgfqpoint{1.402419in}{1.982939in}}%
\pgfpathlineto{\pgfqpoint{1.516681in}{1.908848in}}%
\pgfpathlineto{\pgfqpoint{1.914314in}{1.770210in}}%
\pgfpathlineto{\pgfqpoint{2.317954in}{1.868471in}}%
\pgfpathlineto{\pgfqpoint{2.426210in}{1.913745in}}%
\pgfpathlineto{\pgfqpoint{2.826959in}{2.051740in}}%
\pgfpathlineto{\pgfqpoint{2.938105in}{2.076171in}}%
\pgfpathlineto{\pgfqpoint{3.343488in}{2.084585in}}%
\pgfpathlineto{\pgfqpoint{3.450000in}{2.093036in}}%
\pgfusepath{stroke}%
\end{pgfscope}%
\begin{pgfscope}%
\pgfpathrectangle{\pgfqpoint{0.378629in}{0.330514in}}{\pgfqpoint{3.071371in}{2.319486in}}%
\pgfusepath{clip}%
\pgfsetbuttcap%
\pgfsetroundjoin%
\pgfsetlinewidth{1.405250pt}%
\definecolor{currentstroke}{rgb}{0.141115,0.141115,0.141115}%
\pgfsetstrokecolor{currentstroke}%
\pgfsetdash{}{0pt}%
\pgfpathmoveto{\pgfqpoint{0.378629in}{2.545333in}}%
\pgfpathlineto{\pgfqpoint{0.796598in}{2.474370in}}%
\pgfpathlineto{\pgfqpoint{0.890524in}{2.459455in}}%
\pgfpathlineto{\pgfqpoint{1.305658in}{2.340496in}}%
\pgfpathlineto{\pgfqpoint{1.402419in}{2.322362in}}%
\pgfpathlineto{\pgfqpoint{1.821381in}{2.050696in}}%
\pgfpathlineto{\pgfqpoint{1.914314in}{2.018293in}}%
\pgfpathlineto{\pgfqpoint{2.008652in}{2.041259in}}%
\pgfpathlineto{\pgfqpoint{2.426210in}{2.215888in}}%
\pgfpathlineto{\pgfqpoint{2.513874in}{2.246075in}}%
\pgfpathlineto{\pgfqpoint{2.938105in}{2.339328in}}%
\pgfpathlineto{\pgfqpoint{3.025540in}{2.341143in}}%
\pgfpathlineto{\pgfqpoint{3.450000in}{2.374822in}}%
\pgfusepath{stroke}%
\end{pgfscope}%
\begin{pgfscope}%
\pgfsetrectcap%
\pgfsetmiterjoin%
\pgfsetlinewidth{1.003750pt}%
\definecolor{currentstroke}{rgb}{0.400000,0.400000,0.400000}%
\pgfsetstrokecolor{currentstroke}%
\pgfsetdash{}{0pt}%
\pgfpathmoveto{\pgfqpoint{0.378629in}{0.330514in}}%
\pgfpathlineto{\pgfqpoint{0.378629in}{2.650000in}}%
\pgfusepath{stroke}%
\end{pgfscope}%
\begin{pgfscope}%
\pgfsetrectcap%
\pgfsetmiterjoin%
\pgfsetlinewidth{1.003750pt}%
\definecolor{currentstroke}{rgb}{0.400000,0.400000,0.400000}%
\pgfsetstrokecolor{currentstroke}%
\pgfsetdash{}{0pt}%
\pgfpathmoveto{\pgfqpoint{0.378629in}{0.330514in}}%
\pgfpathlineto{\pgfqpoint{3.450000in}{0.330514in}}%
\pgfusepath{stroke}%
\end{pgfscope}%
\end{pgfpicture}%
\makeatother%
\endgroup%


        \caption{Exemplo de curvas de nível}
        \label{fig:contorno:base}
    \end{figure}


\subsection{Escala de Cores} \label{sec:contorno:cmap}

    Por padrão, o \pyline{tricontour} usa uma escala de tons de cinza para representar a variável dependente, no nosso caso o potencial, sendo branco o menor valor e preto, o maior. Isso pode ser facilmente mudado com o argumento \pyline{cmap}, que pode receber um objeto \pyref{https://matplotlib.org/3.1.0/api/_as_gen/matplotlib.colors.Colormap.html}{Colormap} personalizado, mas em geral os \href{https://matplotlib.org/3.1.0/gallery/color/colormap_reference.html}{mapas de cores do \texttt{Matplotlib}} já são o bastante e só precisam ser passados os seu nomes ou nome de sua versão invertida, com \pyline{'_r'} ao final. No código \ref{code:contorno:cmap}, as cores usadas aqui são da escala \pyline{'winter'} invertida.

    \begin{listing}[H]
        \caption{Curvas de nível com escala de cores}
        \label{code:contorno:cmap}

        \pyinclude[firstline=42, lastline=45]{recursos/contorno/contorno.py}
    \end{listing}

    Agora, para que as cores façam sentido, é preciso de uma legenda para elas, que normalmente são usadas com barras de cores na lateral. A função do \pyplot para isso é a \pyref{https://matplotlib.org/3.1.0/api/_as_gen/matplotlib.pyplot.colorbar.html}{colorbar}, como no código \ref{code:contorno:cmap}.

    \begin{figure}[H]
        \centering
        %% Creator: Matplotlib, PGF backend
%%
%% To include the figure in your LaTeX document, write
%%   \input{<filename>.pgf}
%%
%% Make sure the required packages are loaded in your preamble
%%   \usepackage{pgf}
%%
%% Figures using additional raster images can only be included by \input if
%% they are in the same directory as the main LaTeX file. For loading figures
%% from other directories you can use the `import` package
%%   \usepackage{import}
%% and then include the figures with
%%   \import{<path to file>}{<filename>.pgf}
%%
%% Matplotlib used the following preamble
%%   
%%       \usepackage[portuguese]{babel}
%%       \usepackage[T1]{fontenc}
%%       \usepackage[utf8]{inputenc}
%%   \usepackage{fontspec}
%%
\begingroup%
\makeatletter%
\begin{pgfpicture}%
\pgfpathrectangle{\pgfpointorigin}{\pgfqpoint{3.600000in}{2.800000in}}%
\pgfusepath{use as bounding box, clip}%
\begin{pgfscope}%
\pgfsetbuttcap%
\pgfsetmiterjoin%
\pgfsetlinewidth{0.000000pt}%
\definecolor{currentstroke}{rgb}{0.000000,0.000000,0.000000}%
\pgfsetstrokecolor{currentstroke}%
\pgfsetstrokeopacity{0.000000}%
\pgfsetdash{}{0pt}%
\pgfpathmoveto{\pgfqpoint{0.000000in}{0.000000in}}%
\pgfpathlineto{\pgfqpoint{3.600000in}{0.000000in}}%
\pgfpathlineto{\pgfqpoint{3.600000in}{2.800000in}}%
\pgfpathlineto{\pgfqpoint{0.000000in}{2.800000in}}%
\pgfpathclose%
\pgfusepath{}%
\end{pgfscope}%
\begin{pgfscope}%
\pgfsetbuttcap%
\pgfsetmiterjoin%
\pgfsetlinewidth{0.000000pt}%
\definecolor{currentstroke}{rgb}{0.000000,0.000000,0.000000}%
\pgfsetstrokecolor{currentstroke}%
\pgfsetstrokeopacity{0.000000}%
\pgfsetdash{}{0pt}%
\pgfpathmoveto{\pgfqpoint{0.378629in}{0.330514in}}%
\pgfpathlineto{\pgfqpoint{2.835726in}{0.330514in}}%
\pgfpathlineto{\pgfqpoint{2.835726in}{2.609854in}}%
\pgfpathlineto{\pgfqpoint{0.378629in}{2.609854in}}%
\pgfpathclose%
\pgfusepath{}%
\end{pgfscope}%
\begin{pgfscope}%
\pgfpathrectangle{\pgfqpoint{0.378629in}{0.330514in}}{\pgfqpoint{2.457097in}{2.279340in}}%
\pgfusepath{clip}%
\pgfsetbuttcap%
\pgfsetroundjoin%
\pgfsetlinewidth{0.803000pt}%
\definecolor{currentstroke}{rgb}{0.800000,0.800000,0.800000}%
\pgfsetstrokecolor{currentstroke}%
\pgfsetdash{{2.960000pt}{1.280000pt}}{0.000000pt}%
\pgfpathmoveto{\pgfqpoint{0.583387in}{0.330514in}}%
\pgfpathlineto{\pgfqpoint{0.583387in}{2.609854in}}%
\pgfusepath{stroke}%
\end{pgfscope}%
\begin{pgfscope}%
\definecolor{textcolor}{rgb}{0.150000,0.150000,0.150000}%
\pgfsetstrokecolor{textcolor}%
\pgfsetfillcolor{textcolor}%
\pgftext[x=0.583387in,y=0.252737in,,top]{\color{textcolor}\rmfamily\fontsize{8.330000}{9.996000}\selectfont \(\displaystyle -10\)}%
\end{pgfscope}%
\begin{pgfscope}%
\pgfpathrectangle{\pgfqpoint{0.378629in}{0.330514in}}{\pgfqpoint{2.457097in}{2.279340in}}%
\pgfusepath{clip}%
\pgfsetbuttcap%
\pgfsetroundjoin%
\pgfsetlinewidth{0.803000pt}%
\definecolor{currentstroke}{rgb}{0.800000,0.800000,0.800000}%
\pgfsetstrokecolor{currentstroke}%
\pgfsetdash{{2.960000pt}{1.280000pt}}{0.000000pt}%
\pgfpathmoveto{\pgfqpoint{1.095282in}{0.330514in}}%
\pgfpathlineto{\pgfqpoint{1.095282in}{2.609854in}}%
\pgfusepath{stroke}%
\end{pgfscope}%
\begin{pgfscope}%
\definecolor{textcolor}{rgb}{0.150000,0.150000,0.150000}%
\pgfsetstrokecolor{textcolor}%
\pgfsetfillcolor{textcolor}%
\pgftext[x=1.095282in,y=0.252737in,,top]{\color{textcolor}\rmfamily\fontsize{8.330000}{9.996000}\selectfont \(\displaystyle -5\)}%
\end{pgfscope}%
\begin{pgfscope}%
\pgfpathrectangle{\pgfqpoint{0.378629in}{0.330514in}}{\pgfqpoint{2.457097in}{2.279340in}}%
\pgfusepath{clip}%
\pgfsetbuttcap%
\pgfsetroundjoin%
\pgfsetlinewidth{0.803000pt}%
\definecolor{currentstroke}{rgb}{0.800000,0.800000,0.800000}%
\pgfsetstrokecolor{currentstroke}%
\pgfsetdash{{2.960000pt}{1.280000pt}}{0.000000pt}%
\pgfpathmoveto{\pgfqpoint{1.607177in}{0.330514in}}%
\pgfpathlineto{\pgfqpoint{1.607177in}{2.609854in}}%
\pgfusepath{stroke}%
\end{pgfscope}%
\begin{pgfscope}%
\definecolor{textcolor}{rgb}{0.150000,0.150000,0.150000}%
\pgfsetstrokecolor{textcolor}%
\pgfsetfillcolor{textcolor}%
\pgftext[x=1.607177in,y=0.252737in,,top]{\color{textcolor}\rmfamily\fontsize{8.330000}{9.996000}\selectfont \(\displaystyle 0\)}%
\end{pgfscope}%
\begin{pgfscope}%
\pgfpathrectangle{\pgfqpoint{0.378629in}{0.330514in}}{\pgfqpoint{2.457097in}{2.279340in}}%
\pgfusepath{clip}%
\pgfsetbuttcap%
\pgfsetroundjoin%
\pgfsetlinewidth{0.803000pt}%
\definecolor{currentstroke}{rgb}{0.800000,0.800000,0.800000}%
\pgfsetstrokecolor{currentstroke}%
\pgfsetdash{{2.960000pt}{1.280000pt}}{0.000000pt}%
\pgfpathmoveto{\pgfqpoint{2.119072in}{0.330514in}}%
\pgfpathlineto{\pgfqpoint{2.119072in}{2.609854in}}%
\pgfusepath{stroke}%
\end{pgfscope}%
\begin{pgfscope}%
\definecolor{textcolor}{rgb}{0.150000,0.150000,0.150000}%
\pgfsetstrokecolor{textcolor}%
\pgfsetfillcolor{textcolor}%
\pgftext[x=2.119072in,y=0.252737in,,top]{\color{textcolor}\rmfamily\fontsize{8.330000}{9.996000}\selectfont \(\displaystyle 5\)}%
\end{pgfscope}%
\begin{pgfscope}%
\pgfpathrectangle{\pgfqpoint{0.378629in}{0.330514in}}{\pgfqpoint{2.457097in}{2.279340in}}%
\pgfusepath{clip}%
\pgfsetbuttcap%
\pgfsetroundjoin%
\pgfsetlinewidth{0.803000pt}%
\definecolor{currentstroke}{rgb}{0.800000,0.800000,0.800000}%
\pgfsetstrokecolor{currentstroke}%
\pgfsetdash{{2.960000pt}{1.280000pt}}{0.000000pt}%
\pgfpathmoveto{\pgfqpoint{2.630968in}{0.330514in}}%
\pgfpathlineto{\pgfqpoint{2.630968in}{2.609854in}}%
\pgfusepath{stroke}%
\end{pgfscope}%
\begin{pgfscope}%
\definecolor{textcolor}{rgb}{0.150000,0.150000,0.150000}%
\pgfsetstrokecolor{textcolor}%
\pgfsetfillcolor{textcolor}%
\pgftext[x=2.630968in,y=0.252737in,,top]{\color{textcolor}\rmfamily\fontsize{8.330000}{9.996000}\selectfont \(\displaystyle 10\)}%
\end{pgfscope}%
\begin{pgfscope}%
\pgfpathrectangle{\pgfqpoint{0.378629in}{0.330514in}}{\pgfqpoint{2.457097in}{2.279340in}}%
\pgfusepath{clip}%
\pgfsetbuttcap%
\pgfsetroundjoin%
\pgfsetlinewidth{0.803000pt}%
\definecolor{currentstroke}{rgb}{0.800000,0.800000,0.800000}%
\pgfsetstrokecolor{currentstroke}%
\pgfsetdash{{2.960000pt}{1.280000pt}}{0.000000pt}%
\pgfpathmoveto{\pgfqpoint{0.378629in}{0.549081in}}%
\pgfpathlineto{\pgfqpoint{2.835726in}{0.549081in}}%
\pgfusepath{stroke}%
\end{pgfscope}%
\begin{pgfscope}%
\definecolor{textcolor}{rgb}{0.150000,0.150000,0.150000}%
\pgfsetstrokecolor{textcolor}%
\pgfsetfillcolor{textcolor}%
\pgftext[x=0.150000in,y=0.508935in,left,base]{\color{textcolor}\rmfamily\fontsize{8.330000}{9.996000}\selectfont \(\displaystyle -6\)}%
\end{pgfscope}%
\begin{pgfscope}%
\pgfpathrectangle{\pgfqpoint{0.378629in}{0.330514in}}{\pgfqpoint{2.457097in}{2.279340in}}%
\pgfusepath{clip}%
\pgfsetbuttcap%
\pgfsetroundjoin%
\pgfsetlinewidth{0.803000pt}%
\definecolor{currentstroke}{rgb}{0.800000,0.800000,0.800000}%
\pgfsetstrokecolor{currentstroke}%
\pgfsetdash{{2.960000pt}{1.280000pt}}{0.000000pt}%
\pgfpathmoveto{\pgfqpoint{0.378629in}{0.861319in}}%
\pgfpathlineto{\pgfqpoint{2.835726in}{0.861319in}}%
\pgfusepath{stroke}%
\end{pgfscope}%
\begin{pgfscope}%
\definecolor{textcolor}{rgb}{0.150000,0.150000,0.150000}%
\pgfsetstrokecolor{textcolor}%
\pgfsetfillcolor{textcolor}%
\pgftext[x=0.150000in,y=0.821173in,left,base]{\color{textcolor}\rmfamily\fontsize{8.330000}{9.996000}\selectfont \(\displaystyle -4\)}%
\end{pgfscope}%
\begin{pgfscope}%
\pgfpathrectangle{\pgfqpoint{0.378629in}{0.330514in}}{\pgfqpoint{2.457097in}{2.279340in}}%
\pgfusepath{clip}%
\pgfsetbuttcap%
\pgfsetroundjoin%
\pgfsetlinewidth{0.803000pt}%
\definecolor{currentstroke}{rgb}{0.800000,0.800000,0.800000}%
\pgfsetstrokecolor{currentstroke}%
\pgfsetdash{{2.960000pt}{1.280000pt}}{0.000000pt}%
\pgfpathmoveto{\pgfqpoint{0.378629in}{1.173558in}}%
\pgfpathlineto{\pgfqpoint{2.835726in}{1.173558in}}%
\pgfusepath{stroke}%
\end{pgfscope}%
\begin{pgfscope}%
\definecolor{textcolor}{rgb}{0.150000,0.150000,0.150000}%
\pgfsetstrokecolor{textcolor}%
\pgfsetfillcolor{textcolor}%
\pgftext[x=0.150000in,y=1.133412in,left,base]{\color{textcolor}\rmfamily\fontsize{8.330000}{9.996000}\selectfont \(\displaystyle -2\)}%
\end{pgfscope}%
\begin{pgfscope}%
\pgfpathrectangle{\pgfqpoint{0.378629in}{0.330514in}}{\pgfqpoint{2.457097in}{2.279340in}}%
\pgfusepath{clip}%
\pgfsetbuttcap%
\pgfsetroundjoin%
\pgfsetlinewidth{0.803000pt}%
\definecolor{currentstroke}{rgb}{0.800000,0.800000,0.800000}%
\pgfsetstrokecolor{currentstroke}%
\pgfsetdash{{2.960000pt}{1.280000pt}}{0.000000pt}%
\pgfpathmoveto{\pgfqpoint{0.378629in}{1.485796in}}%
\pgfpathlineto{\pgfqpoint{2.835726in}{1.485796in}}%
\pgfusepath{stroke}%
\end{pgfscope}%
\begin{pgfscope}%
\definecolor{textcolor}{rgb}{0.150000,0.150000,0.150000}%
\pgfsetstrokecolor{textcolor}%
\pgfsetfillcolor{textcolor}%
\pgftext[x=0.241822in,y=1.445650in,left,base]{\color{textcolor}\rmfamily\fontsize{8.330000}{9.996000}\selectfont \(\displaystyle 0\)}%
\end{pgfscope}%
\begin{pgfscope}%
\pgfpathrectangle{\pgfqpoint{0.378629in}{0.330514in}}{\pgfqpoint{2.457097in}{2.279340in}}%
\pgfusepath{clip}%
\pgfsetbuttcap%
\pgfsetroundjoin%
\pgfsetlinewidth{0.803000pt}%
\definecolor{currentstroke}{rgb}{0.800000,0.800000,0.800000}%
\pgfsetstrokecolor{currentstroke}%
\pgfsetdash{{2.960000pt}{1.280000pt}}{0.000000pt}%
\pgfpathmoveto{\pgfqpoint{0.378629in}{1.798034in}}%
\pgfpathlineto{\pgfqpoint{2.835726in}{1.798034in}}%
\pgfusepath{stroke}%
\end{pgfscope}%
\begin{pgfscope}%
\definecolor{textcolor}{rgb}{0.150000,0.150000,0.150000}%
\pgfsetstrokecolor{textcolor}%
\pgfsetfillcolor{textcolor}%
\pgftext[x=0.241822in,y=1.757888in,left,base]{\color{textcolor}\rmfamily\fontsize{8.330000}{9.996000}\selectfont \(\displaystyle 2\)}%
\end{pgfscope}%
\begin{pgfscope}%
\pgfpathrectangle{\pgfqpoint{0.378629in}{0.330514in}}{\pgfqpoint{2.457097in}{2.279340in}}%
\pgfusepath{clip}%
\pgfsetbuttcap%
\pgfsetroundjoin%
\pgfsetlinewidth{0.803000pt}%
\definecolor{currentstroke}{rgb}{0.800000,0.800000,0.800000}%
\pgfsetstrokecolor{currentstroke}%
\pgfsetdash{{2.960000pt}{1.280000pt}}{0.000000pt}%
\pgfpathmoveto{\pgfqpoint{0.378629in}{2.110273in}}%
\pgfpathlineto{\pgfqpoint{2.835726in}{2.110273in}}%
\pgfusepath{stroke}%
\end{pgfscope}%
\begin{pgfscope}%
\definecolor{textcolor}{rgb}{0.150000,0.150000,0.150000}%
\pgfsetstrokecolor{textcolor}%
\pgfsetfillcolor{textcolor}%
\pgftext[x=0.241822in,y=2.070127in,left,base]{\color{textcolor}\rmfamily\fontsize{8.330000}{9.996000}\selectfont \(\displaystyle 4\)}%
\end{pgfscope}%
\begin{pgfscope}%
\pgfpathrectangle{\pgfqpoint{0.378629in}{0.330514in}}{\pgfqpoint{2.457097in}{2.279340in}}%
\pgfusepath{clip}%
\pgfsetbuttcap%
\pgfsetroundjoin%
\pgfsetlinewidth{0.803000pt}%
\definecolor{currentstroke}{rgb}{0.800000,0.800000,0.800000}%
\pgfsetstrokecolor{currentstroke}%
\pgfsetdash{{2.960000pt}{1.280000pt}}{0.000000pt}%
\pgfpathmoveto{\pgfqpoint{0.378629in}{2.422511in}}%
\pgfpathlineto{\pgfqpoint{2.835726in}{2.422511in}}%
\pgfusepath{stroke}%
\end{pgfscope}%
\begin{pgfscope}%
\definecolor{textcolor}{rgb}{0.150000,0.150000,0.150000}%
\pgfsetstrokecolor{textcolor}%
\pgfsetfillcolor{textcolor}%
\pgftext[x=0.241822in,y=2.382365in,left,base]{\color{textcolor}\rmfamily\fontsize{8.330000}{9.996000}\selectfont \(\displaystyle 6\)}%
\end{pgfscope}%
\begin{pgfscope}%
\pgfpathrectangle{\pgfqpoint{0.378629in}{0.330514in}}{\pgfqpoint{2.457097in}{2.279340in}}%
\pgfusepath{clip}%
\pgfsetbuttcap%
\pgfsetroundjoin%
\pgfsetlinewidth{1.405250pt}%
\definecolor{currentstroke}{rgb}{0.000000,0.874510,0.562745}%
\pgfsetstrokecolor{currentstroke}%
\pgfsetdash{}{0pt}%
\pgfpathmoveto{\pgfqpoint{0.378629in}{0.472201in}}%
\pgfpathlineto{\pgfqpoint{0.465233in}{0.475502in}}%
\pgfpathlineto{\pgfqpoint{0.788145in}{0.439610in}}%
\pgfpathlineto{\pgfqpoint{0.875198in}{0.435731in}}%
\pgfpathlineto{\pgfqpoint{1.197661in}{0.434734in}}%
\pgfpathlineto{\pgfqpoint{1.285058in}{0.432753in}}%
\pgfpathlineto{\pgfqpoint{1.607177in}{0.422483in}}%
\pgfpathlineto{\pgfqpoint{1.926377in}{0.423478in}}%
\pgfpathlineto{\pgfqpoint{2.016693in}{0.423478in}}%
\pgfpathlineto{\pgfqpoint{2.336861in}{0.434688in}}%
\pgfpathlineto{\pgfqpoint{2.426210in}{0.434421in}}%
\pgfpathlineto{\pgfqpoint{2.515289in}{0.424233in}}%
\pgfpathlineto{\pgfqpoint{2.835726in}{0.408431in}}%
\pgfusepath{stroke}%
\end{pgfscope}%
\begin{pgfscope}%
\pgfpathrectangle{\pgfqpoint{0.378629in}{0.330514in}}{\pgfqpoint{2.457097in}{2.279340in}}%
\pgfusepath{clip}%
\pgfsetbuttcap%
\pgfsetroundjoin%
\pgfsetlinewidth{1.405250pt}%
\definecolor{currentstroke}{rgb}{0.000000,0.749020,0.625490}%
\pgfsetstrokecolor{currentstroke}%
\pgfsetdash{}{0pt}%
\pgfpathmoveto{\pgfqpoint{0.378629in}{0.698597in}}%
\pgfpathlineto{\pgfqpoint{0.712675in}{0.711332in}}%
\pgfpathlineto{\pgfqpoint{0.788145in}{0.702943in}}%
\pgfpathlineto{\pgfqpoint{1.120417in}{0.688139in}}%
\pgfpathlineto{\pgfqpoint{1.197661in}{0.687901in}}%
\pgfpathlineto{\pgfqpoint{1.534763in}{0.680259in}}%
\pgfpathlineto{\pgfqpoint{1.607177in}{0.677950in}}%
\pgfpathlineto{\pgfqpoint{1.678935in}{0.678174in}}%
\pgfpathlineto{\pgfqpoint{2.016693in}{0.678174in}}%
\pgfpathlineto{\pgfqpoint{2.088669in}{0.680694in}}%
\pgfpathlineto{\pgfqpoint{2.426210in}{0.679684in}}%
\pgfpathlineto{\pgfqpoint{2.762730in}{0.641196in}}%
\pgfpathlineto{\pgfqpoint{2.835726in}{0.637596in}}%
\pgfusepath{stroke}%
\end{pgfscope}%
\begin{pgfscope}%
\pgfpathrectangle{\pgfqpoint{0.378629in}{0.330514in}}{\pgfqpoint{2.457097in}{2.279340in}}%
\pgfusepath{clip}%
\pgfsetbuttcap%
\pgfsetroundjoin%
\pgfsetlinewidth{1.405250pt}%
\definecolor{currentstroke}{rgb}{0.000000,0.623529,0.688235}%
\pgfsetstrokecolor{currentstroke}%
\pgfsetdash{}{0pt}%
\pgfpathmoveto{\pgfqpoint{0.378629in}{0.967861in}}%
\pgfpathlineto{\pgfqpoint{0.618732in}{0.977015in}}%
\pgfpathlineto{\pgfqpoint{0.788145in}{0.969448in}}%
\pgfpathlineto{\pgfqpoint{0.957055in}{0.957122in}}%
\pgfpathlineto{\pgfqpoint{1.197661in}{0.946416in}}%
\pgfpathlineto{\pgfqpoint{1.364636in}{0.926787in}}%
\pgfpathlineto{\pgfqpoint{1.607177in}{0.921066in}}%
\pgfpathlineto{\pgfqpoint{1.846162in}{0.921066in}}%
\pgfpathlineto{\pgfqpoint{2.016693in}{0.933536in}}%
\pgfpathlineto{\pgfqpoint{2.256879in}{0.932786in}}%
\pgfpathlineto{\pgfqpoint{2.426210in}{0.961060in}}%
\pgfpathlineto{\pgfqpoint{2.593093in}{0.949261in}}%
\pgfpathlineto{\pgfqpoint{2.835726in}{0.921511in}}%
\pgfusepath{stroke}%
\end{pgfscope}%
\begin{pgfscope}%
\pgfpathrectangle{\pgfqpoint{0.378629in}{0.330514in}}{\pgfqpoint{2.457097in}{2.279340in}}%
\pgfusepath{clip}%
\pgfsetbuttcap%
\pgfsetroundjoin%
\pgfsetlinewidth{1.405250pt}%
\definecolor{currentstroke}{rgb}{0.000000,0.498039,0.750980}%
\pgfsetstrokecolor{currentstroke}%
\pgfsetdash{}{0pt}%
\pgfpathmoveto{\pgfqpoint{0.378629in}{1.258448in}}%
\pgfpathlineto{\pgfqpoint{0.382274in}{1.257732in}}%
\pgfpathlineto{\pgfqpoint{0.788145in}{1.238103in}}%
\pgfpathlineto{\pgfqpoint{0.789401in}{1.237825in}}%
\pgfpathlineto{\pgfqpoint{1.197661in}{1.208601in}}%
\pgfpathlineto{\pgfqpoint{1.200143in}{1.207526in}}%
\pgfpathlineto{\pgfqpoint{1.607177in}{1.159464in}}%
\pgfpathlineto{\pgfqpoint{2.016693in}{1.189170in}}%
\pgfpathlineto{\pgfqpoint{2.421305in}{1.257974in}}%
\pgfpathlineto{\pgfqpoint{2.426210in}{1.260056in}}%
\pgfpathlineto{\pgfqpoint{2.608217in}{1.244679in}}%
\pgfpathlineto{\pgfqpoint{2.829808in}{1.229012in}}%
\pgfpathlineto{\pgfqpoint{2.835726in}{1.228335in}}%
\pgfusepath{stroke}%
\end{pgfscope}%
\begin{pgfscope}%
\pgfpathrectangle{\pgfqpoint{0.378629in}{0.330514in}}{\pgfqpoint{2.457097in}{2.279340in}}%
\pgfusepath{clip}%
\pgfsetbuttcap%
\pgfsetroundjoin%
\pgfsetlinewidth{1.405250pt}%
\definecolor{currentstroke}{rgb}{0.000000,0.372549,0.813725}%
\pgfsetstrokecolor{currentstroke}%
\pgfsetdash{}{0pt}%
\pgfpathmoveto{\pgfqpoint{0.378629in}{1.713795in}}%
\pgfpathlineto{\pgfqpoint{0.625311in}{1.665403in}}%
\pgfpathlineto{\pgfqpoint{0.788145in}{1.657527in}}%
\pgfpathlineto{\pgfqpoint{1.040638in}{1.601784in}}%
\pgfpathlineto{\pgfqpoint{1.197661in}{1.590544in}}%
\pgfpathlineto{\pgfqpoint{1.448335in}{1.481917in}}%
\pgfpathlineto{\pgfqpoint{1.607177in}{1.463161in}}%
\pgfpathlineto{\pgfqpoint{1.766988in}{1.474754in}}%
\pgfpathlineto{\pgfqpoint{2.016693in}{1.548717in}}%
\pgfpathlineto{\pgfqpoint{2.176086in}{1.575822in}}%
\pgfpathlineto{\pgfqpoint{2.426210in}{1.682000in}}%
\pgfpathlineto{\pgfqpoint{2.589258in}{1.668224in}}%
\pgfpathlineto{\pgfqpoint{2.835726in}{1.672235in}}%
\pgfusepath{stroke}%
\end{pgfscope}%
\begin{pgfscope}%
\pgfpathrectangle{\pgfqpoint{0.378629in}{0.330514in}}{\pgfqpoint{2.457097in}{2.279340in}}%
\pgfusepath{clip}%
\pgfsetbuttcap%
\pgfsetroundjoin%
\pgfsetlinewidth{1.405250pt}%
\definecolor{currentstroke}{rgb}{0.000000,0.247059,0.876471}%
\pgfsetstrokecolor{currentstroke}%
\pgfsetdash{}{0pt}%
\pgfpathmoveto{\pgfqpoint{0.378629in}{2.139660in}}%
\pgfpathlineto{\pgfqpoint{0.462536in}{2.122161in}}%
\pgfpathlineto{\pgfqpoint{0.788145in}{2.058648in}}%
\pgfpathlineto{\pgfqpoint{0.870548in}{2.029643in}}%
\pgfpathlineto{\pgfqpoint{1.197661in}{1.954339in}}%
\pgfpathlineto{\pgfqpoint{1.289071in}{1.881530in}}%
\pgfpathlineto{\pgfqpoint{1.607177in}{1.745291in}}%
\pgfpathlineto{\pgfqpoint{1.930089in}{1.841852in}}%
\pgfpathlineto{\pgfqpoint{2.016693in}{1.886342in}}%
\pgfpathlineto{\pgfqpoint{2.337293in}{2.021948in}}%
\pgfpathlineto{\pgfqpoint{2.426210in}{2.045957in}}%
\pgfpathlineto{\pgfqpoint{2.750516in}{2.054225in}}%
\pgfpathlineto{\pgfqpoint{2.835726in}{2.062530in}}%
\pgfusepath{stroke}%
\end{pgfscope}%
\begin{pgfscope}%
\pgfpathrectangle{\pgfqpoint{0.378629in}{0.330514in}}{\pgfqpoint{2.457097in}{2.279340in}}%
\pgfusepath{clip}%
\pgfsetbuttcap%
\pgfsetroundjoin%
\pgfsetlinewidth{1.405250pt}%
\definecolor{currentstroke}{rgb}{0.000000,0.121569,0.939216}%
\pgfsetstrokecolor{currentstroke}%
\pgfsetdash{}{0pt}%
\pgfpathmoveto{\pgfqpoint{0.378629in}{2.506999in}}%
\pgfpathlineto{\pgfqpoint{0.713004in}{2.437264in}}%
\pgfpathlineto{\pgfqpoint{0.788145in}{2.422607in}}%
\pgfpathlineto{\pgfqpoint{1.120253in}{2.305707in}}%
\pgfpathlineto{\pgfqpoint{1.197661in}{2.287887in}}%
\pgfpathlineto{\pgfqpoint{1.532831in}{2.020922in}}%
\pgfpathlineto{\pgfqpoint{1.607177in}{1.989081in}}%
\pgfpathlineto{\pgfqpoint{1.682647in}{2.011649in}}%
\pgfpathlineto{\pgfqpoint{2.016693in}{2.183256in}}%
\pgfpathlineto{\pgfqpoint{2.086825in}{2.212920in}}%
\pgfpathlineto{\pgfqpoint{2.426210in}{2.304560in}}%
\pgfpathlineto{\pgfqpoint{2.496158in}{2.306343in}}%
\pgfpathlineto{\pgfqpoint{2.835726in}{2.339438in}}%
\pgfusepath{stroke}%
\end{pgfscope}%
\begin{pgfscope}%
\pgfsetrectcap%
\pgfsetmiterjoin%
\pgfsetlinewidth{1.003750pt}%
\definecolor{currentstroke}{rgb}{0.400000,0.400000,0.400000}%
\pgfsetstrokecolor{currentstroke}%
\pgfsetdash{}{0pt}%
\pgfpathmoveto{\pgfqpoint{0.378629in}{0.330514in}}%
\pgfpathlineto{\pgfqpoint{0.378629in}{2.609854in}}%
\pgfusepath{stroke}%
\end{pgfscope}%
\begin{pgfscope}%
\pgfsetrectcap%
\pgfsetmiterjoin%
\pgfsetlinewidth{1.003750pt}%
\definecolor{currentstroke}{rgb}{0.400000,0.400000,0.400000}%
\pgfsetstrokecolor{currentstroke}%
\pgfsetdash{}{0pt}%
\pgfpathmoveto{\pgfqpoint{0.378629in}{0.330514in}}%
\pgfpathlineto{\pgfqpoint{2.835726in}{0.330514in}}%
\pgfusepath{stroke}%
\end{pgfscope}%
\begin{pgfscope}%
\pgfpathrectangle{\pgfqpoint{2.989294in}{0.330514in}}{\pgfqpoint{0.113967in}{2.279340in}}%
\pgfusepath{clip}%
\pgfsetbuttcap%
\pgfsetmiterjoin%
\definecolor{currentfill}{rgb}{1.000000,1.000000,1.000000}%
\pgfsetfillcolor{currentfill}%
\pgfsetlinewidth{0.010037pt}%
\definecolor{currentstroke}{rgb}{1.000000,1.000000,1.000000}%
\pgfsetstrokecolor{currentstroke}%
\pgfsetdash{}{0pt}%
\pgfpathmoveto{\pgfqpoint{2.989294in}{0.330514in}}%
\pgfpathlineto{\pgfqpoint{2.989294in}{0.615432in}}%
\pgfpathlineto{\pgfqpoint{2.989294in}{2.324937in}}%
\pgfpathlineto{\pgfqpoint{2.989294in}{2.609854in}}%
\pgfpathlineto{\pgfqpoint{3.103261in}{2.609854in}}%
\pgfpathlineto{\pgfqpoint{3.103261in}{2.324937in}}%
\pgfpathlineto{\pgfqpoint{3.103261in}{0.615432in}}%
\pgfpathlineto{\pgfqpoint{3.103261in}{0.330514in}}%
\pgfpathclose%
\pgfusepath{stroke,fill}%
\end{pgfscope}%
\begin{pgfscope}%
\pgfpathrectangle{\pgfqpoint{2.989294in}{0.330514in}}{\pgfqpoint{0.113967in}{2.279340in}}%
\pgfusepath{clip}%
\pgfsetbuttcap%
\pgfsetroundjoin%
\pgfsetlinewidth{0.803000pt}%
\definecolor{currentstroke}{rgb}{0.800000,0.800000,0.800000}%
\pgfsetstrokecolor{currentstroke}%
\pgfsetdash{{2.960000pt}{1.280000pt}}{0.000000pt}%
\pgfpathmoveto{\pgfqpoint{2.989294in}{0.330514in}}%
\pgfpathlineto{\pgfqpoint{3.103261in}{0.330514in}}%
\pgfusepath{stroke}%
\end{pgfscope}%
\begin{pgfscope}%
\definecolor{textcolor}{rgb}{0.150000,0.150000,0.150000}%
\pgfsetstrokecolor{textcolor}%
\pgfsetfillcolor{textcolor}%
\pgftext[x=3.181039in,y=0.290368in,left,base]{\color{textcolor}\rmfamily\fontsize{8.330000}{9.996000}\selectfont \(\displaystyle 0.2\)}%
\end{pgfscope}%
\begin{pgfscope}%
\pgfpathrectangle{\pgfqpoint{2.989294in}{0.330514in}}{\pgfqpoint{0.113967in}{2.279340in}}%
\pgfusepath{clip}%
\pgfsetbuttcap%
\pgfsetroundjoin%
\pgfsetlinewidth{0.803000pt}%
\definecolor{currentstroke}{rgb}{0.800000,0.800000,0.800000}%
\pgfsetstrokecolor{currentstroke}%
\pgfsetdash{{2.960000pt}{1.280000pt}}{0.000000pt}%
\pgfpathmoveto{\pgfqpoint{2.989294in}{0.615432in}}%
\pgfpathlineto{\pgfqpoint{3.103261in}{0.615432in}}%
\pgfusepath{stroke}%
\end{pgfscope}%
\begin{pgfscope}%
\definecolor{textcolor}{rgb}{0.150000,0.150000,0.150000}%
\pgfsetstrokecolor{textcolor}%
\pgfsetfillcolor{textcolor}%
\pgftext[x=3.181039in,y=0.575286in,left,base]{\color{textcolor}\rmfamily\fontsize{8.330000}{9.996000}\selectfont \(\displaystyle 0.4\)}%
\end{pgfscope}%
\begin{pgfscope}%
\pgfpathrectangle{\pgfqpoint{2.989294in}{0.330514in}}{\pgfqpoint{0.113967in}{2.279340in}}%
\pgfusepath{clip}%
\pgfsetbuttcap%
\pgfsetroundjoin%
\pgfsetlinewidth{0.803000pt}%
\definecolor{currentstroke}{rgb}{0.800000,0.800000,0.800000}%
\pgfsetstrokecolor{currentstroke}%
\pgfsetdash{{2.960000pt}{1.280000pt}}{0.000000pt}%
\pgfpathmoveto{\pgfqpoint{2.989294in}{0.900349in}}%
\pgfpathlineto{\pgfqpoint{3.103261in}{0.900349in}}%
\pgfusepath{stroke}%
\end{pgfscope}%
\begin{pgfscope}%
\definecolor{textcolor}{rgb}{0.150000,0.150000,0.150000}%
\pgfsetstrokecolor{textcolor}%
\pgfsetfillcolor{textcolor}%
\pgftext[x=3.181039in,y=0.860203in,left,base]{\color{textcolor}\rmfamily\fontsize{8.330000}{9.996000}\selectfont \(\displaystyle 0.6\)}%
\end{pgfscope}%
\begin{pgfscope}%
\pgfpathrectangle{\pgfqpoint{2.989294in}{0.330514in}}{\pgfqpoint{0.113967in}{2.279340in}}%
\pgfusepath{clip}%
\pgfsetbuttcap%
\pgfsetroundjoin%
\pgfsetlinewidth{0.803000pt}%
\definecolor{currentstroke}{rgb}{0.800000,0.800000,0.800000}%
\pgfsetstrokecolor{currentstroke}%
\pgfsetdash{{2.960000pt}{1.280000pt}}{0.000000pt}%
\pgfpathmoveto{\pgfqpoint{2.989294in}{1.185267in}}%
\pgfpathlineto{\pgfqpoint{3.103261in}{1.185267in}}%
\pgfusepath{stroke}%
\end{pgfscope}%
\begin{pgfscope}%
\definecolor{textcolor}{rgb}{0.150000,0.150000,0.150000}%
\pgfsetstrokecolor{textcolor}%
\pgfsetfillcolor{textcolor}%
\pgftext[x=3.181039in,y=1.145121in,left,base]{\color{textcolor}\rmfamily\fontsize{8.330000}{9.996000}\selectfont \(\displaystyle 0.8\)}%
\end{pgfscope}%
\begin{pgfscope}%
\pgfpathrectangle{\pgfqpoint{2.989294in}{0.330514in}}{\pgfqpoint{0.113967in}{2.279340in}}%
\pgfusepath{clip}%
\pgfsetbuttcap%
\pgfsetroundjoin%
\pgfsetlinewidth{0.803000pt}%
\definecolor{currentstroke}{rgb}{0.800000,0.800000,0.800000}%
\pgfsetstrokecolor{currentstroke}%
\pgfsetdash{{2.960000pt}{1.280000pt}}{0.000000pt}%
\pgfpathmoveto{\pgfqpoint{2.989294in}{1.470184in}}%
\pgfpathlineto{\pgfqpoint{3.103261in}{1.470184in}}%
\pgfusepath{stroke}%
\end{pgfscope}%
\begin{pgfscope}%
\definecolor{textcolor}{rgb}{0.150000,0.150000,0.150000}%
\pgfsetstrokecolor{textcolor}%
\pgfsetfillcolor{textcolor}%
\pgftext[x=3.181039in,y=1.430038in,left,base]{\color{textcolor}\rmfamily\fontsize{8.330000}{9.996000}\selectfont \(\displaystyle 1.0\)}%
\end{pgfscope}%
\begin{pgfscope}%
\pgfpathrectangle{\pgfqpoint{2.989294in}{0.330514in}}{\pgfqpoint{0.113967in}{2.279340in}}%
\pgfusepath{clip}%
\pgfsetbuttcap%
\pgfsetroundjoin%
\pgfsetlinewidth{0.803000pt}%
\definecolor{currentstroke}{rgb}{0.800000,0.800000,0.800000}%
\pgfsetstrokecolor{currentstroke}%
\pgfsetdash{{2.960000pt}{1.280000pt}}{0.000000pt}%
\pgfpathmoveto{\pgfqpoint{2.989294in}{1.755102in}}%
\pgfpathlineto{\pgfqpoint{3.103261in}{1.755102in}}%
\pgfusepath{stroke}%
\end{pgfscope}%
\begin{pgfscope}%
\definecolor{textcolor}{rgb}{0.150000,0.150000,0.150000}%
\pgfsetstrokecolor{textcolor}%
\pgfsetfillcolor{textcolor}%
\pgftext[x=3.181039in,y=1.714956in,left,base]{\color{textcolor}\rmfamily\fontsize{8.330000}{9.996000}\selectfont \(\displaystyle 1.2\)}%
\end{pgfscope}%
\begin{pgfscope}%
\pgfpathrectangle{\pgfqpoint{2.989294in}{0.330514in}}{\pgfqpoint{0.113967in}{2.279340in}}%
\pgfusepath{clip}%
\pgfsetbuttcap%
\pgfsetroundjoin%
\pgfsetlinewidth{0.803000pt}%
\definecolor{currentstroke}{rgb}{0.800000,0.800000,0.800000}%
\pgfsetstrokecolor{currentstroke}%
\pgfsetdash{{2.960000pt}{1.280000pt}}{0.000000pt}%
\pgfpathmoveto{\pgfqpoint{2.989294in}{2.040019in}}%
\pgfpathlineto{\pgfqpoint{3.103261in}{2.040019in}}%
\pgfusepath{stroke}%
\end{pgfscope}%
\begin{pgfscope}%
\definecolor{textcolor}{rgb}{0.150000,0.150000,0.150000}%
\pgfsetstrokecolor{textcolor}%
\pgfsetfillcolor{textcolor}%
\pgftext[x=3.181039in,y=1.999873in,left,base]{\color{textcolor}\rmfamily\fontsize{8.330000}{9.996000}\selectfont \(\displaystyle 1.4\)}%
\end{pgfscope}%
\begin{pgfscope}%
\pgfpathrectangle{\pgfqpoint{2.989294in}{0.330514in}}{\pgfqpoint{0.113967in}{2.279340in}}%
\pgfusepath{clip}%
\pgfsetbuttcap%
\pgfsetroundjoin%
\pgfsetlinewidth{0.803000pt}%
\definecolor{currentstroke}{rgb}{0.800000,0.800000,0.800000}%
\pgfsetstrokecolor{currentstroke}%
\pgfsetdash{{2.960000pt}{1.280000pt}}{0.000000pt}%
\pgfpathmoveto{\pgfqpoint{2.989294in}{2.324937in}}%
\pgfpathlineto{\pgfqpoint{3.103261in}{2.324937in}}%
\pgfusepath{stroke}%
\end{pgfscope}%
\begin{pgfscope}%
\definecolor{textcolor}{rgb}{0.150000,0.150000,0.150000}%
\pgfsetstrokecolor{textcolor}%
\pgfsetfillcolor{textcolor}%
\pgftext[x=3.181039in,y=2.284791in,left,base]{\color{textcolor}\rmfamily\fontsize{8.330000}{9.996000}\selectfont \(\displaystyle 1.6\)}%
\end{pgfscope}%
\begin{pgfscope}%
\pgfpathrectangle{\pgfqpoint{2.989294in}{0.330514in}}{\pgfqpoint{0.113967in}{2.279340in}}%
\pgfusepath{clip}%
\pgfsetbuttcap%
\pgfsetroundjoin%
\pgfsetlinewidth{0.803000pt}%
\definecolor{currentstroke}{rgb}{0.800000,0.800000,0.800000}%
\pgfsetstrokecolor{currentstroke}%
\pgfsetdash{{2.960000pt}{1.280000pt}}{0.000000pt}%
\pgfpathmoveto{\pgfqpoint{2.989294in}{2.609854in}}%
\pgfpathlineto{\pgfqpoint{3.103261in}{2.609854in}}%
\pgfusepath{stroke}%
\end{pgfscope}%
\begin{pgfscope}%
\definecolor{textcolor}{rgb}{0.150000,0.150000,0.150000}%
\pgfsetstrokecolor{textcolor}%
\pgfsetfillcolor{textcolor}%
\pgftext[x=3.181039in,y=2.569708in,left,base]{\color{textcolor}\rmfamily\fontsize{8.330000}{9.996000}\selectfont \(\displaystyle 1.8\)}%
\end{pgfscope}%
\begin{pgfscope}%
\pgfpathrectangle{\pgfqpoint{2.989294in}{0.330514in}}{\pgfqpoint{0.113967in}{2.279340in}}%
\pgfusepath{clip}%
\pgfsetbuttcap%
\pgfsetroundjoin%
\pgfsetlinewidth{1.405250pt}%
\definecolor{currentstroke}{rgb}{0.000000,1.000000,0.500000}%
\pgfsetstrokecolor{currentstroke}%
\pgfsetdash{}{0pt}%
\pgfpathmoveto{\pgfqpoint{2.989294in}{0.330514in}}%
\pgfpathlineto{\pgfqpoint{3.103261in}{0.330514in}}%
\pgfusepath{stroke}%
\end{pgfscope}%
\begin{pgfscope}%
\pgfpathrectangle{\pgfqpoint{2.989294in}{0.330514in}}{\pgfqpoint{0.113967in}{2.279340in}}%
\pgfusepath{clip}%
\pgfsetbuttcap%
\pgfsetroundjoin%
\pgfsetlinewidth{1.405250pt}%
\definecolor{currentstroke}{rgb}{0.000000,0.874510,0.562745}%
\pgfsetstrokecolor{currentstroke}%
\pgfsetdash{}{0pt}%
\pgfpathmoveto{\pgfqpoint{2.989294in}{0.615432in}}%
\pgfpathlineto{\pgfqpoint{3.103261in}{0.615432in}}%
\pgfusepath{stroke}%
\end{pgfscope}%
\begin{pgfscope}%
\pgfpathrectangle{\pgfqpoint{2.989294in}{0.330514in}}{\pgfqpoint{0.113967in}{2.279340in}}%
\pgfusepath{clip}%
\pgfsetbuttcap%
\pgfsetroundjoin%
\pgfsetlinewidth{1.405250pt}%
\definecolor{currentstroke}{rgb}{0.000000,0.749020,0.625490}%
\pgfsetstrokecolor{currentstroke}%
\pgfsetdash{}{0pt}%
\pgfpathmoveto{\pgfqpoint{2.989294in}{0.900349in}}%
\pgfpathlineto{\pgfqpoint{3.103261in}{0.900349in}}%
\pgfusepath{stroke}%
\end{pgfscope}%
\begin{pgfscope}%
\pgfpathrectangle{\pgfqpoint{2.989294in}{0.330514in}}{\pgfqpoint{0.113967in}{2.279340in}}%
\pgfusepath{clip}%
\pgfsetbuttcap%
\pgfsetroundjoin%
\pgfsetlinewidth{1.405250pt}%
\definecolor{currentstroke}{rgb}{0.000000,0.623529,0.688235}%
\pgfsetstrokecolor{currentstroke}%
\pgfsetdash{}{0pt}%
\pgfpathmoveto{\pgfqpoint{2.989294in}{1.185267in}}%
\pgfpathlineto{\pgfqpoint{3.103261in}{1.185267in}}%
\pgfusepath{stroke}%
\end{pgfscope}%
\begin{pgfscope}%
\pgfpathrectangle{\pgfqpoint{2.989294in}{0.330514in}}{\pgfqpoint{0.113967in}{2.279340in}}%
\pgfusepath{clip}%
\pgfsetbuttcap%
\pgfsetroundjoin%
\pgfsetlinewidth{1.405250pt}%
\definecolor{currentstroke}{rgb}{0.000000,0.498039,0.750980}%
\pgfsetstrokecolor{currentstroke}%
\pgfsetdash{}{0pt}%
\pgfpathmoveto{\pgfqpoint{2.989294in}{1.470184in}}%
\pgfpathlineto{\pgfqpoint{3.103261in}{1.470184in}}%
\pgfusepath{stroke}%
\end{pgfscope}%
\begin{pgfscope}%
\pgfpathrectangle{\pgfqpoint{2.989294in}{0.330514in}}{\pgfqpoint{0.113967in}{2.279340in}}%
\pgfusepath{clip}%
\pgfsetbuttcap%
\pgfsetroundjoin%
\pgfsetlinewidth{1.405250pt}%
\definecolor{currentstroke}{rgb}{0.000000,0.372549,0.813725}%
\pgfsetstrokecolor{currentstroke}%
\pgfsetdash{}{0pt}%
\pgfpathmoveto{\pgfqpoint{2.989294in}{1.755102in}}%
\pgfpathlineto{\pgfqpoint{3.103261in}{1.755102in}}%
\pgfusepath{stroke}%
\end{pgfscope}%
\begin{pgfscope}%
\pgfpathrectangle{\pgfqpoint{2.989294in}{0.330514in}}{\pgfqpoint{0.113967in}{2.279340in}}%
\pgfusepath{clip}%
\pgfsetbuttcap%
\pgfsetroundjoin%
\pgfsetlinewidth{1.405250pt}%
\definecolor{currentstroke}{rgb}{0.000000,0.247059,0.876471}%
\pgfsetstrokecolor{currentstroke}%
\pgfsetdash{}{0pt}%
\pgfpathmoveto{\pgfqpoint{2.989294in}{2.040019in}}%
\pgfpathlineto{\pgfqpoint{3.103261in}{2.040019in}}%
\pgfusepath{stroke}%
\end{pgfscope}%
\begin{pgfscope}%
\pgfpathrectangle{\pgfqpoint{2.989294in}{0.330514in}}{\pgfqpoint{0.113967in}{2.279340in}}%
\pgfusepath{clip}%
\pgfsetbuttcap%
\pgfsetroundjoin%
\pgfsetlinewidth{1.405250pt}%
\definecolor{currentstroke}{rgb}{0.000000,0.121569,0.939216}%
\pgfsetstrokecolor{currentstroke}%
\pgfsetdash{}{0pt}%
\pgfpathmoveto{\pgfqpoint{2.989294in}{2.324937in}}%
\pgfpathlineto{\pgfqpoint{3.103261in}{2.324937in}}%
\pgfusepath{stroke}%
\end{pgfscope}%
\begin{pgfscope}%
\pgfpathrectangle{\pgfqpoint{2.989294in}{0.330514in}}{\pgfqpoint{0.113967in}{2.279340in}}%
\pgfusepath{clip}%
\pgfsetbuttcap%
\pgfsetroundjoin%
\pgfsetlinewidth{1.405250pt}%
\definecolor{currentstroke}{rgb}{0.000000,0.000000,1.000000}%
\pgfsetstrokecolor{currentstroke}%
\pgfsetdash{}{0pt}%
\pgfpathmoveto{\pgfqpoint{2.989294in}{2.609854in}}%
\pgfpathlineto{\pgfqpoint{3.103261in}{2.609854in}}%
\pgfusepath{stroke}%
\end{pgfscope}%
\begin{pgfscope}%
\pgfsetbuttcap%
\pgfsetmiterjoin%
\pgfsetlinewidth{1.003750pt}%
\definecolor{currentstroke}{rgb}{0.400000,0.400000,0.400000}%
\pgfsetstrokecolor{currentstroke}%
\pgfsetdash{}{0pt}%
\pgfpathmoveto{\pgfqpoint{2.989294in}{0.330514in}}%
\pgfpathlineto{\pgfqpoint{2.989294in}{0.615432in}}%
\pgfpathlineto{\pgfqpoint{2.989294in}{2.324937in}}%
\pgfpathlineto{\pgfqpoint{2.989294in}{2.609854in}}%
\pgfpathlineto{\pgfqpoint{3.103261in}{2.609854in}}%
\pgfpathlineto{\pgfqpoint{3.103261in}{2.324937in}}%
\pgfpathlineto{\pgfqpoint{3.103261in}{0.615432in}}%
\pgfpathlineto{\pgfqpoint{3.103261in}{0.330514in}}%
\pgfpathclose%
\pgfusepath{stroke}%
\end{pgfscope}%
\end{pgfpicture}%
\makeatother%
\endgroup%


        \caption{Exemplo de escala de cores}
        \label{fig:contorno:cmap}
    \end{figure}


\subsection{Escolha dos Níveis}

    É possível escolher quais os níveis onde serão feitas as curvas com um quarto argumento posicional. Esse argumento deve ser uma lista com as alturas ou, nesse caso, as equipotenciais que serão desenhadas. No exemplo \ref{code:contorno:niveis}, essa lista está na variável \pyline{niveis}.

    \begin{listing}[H]
        \caption{Curvas desenhadas em níveis escolhidos}
        \label{code:contorno:niveis}

        \pyinclude[firstline=53, lastline=59]{recursos/contorno/contorno.py}
    \end{listing}

    \begin{figure}[H]
        \centering
        %% Creator: Matplotlib, PGF backend
%%
%% To include the figure in your LaTeX document, write
%%   \input{<filename>.pgf}
%%
%% Make sure the required packages are loaded in your preamble
%%   \usepackage{pgf}
%%
%% Figures using additional raster images can only be included by \input if
%% they are in the same directory as the main LaTeX file. For loading figures
%% from other directories you can use the `import` package
%%   \usepackage{import}
%% and then include the figures with
%%   \import{<path to file>}{<filename>.pgf}
%%
%% Matplotlib used the following preamble
%%   
%%       \usepackage[portuguese]{babel}
%%       \usepackage[T1]{fontenc}
%%       \usepackage[utf8]{inputenc}
%%   \usepackage{fontspec}
%%
\begingroup%
\makeatletter%
\begin{pgfpicture}%
\pgfpathrectangle{\pgfpointorigin}{\pgfqpoint{3.600000in}{2.800000in}}%
\pgfusepath{use as bounding box, clip}%
\begin{pgfscope}%
\pgfsetbuttcap%
\pgfsetmiterjoin%
\pgfsetlinewidth{0.000000pt}%
\definecolor{currentstroke}{rgb}{0.000000,0.000000,0.000000}%
\pgfsetstrokecolor{currentstroke}%
\pgfsetstrokeopacity{0.000000}%
\pgfsetdash{}{0pt}%
\pgfpathmoveto{\pgfqpoint{0.000000in}{0.000000in}}%
\pgfpathlineto{\pgfqpoint{3.600000in}{0.000000in}}%
\pgfpathlineto{\pgfqpoint{3.600000in}{2.800000in}}%
\pgfpathlineto{\pgfqpoint{0.000000in}{2.800000in}}%
\pgfpathclose%
\pgfusepath{}%
\end{pgfscope}%
\begin{pgfscope}%
\pgfsetbuttcap%
\pgfsetmiterjoin%
\pgfsetlinewidth{0.000000pt}%
\definecolor{currentstroke}{rgb}{0.000000,0.000000,0.000000}%
\pgfsetstrokecolor{currentstroke}%
\pgfsetstrokeopacity{0.000000}%
\pgfsetdash{}{0pt}%
\pgfpathmoveto{\pgfqpoint{0.378629in}{0.330514in}}%
\pgfpathlineto{\pgfqpoint{2.835726in}{0.330514in}}%
\pgfpathlineto{\pgfqpoint{2.835726in}{2.609854in}}%
\pgfpathlineto{\pgfqpoint{0.378629in}{2.609854in}}%
\pgfpathclose%
\pgfusepath{}%
\end{pgfscope}%
\begin{pgfscope}%
\pgfpathrectangle{\pgfqpoint{0.378629in}{0.330514in}}{\pgfqpoint{2.457097in}{2.279340in}}%
\pgfusepath{clip}%
\pgfsetbuttcap%
\pgfsetroundjoin%
\pgfsetlinewidth{0.803000pt}%
\definecolor{currentstroke}{rgb}{0.800000,0.800000,0.800000}%
\pgfsetstrokecolor{currentstroke}%
\pgfsetdash{{2.960000pt}{1.280000pt}}{0.000000pt}%
\pgfpathmoveto{\pgfqpoint{0.583387in}{0.330514in}}%
\pgfpathlineto{\pgfqpoint{0.583387in}{2.609854in}}%
\pgfusepath{stroke}%
\end{pgfscope}%
\begin{pgfscope}%
\definecolor{textcolor}{rgb}{0.150000,0.150000,0.150000}%
\pgfsetstrokecolor{textcolor}%
\pgfsetfillcolor{textcolor}%
\pgftext[x=0.583387in,y=0.252737in,,top]{\color{textcolor}\rmfamily\fontsize{8.330000}{9.996000}\selectfont \(\displaystyle -10\)}%
\end{pgfscope}%
\begin{pgfscope}%
\pgfpathrectangle{\pgfqpoint{0.378629in}{0.330514in}}{\pgfqpoint{2.457097in}{2.279340in}}%
\pgfusepath{clip}%
\pgfsetbuttcap%
\pgfsetroundjoin%
\pgfsetlinewidth{0.803000pt}%
\definecolor{currentstroke}{rgb}{0.800000,0.800000,0.800000}%
\pgfsetstrokecolor{currentstroke}%
\pgfsetdash{{2.960000pt}{1.280000pt}}{0.000000pt}%
\pgfpathmoveto{\pgfqpoint{1.095282in}{0.330514in}}%
\pgfpathlineto{\pgfqpoint{1.095282in}{2.609854in}}%
\pgfusepath{stroke}%
\end{pgfscope}%
\begin{pgfscope}%
\definecolor{textcolor}{rgb}{0.150000,0.150000,0.150000}%
\pgfsetstrokecolor{textcolor}%
\pgfsetfillcolor{textcolor}%
\pgftext[x=1.095282in,y=0.252737in,,top]{\color{textcolor}\rmfamily\fontsize{8.330000}{9.996000}\selectfont \(\displaystyle -5\)}%
\end{pgfscope}%
\begin{pgfscope}%
\pgfpathrectangle{\pgfqpoint{0.378629in}{0.330514in}}{\pgfqpoint{2.457097in}{2.279340in}}%
\pgfusepath{clip}%
\pgfsetbuttcap%
\pgfsetroundjoin%
\pgfsetlinewidth{0.803000pt}%
\definecolor{currentstroke}{rgb}{0.800000,0.800000,0.800000}%
\pgfsetstrokecolor{currentstroke}%
\pgfsetdash{{2.960000pt}{1.280000pt}}{0.000000pt}%
\pgfpathmoveto{\pgfqpoint{1.607177in}{0.330514in}}%
\pgfpathlineto{\pgfqpoint{1.607177in}{2.609854in}}%
\pgfusepath{stroke}%
\end{pgfscope}%
\begin{pgfscope}%
\definecolor{textcolor}{rgb}{0.150000,0.150000,0.150000}%
\pgfsetstrokecolor{textcolor}%
\pgfsetfillcolor{textcolor}%
\pgftext[x=1.607177in,y=0.252737in,,top]{\color{textcolor}\rmfamily\fontsize{8.330000}{9.996000}\selectfont \(\displaystyle 0\)}%
\end{pgfscope}%
\begin{pgfscope}%
\pgfpathrectangle{\pgfqpoint{0.378629in}{0.330514in}}{\pgfqpoint{2.457097in}{2.279340in}}%
\pgfusepath{clip}%
\pgfsetbuttcap%
\pgfsetroundjoin%
\pgfsetlinewidth{0.803000pt}%
\definecolor{currentstroke}{rgb}{0.800000,0.800000,0.800000}%
\pgfsetstrokecolor{currentstroke}%
\pgfsetdash{{2.960000pt}{1.280000pt}}{0.000000pt}%
\pgfpathmoveto{\pgfqpoint{2.119072in}{0.330514in}}%
\pgfpathlineto{\pgfqpoint{2.119072in}{2.609854in}}%
\pgfusepath{stroke}%
\end{pgfscope}%
\begin{pgfscope}%
\definecolor{textcolor}{rgb}{0.150000,0.150000,0.150000}%
\pgfsetstrokecolor{textcolor}%
\pgfsetfillcolor{textcolor}%
\pgftext[x=2.119072in,y=0.252737in,,top]{\color{textcolor}\rmfamily\fontsize{8.330000}{9.996000}\selectfont \(\displaystyle 5\)}%
\end{pgfscope}%
\begin{pgfscope}%
\pgfpathrectangle{\pgfqpoint{0.378629in}{0.330514in}}{\pgfqpoint{2.457097in}{2.279340in}}%
\pgfusepath{clip}%
\pgfsetbuttcap%
\pgfsetroundjoin%
\pgfsetlinewidth{0.803000pt}%
\definecolor{currentstroke}{rgb}{0.800000,0.800000,0.800000}%
\pgfsetstrokecolor{currentstroke}%
\pgfsetdash{{2.960000pt}{1.280000pt}}{0.000000pt}%
\pgfpathmoveto{\pgfqpoint{2.630968in}{0.330514in}}%
\pgfpathlineto{\pgfqpoint{2.630968in}{2.609854in}}%
\pgfusepath{stroke}%
\end{pgfscope}%
\begin{pgfscope}%
\definecolor{textcolor}{rgb}{0.150000,0.150000,0.150000}%
\pgfsetstrokecolor{textcolor}%
\pgfsetfillcolor{textcolor}%
\pgftext[x=2.630968in,y=0.252737in,,top]{\color{textcolor}\rmfamily\fontsize{8.330000}{9.996000}\selectfont \(\displaystyle 10\)}%
\end{pgfscope}%
\begin{pgfscope}%
\pgfpathrectangle{\pgfqpoint{0.378629in}{0.330514in}}{\pgfqpoint{2.457097in}{2.279340in}}%
\pgfusepath{clip}%
\pgfsetbuttcap%
\pgfsetroundjoin%
\pgfsetlinewidth{0.803000pt}%
\definecolor{currentstroke}{rgb}{0.800000,0.800000,0.800000}%
\pgfsetstrokecolor{currentstroke}%
\pgfsetdash{{2.960000pt}{1.280000pt}}{0.000000pt}%
\pgfpathmoveto{\pgfqpoint{0.378629in}{0.549081in}}%
\pgfpathlineto{\pgfqpoint{2.835726in}{0.549081in}}%
\pgfusepath{stroke}%
\end{pgfscope}%
\begin{pgfscope}%
\definecolor{textcolor}{rgb}{0.150000,0.150000,0.150000}%
\pgfsetstrokecolor{textcolor}%
\pgfsetfillcolor{textcolor}%
\pgftext[x=0.150000in,y=0.508935in,left,base]{\color{textcolor}\rmfamily\fontsize{8.330000}{9.996000}\selectfont \(\displaystyle -6\)}%
\end{pgfscope}%
\begin{pgfscope}%
\pgfpathrectangle{\pgfqpoint{0.378629in}{0.330514in}}{\pgfqpoint{2.457097in}{2.279340in}}%
\pgfusepath{clip}%
\pgfsetbuttcap%
\pgfsetroundjoin%
\pgfsetlinewidth{0.803000pt}%
\definecolor{currentstroke}{rgb}{0.800000,0.800000,0.800000}%
\pgfsetstrokecolor{currentstroke}%
\pgfsetdash{{2.960000pt}{1.280000pt}}{0.000000pt}%
\pgfpathmoveto{\pgfqpoint{0.378629in}{0.861319in}}%
\pgfpathlineto{\pgfqpoint{2.835726in}{0.861319in}}%
\pgfusepath{stroke}%
\end{pgfscope}%
\begin{pgfscope}%
\definecolor{textcolor}{rgb}{0.150000,0.150000,0.150000}%
\pgfsetstrokecolor{textcolor}%
\pgfsetfillcolor{textcolor}%
\pgftext[x=0.150000in,y=0.821173in,left,base]{\color{textcolor}\rmfamily\fontsize{8.330000}{9.996000}\selectfont \(\displaystyle -4\)}%
\end{pgfscope}%
\begin{pgfscope}%
\pgfpathrectangle{\pgfqpoint{0.378629in}{0.330514in}}{\pgfqpoint{2.457097in}{2.279340in}}%
\pgfusepath{clip}%
\pgfsetbuttcap%
\pgfsetroundjoin%
\pgfsetlinewidth{0.803000pt}%
\definecolor{currentstroke}{rgb}{0.800000,0.800000,0.800000}%
\pgfsetstrokecolor{currentstroke}%
\pgfsetdash{{2.960000pt}{1.280000pt}}{0.000000pt}%
\pgfpathmoveto{\pgfqpoint{0.378629in}{1.173558in}}%
\pgfpathlineto{\pgfqpoint{2.835726in}{1.173558in}}%
\pgfusepath{stroke}%
\end{pgfscope}%
\begin{pgfscope}%
\definecolor{textcolor}{rgb}{0.150000,0.150000,0.150000}%
\pgfsetstrokecolor{textcolor}%
\pgfsetfillcolor{textcolor}%
\pgftext[x=0.150000in,y=1.133412in,left,base]{\color{textcolor}\rmfamily\fontsize{8.330000}{9.996000}\selectfont \(\displaystyle -2\)}%
\end{pgfscope}%
\begin{pgfscope}%
\pgfpathrectangle{\pgfqpoint{0.378629in}{0.330514in}}{\pgfqpoint{2.457097in}{2.279340in}}%
\pgfusepath{clip}%
\pgfsetbuttcap%
\pgfsetroundjoin%
\pgfsetlinewidth{0.803000pt}%
\definecolor{currentstroke}{rgb}{0.800000,0.800000,0.800000}%
\pgfsetstrokecolor{currentstroke}%
\pgfsetdash{{2.960000pt}{1.280000pt}}{0.000000pt}%
\pgfpathmoveto{\pgfqpoint{0.378629in}{1.485796in}}%
\pgfpathlineto{\pgfqpoint{2.835726in}{1.485796in}}%
\pgfusepath{stroke}%
\end{pgfscope}%
\begin{pgfscope}%
\definecolor{textcolor}{rgb}{0.150000,0.150000,0.150000}%
\pgfsetstrokecolor{textcolor}%
\pgfsetfillcolor{textcolor}%
\pgftext[x=0.241822in,y=1.445650in,left,base]{\color{textcolor}\rmfamily\fontsize{8.330000}{9.996000}\selectfont \(\displaystyle 0\)}%
\end{pgfscope}%
\begin{pgfscope}%
\pgfpathrectangle{\pgfqpoint{0.378629in}{0.330514in}}{\pgfqpoint{2.457097in}{2.279340in}}%
\pgfusepath{clip}%
\pgfsetbuttcap%
\pgfsetroundjoin%
\pgfsetlinewidth{0.803000pt}%
\definecolor{currentstroke}{rgb}{0.800000,0.800000,0.800000}%
\pgfsetstrokecolor{currentstroke}%
\pgfsetdash{{2.960000pt}{1.280000pt}}{0.000000pt}%
\pgfpathmoveto{\pgfqpoint{0.378629in}{1.798034in}}%
\pgfpathlineto{\pgfqpoint{2.835726in}{1.798034in}}%
\pgfusepath{stroke}%
\end{pgfscope}%
\begin{pgfscope}%
\definecolor{textcolor}{rgb}{0.150000,0.150000,0.150000}%
\pgfsetstrokecolor{textcolor}%
\pgfsetfillcolor{textcolor}%
\pgftext[x=0.241822in,y=1.757888in,left,base]{\color{textcolor}\rmfamily\fontsize{8.330000}{9.996000}\selectfont \(\displaystyle 2\)}%
\end{pgfscope}%
\begin{pgfscope}%
\pgfpathrectangle{\pgfqpoint{0.378629in}{0.330514in}}{\pgfqpoint{2.457097in}{2.279340in}}%
\pgfusepath{clip}%
\pgfsetbuttcap%
\pgfsetroundjoin%
\pgfsetlinewidth{0.803000pt}%
\definecolor{currentstroke}{rgb}{0.800000,0.800000,0.800000}%
\pgfsetstrokecolor{currentstroke}%
\pgfsetdash{{2.960000pt}{1.280000pt}}{0.000000pt}%
\pgfpathmoveto{\pgfqpoint{0.378629in}{2.110273in}}%
\pgfpathlineto{\pgfqpoint{2.835726in}{2.110273in}}%
\pgfusepath{stroke}%
\end{pgfscope}%
\begin{pgfscope}%
\definecolor{textcolor}{rgb}{0.150000,0.150000,0.150000}%
\pgfsetstrokecolor{textcolor}%
\pgfsetfillcolor{textcolor}%
\pgftext[x=0.241822in,y=2.070127in,left,base]{\color{textcolor}\rmfamily\fontsize{8.330000}{9.996000}\selectfont \(\displaystyle 4\)}%
\end{pgfscope}%
\begin{pgfscope}%
\pgfpathrectangle{\pgfqpoint{0.378629in}{0.330514in}}{\pgfqpoint{2.457097in}{2.279340in}}%
\pgfusepath{clip}%
\pgfsetbuttcap%
\pgfsetroundjoin%
\pgfsetlinewidth{0.803000pt}%
\definecolor{currentstroke}{rgb}{0.800000,0.800000,0.800000}%
\pgfsetstrokecolor{currentstroke}%
\pgfsetdash{{2.960000pt}{1.280000pt}}{0.000000pt}%
\pgfpathmoveto{\pgfqpoint{0.378629in}{2.422511in}}%
\pgfpathlineto{\pgfqpoint{2.835726in}{2.422511in}}%
\pgfusepath{stroke}%
\end{pgfscope}%
\begin{pgfscope}%
\definecolor{textcolor}{rgb}{0.150000,0.150000,0.150000}%
\pgfsetstrokecolor{textcolor}%
\pgfsetfillcolor{textcolor}%
\pgftext[x=0.241822in,y=2.382365in,left,base]{\color{textcolor}\rmfamily\fontsize{8.330000}{9.996000}\selectfont \(\displaystyle 6\)}%
\end{pgfscope}%
\begin{pgfscope}%
\pgfpathrectangle{\pgfqpoint{0.378629in}{0.330514in}}{\pgfqpoint{2.457097in}{2.279340in}}%
\pgfusepath{clip}%
\pgfsetbuttcap%
\pgfsetroundjoin%
\pgfsetlinewidth{1.405250pt}%
\definecolor{currentstroke}{rgb}{0.000000,1.000000,0.500000}%
\pgfsetstrokecolor{currentstroke}%
\pgfsetdash{}{0pt}%
\pgfpathmoveto{\pgfqpoint{0.378629in}{0.404282in}}%
\pgfpathlineto{\pgfqpoint{0.391001in}{0.404753in}}%
\pgfpathlineto{\pgfqpoint{0.788145in}{0.360610in}}%
\pgfpathlineto{\pgfqpoint{0.801632in}{0.360009in}}%
\pgfpathlineto{\pgfqpoint{1.197661in}{0.358785in}}%
\pgfpathlineto{\pgfqpoint{1.210146in}{0.358502in}}%
\pgfpathlineto{\pgfqpoint{1.607177in}{0.345842in}}%
\pgfpathlineto{\pgfqpoint{2.000610in}{0.347070in}}%
\pgfpathlineto{\pgfqpoint{2.016693in}{0.347070in}}%
\pgfpathlineto{\pgfqpoint{2.411318in}{0.360887in}}%
\pgfpathlineto{\pgfqpoint{2.426210in}{0.360842in}}%
\pgfpathlineto{\pgfqpoint{2.441056in}{0.359144in}}%
\pgfpathlineto{\pgfqpoint{2.835726in}{0.339681in}}%
\pgfusepath{stroke}%
\end{pgfscope}%
\begin{pgfscope}%
\pgfpathrectangle{\pgfqpoint{0.378629in}{0.330514in}}{\pgfqpoint{2.457097in}{2.279340in}}%
\pgfusepath{clip}%
\pgfsetbuttcap%
\pgfsetroundjoin%
\pgfsetlinewidth{1.405250pt}%
\definecolor{currentstroke}{rgb}{0.000000,0.756863,0.621569}%
\pgfsetstrokecolor{currentstroke}%
\pgfsetdash{}{0pt}%
\pgfpathmoveto{\pgfqpoint{0.378629in}{0.766516in}}%
\pgfpathlineto{\pgfqpoint{0.786908in}{0.782081in}}%
\pgfpathlineto{\pgfqpoint{0.788145in}{0.781943in}}%
\pgfpathlineto{\pgfqpoint{1.193983in}{0.763862in}}%
\pgfpathlineto{\pgfqpoint{1.197661in}{0.763850in}}%
\pgfpathlineto{\pgfqpoint{1.443371in}{0.758281in}}%
\pgfpathlineto{\pgfqpoint{1.607177in}{0.754417in}}%
\pgfpathlineto{\pgfqpoint{2.014292in}{0.754417in}}%
\pgfpathlineto{\pgfqpoint{2.016693in}{0.754592in}}%
\pgfpathlineto{\pgfqpoint{2.425009in}{0.753318in}}%
\pgfpathlineto{\pgfqpoint{2.426210in}{0.753518in}}%
\pgfpathlineto{\pgfqpoint{2.427393in}{0.753435in}}%
\pgfpathlineto{\pgfqpoint{2.835726in}{0.706734in}}%
\pgfusepath{stroke}%
\end{pgfscope}%
\begin{pgfscope}%
\pgfpathrectangle{\pgfqpoint{0.378629in}{0.330514in}}{\pgfqpoint{2.457097in}{2.279340in}}%
\pgfusepath{clip}%
\pgfsetbuttcap%
\pgfsetroundjoin%
\pgfsetlinewidth{1.405250pt}%
\definecolor{currentstroke}{rgb}{0.000000,0.498039,0.750980}%
\pgfsetstrokecolor{currentstroke}%
\pgfsetdash{}{0pt}%
\pgfpathmoveto{\pgfqpoint{0.378629in}{1.258448in}}%
\pgfpathlineto{\pgfqpoint{0.382274in}{1.257732in}}%
\pgfpathlineto{\pgfqpoint{0.788145in}{1.238103in}}%
\pgfpathlineto{\pgfqpoint{0.789401in}{1.237825in}}%
\pgfpathlineto{\pgfqpoint{1.197661in}{1.208601in}}%
\pgfpathlineto{\pgfqpoint{1.200143in}{1.207526in}}%
\pgfpathlineto{\pgfqpoint{1.607177in}{1.159464in}}%
\pgfpathlineto{\pgfqpoint{2.016693in}{1.189170in}}%
\pgfpathlineto{\pgfqpoint{2.421305in}{1.257974in}}%
\pgfpathlineto{\pgfqpoint{2.426210in}{1.260056in}}%
\pgfpathlineto{\pgfqpoint{2.608217in}{1.244679in}}%
\pgfpathlineto{\pgfqpoint{2.829808in}{1.229012in}}%
\pgfpathlineto{\pgfqpoint{2.835726in}{1.228335in}}%
\pgfusepath{stroke}%
\end{pgfscope}%
\begin{pgfscope}%
\pgfpathrectangle{\pgfqpoint{0.378629in}{0.330514in}}{\pgfqpoint{2.457097in}{2.279340in}}%
\pgfusepath{clip}%
\pgfsetbuttcap%
\pgfsetroundjoin%
\pgfsetlinewidth{1.405250pt}%
\definecolor{currentstroke}{rgb}{0.000000,0.243137,0.878431}%
\pgfsetstrokecolor{currentstroke}%
\pgfsetdash{}{0pt}%
\pgfpathmoveto{\pgfqpoint{0.378629in}{2.009771in}}%
\pgfpathlineto{\pgfqpoint{0.783284in}{1.930388in}}%
\pgfpathlineto{\pgfqpoint{0.788145in}{1.930153in}}%
\pgfpathlineto{\pgfqpoint{0.970152in}{1.889971in}}%
\pgfpathlineto{\pgfqpoint{1.197661in}{1.837597in}}%
\pgfpathlineto{\pgfqpoint{1.203755in}{1.832743in}}%
\pgfpathlineto{\pgfqpoint{1.607177in}{1.659965in}}%
\pgfpathlineto{\pgfqpoint{2.016693in}{1.782422in}}%
\pgfpathlineto{\pgfqpoint{2.424957in}{1.955108in}}%
\pgfpathlineto{\pgfqpoint{2.426210in}{1.955447in}}%
\pgfpathlineto{\pgfqpoint{2.528589in}{1.958057in}}%
\pgfpathlineto{\pgfqpoint{2.835726in}{1.963054in}}%
\pgfusepath{stroke}%
\end{pgfscope}%
\begin{pgfscope}%
\pgfpathrectangle{\pgfqpoint{0.378629in}{0.330514in}}{\pgfqpoint{2.457097in}{2.279340in}}%
\pgfusepath{clip}%
\pgfsetbuttcap%
\pgfsetroundjoin%
\pgfsetlinewidth{1.405250pt}%
\definecolor{currentstroke}{rgb}{0.000000,0.000000,1.000000}%
\pgfsetstrokecolor{currentstroke}%
\pgfsetdash{}{0pt}%
\pgfpathmoveto{\pgfqpoint{0.378629in}{2.598834in}}%
\pgfpathlineto{\pgfqpoint{0.775621in}{2.516039in}}%
\pgfpathlineto{\pgfqpoint{0.788145in}{2.513597in}}%
\pgfpathlineto{\pgfqpoint{1.182679in}{2.374723in}}%
\pgfpathlineto{\pgfqpoint{1.197661in}{2.371274in}}%
\pgfpathlineto{\pgfqpoint{1.593770in}{2.055770in}}%
\pgfpathlineto{\pgfqpoint{1.607177in}{2.050029in}}%
\pgfpathlineto{\pgfqpoint{1.620787in}{2.054098in}}%
\pgfpathlineto{\pgfqpoint{2.016693in}{2.257484in}}%
\pgfpathlineto{\pgfqpoint{2.024207in}{2.260663in}}%
\pgfpathlineto{\pgfqpoint{2.426210in}{2.369210in}}%
\pgfpathlineto{\pgfqpoint{2.432569in}{2.369372in}}%
\pgfpathlineto{\pgfqpoint{2.835726in}{2.408666in}}%
\pgfusepath{stroke}%
\end{pgfscope}%
\begin{pgfscope}%
\pgfsetrectcap%
\pgfsetmiterjoin%
\pgfsetlinewidth{1.003750pt}%
\definecolor{currentstroke}{rgb}{0.400000,0.400000,0.400000}%
\pgfsetstrokecolor{currentstroke}%
\pgfsetdash{}{0pt}%
\pgfpathmoveto{\pgfqpoint{0.378629in}{0.330514in}}%
\pgfpathlineto{\pgfqpoint{0.378629in}{2.609854in}}%
\pgfusepath{stroke}%
\end{pgfscope}%
\begin{pgfscope}%
\pgfsetrectcap%
\pgfsetmiterjoin%
\pgfsetlinewidth{1.003750pt}%
\definecolor{currentstroke}{rgb}{0.400000,0.400000,0.400000}%
\pgfsetstrokecolor{currentstroke}%
\pgfsetdash{}{0pt}%
\pgfpathmoveto{\pgfqpoint{0.378629in}{0.330514in}}%
\pgfpathlineto{\pgfqpoint{2.835726in}{0.330514in}}%
\pgfusepath{stroke}%
\end{pgfscope}%
\begin{pgfscope}%
\pgfpathrectangle{\pgfqpoint{2.989294in}{0.330514in}}{\pgfqpoint{0.113967in}{2.279340in}}%
\pgfusepath{clip}%
\pgfsetbuttcap%
\pgfsetmiterjoin%
\definecolor{currentfill}{rgb}{1.000000,1.000000,1.000000}%
\pgfsetfillcolor{currentfill}%
\pgfsetlinewidth{0.010037pt}%
\definecolor{currentstroke}{rgb}{1.000000,1.000000,1.000000}%
\pgfsetstrokecolor{currentstroke}%
\pgfsetdash{}{0pt}%
\pgfpathmoveto{\pgfqpoint{2.989294in}{0.330514in}}%
\pgfpathlineto{\pgfqpoint{2.989294in}{0.900349in}}%
\pgfpathlineto{\pgfqpoint{2.989294in}{2.040019in}}%
\pgfpathlineto{\pgfqpoint{2.989294in}{2.609854in}}%
\pgfpathlineto{\pgfqpoint{3.103261in}{2.609854in}}%
\pgfpathlineto{\pgfqpoint{3.103261in}{2.040019in}}%
\pgfpathlineto{\pgfqpoint{3.103261in}{0.900349in}}%
\pgfpathlineto{\pgfqpoint{3.103261in}{0.330514in}}%
\pgfpathclose%
\pgfusepath{stroke,fill}%
\end{pgfscope}%
\begin{pgfscope}%
\pgfpathrectangle{\pgfqpoint{2.989294in}{0.330514in}}{\pgfqpoint{0.113967in}{2.279340in}}%
\pgfusepath{clip}%
\pgfsetbuttcap%
\pgfsetroundjoin%
\pgfsetlinewidth{0.803000pt}%
\definecolor{currentstroke}{rgb}{0.800000,0.800000,0.800000}%
\pgfsetstrokecolor{currentstroke}%
\pgfsetdash{{2.960000pt}{1.280000pt}}{0.000000pt}%
\pgfpathmoveto{\pgfqpoint{2.989294in}{0.330514in}}%
\pgfpathlineto{\pgfqpoint{3.103261in}{0.330514in}}%
\pgfusepath{stroke}%
\end{pgfscope}%
\begin{pgfscope}%
\definecolor{textcolor}{rgb}{0.150000,0.150000,0.150000}%
\pgfsetstrokecolor{textcolor}%
\pgfsetfillcolor{textcolor}%
\pgftext[x=3.181039in,y=0.290368in,left,base]{\color{textcolor}\rmfamily\fontsize{8.330000}{9.996000}\selectfont \(\displaystyle 0.34\)}%
\end{pgfscope}%
\begin{pgfscope}%
\pgfpathrectangle{\pgfqpoint{2.989294in}{0.330514in}}{\pgfqpoint{0.113967in}{2.279340in}}%
\pgfusepath{clip}%
\pgfsetbuttcap%
\pgfsetroundjoin%
\pgfsetlinewidth{0.803000pt}%
\definecolor{currentstroke}{rgb}{0.800000,0.800000,0.800000}%
\pgfsetstrokecolor{currentstroke}%
\pgfsetdash{{2.960000pt}{1.280000pt}}{0.000000pt}%
\pgfpathmoveto{\pgfqpoint{2.989294in}{0.900349in}}%
\pgfpathlineto{\pgfqpoint{3.103261in}{0.900349in}}%
\pgfusepath{stroke}%
\end{pgfscope}%
\begin{pgfscope}%
\definecolor{textcolor}{rgb}{0.150000,0.150000,0.150000}%
\pgfsetstrokecolor{textcolor}%
\pgfsetfillcolor{textcolor}%
\pgftext[x=3.181039in,y=0.860203in,left,base]{\color{textcolor}\rmfamily\fontsize{8.330000}{9.996000}\selectfont \(\displaystyle 0.66\)}%
\end{pgfscope}%
\begin{pgfscope}%
\pgfpathrectangle{\pgfqpoint{2.989294in}{0.330514in}}{\pgfqpoint{0.113967in}{2.279340in}}%
\pgfusepath{clip}%
\pgfsetbuttcap%
\pgfsetroundjoin%
\pgfsetlinewidth{0.803000pt}%
\definecolor{currentstroke}{rgb}{0.800000,0.800000,0.800000}%
\pgfsetstrokecolor{currentstroke}%
\pgfsetdash{{2.960000pt}{1.280000pt}}{0.000000pt}%
\pgfpathmoveto{\pgfqpoint{2.989294in}{1.470184in}}%
\pgfpathlineto{\pgfqpoint{3.103261in}{1.470184in}}%
\pgfusepath{stroke}%
\end{pgfscope}%
\begin{pgfscope}%
\definecolor{textcolor}{rgb}{0.150000,0.150000,0.150000}%
\pgfsetstrokecolor{textcolor}%
\pgfsetfillcolor{textcolor}%
\pgftext[x=3.181039in,y=1.430038in,left,base]{\color{textcolor}\rmfamily\fontsize{8.330000}{9.996000}\selectfont \(\displaystyle 1.00\)}%
\end{pgfscope}%
\begin{pgfscope}%
\pgfpathrectangle{\pgfqpoint{2.989294in}{0.330514in}}{\pgfqpoint{0.113967in}{2.279340in}}%
\pgfusepath{clip}%
\pgfsetbuttcap%
\pgfsetroundjoin%
\pgfsetlinewidth{0.803000pt}%
\definecolor{currentstroke}{rgb}{0.800000,0.800000,0.800000}%
\pgfsetstrokecolor{currentstroke}%
\pgfsetdash{{2.960000pt}{1.280000pt}}{0.000000pt}%
\pgfpathmoveto{\pgfqpoint{2.989294in}{2.040019in}}%
\pgfpathlineto{\pgfqpoint{3.103261in}{2.040019in}}%
\pgfusepath{stroke}%
\end{pgfscope}%
\begin{pgfscope}%
\definecolor{textcolor}{rgb}{0.150000,0.150000,0.150000}%
\pgfsetstrokecolor{textcolor}%
\pgfsetfillcolor{textcolor}%
\pgftext[x=3.181039in,y=1.999873in,left,base]{\color{textcolor}\rmfamily\fontsize{8.330000}{9.996000}\selectfont \(\displaystyle 1.33\)}%
\end{pgfscope}%
\begin{pgfscope}%
\pgfpathrectangle{\pgfqpoint{2.989294in}{0.330514in}}{\pgfqpoint{0.113967in}{2.279340in}}%
\pgfusepath{clip}%
\pgfsetbuttcap%
\pgfsetroundjoin%
\pgfsetlinewidth{0.803000pt}%
\definecolor{currentstroke}{rgb}{0.800000,0.800000,0.800000}%
\pgfsetstrokecolor{currentstroke}%
\pgfsetdash{{2.960000pt}{1.280000pt}}{0.000000pt}%
\pgfpathmoveto{\pgfqpoint{2.989294in}{2.609854in}}%
\pgfpathlineto{\pgfqpoint{3.103261in}{2.609854in}}%
\pgfusepath{stroke}%
\end{pgfscope}%
\begin{pgfscope}%
\definecolor{textcolor}{rgb}{0.150000,0.150000,0.150000}%
\pgfsetstrokecolor{textcolor}%
\pgfsetfillcolor{textcolor}%
\pgftext[x=3.181039in,y=2.569708in,left,base]{\color{textcolor}\rmfamily\fontsize{8.330000}{9.996000}\selectfont \(\displaystyle 1.65\)}%
\end{pgfscope}%
\begin{pgfscope}%
\pgfpathrectangle{\pgfqpoint{2.989294in}{0.330514in}}{\pgfqpoint{0.113967in}{2.279340in}}%
\pgfusepath{clip}%
\pgfsetbuttcap%
\pgfsetroundjoin%
\pgfsetlinewidth{1.405250pt}%
\definecolor{currentstroke}{rgb}{0.000000,1.000000,0.500000}%
\pgfsetstrokecolor{currentstroke}%
\pgfsetdash{}{0pt}%
\pgfpathmoveto{\pgfqpoint{2.989294in}{0.330514in}}%
\pgfpathlineto{\pgfqpoint{3.103261in}{0.330514in}}%
\pgfusepath{stroke}%
\end{pgfscope}%
\begin{pgfscope}%
\pgfpathrectangle{\pgfqpoint{2.989294in}{0.330514in}}{\pgfqpoint{0.113967in}{2.279340in}}%
\pgfusepath{clip}%
\pgfsetbuttcap%
\pgfsetroundjoin%
\pgfsetlinewidth{1.405250pt}%
\definecolor{currentstroke}{rgb}{0.000000,0.756863,0.621569}%
\pgfsetstrokecolor{currentstroke}%
\pgfsetdash{}{0pt}%
\pgfpathmoveto{\pgfqpoint{2.989294in}{0.900349in}}%
\pgfpathlineto{\pgfqpoint{3.103261in}{0.900349in}}%
\pgfusepath{stroke}%
\end{pgfscope}%
\begin{pgfscope}%
\pgfpathrectangle{\pgfqpoint{2.989294in}{0.330514in}}{\pgfqpoint{0.113967in}{2.279340in}}%
\pgfusepath{clip}%
\pgfsetbuttcap%
\pgfsetroundjoin%
\pgfsetlinewidth{1.405250pt}%
\definecolor{currentstroke}{rgb}{0.000000,0.498039,0.750980}%
\pgfsetstrokecolor{currentstroke}%
\pgfsetdash{}{0pt}%
\pgfpathmoveto{\pgfqpoint{2.989294in}{1.470184in}}%
\pgfpathlineto{\pgfqpoint{3.103261in}{1.470184in}}%
\pgfusepath{stroke}%
\end{pgfscope}%
\begin{pgfscope}%
\pgfpathrectangle{\pgfqpoint{2.989294in}{0.330514in}}{\pgfqpoint{0.113967in}{2.279340in}}%
\pgfusepath{clip}%
\pgfsetbuttcap%
\pgfsetroundjoin%
\pgfsetlinewidth{1.405250pt}%
\definecolor{currentstroke}{rgb}{0.000000,0.243137,0.878431}%
\pgfsetstrokecolor{currentstroke}%
\pgfsetdash{}{0pt}%
\pgfpathmoveto{\pgfqpoint{2.989294in}{2.040019in}}%
\pgfpathlineto{\pgfqpoint{3.103261in}{2.040019in}}%
\pgfusepath{stroke}%
\end{pgfscope}%
\begin{pgfscope}%
\pgfpathrectangle{\pgfqpoint{2.989294in}{0.330514in}}{\pgfqpoint{0.113967in}{2.279340in}}%
\pgfusepath{clip}%
\pgfsetbuttcap%
\pgfsetroundjoin%
\pgfsetlinewidth{1.405250pt}%
\definecolor{currentstroke}{rgb}{0.000000,0.000000,1.000000}%
\pgfsetstrokecolor{currentstroke}%
\pgfsetdash{}{0pt}%
\pgfpathmoveto{\pgfqpoint{2.989294in}{2.609854in}}%
\pgfpathlineto{\pgfqpoint{3.103261in}{2.609854in}}%
\pgfusepath{stroke}%
\end{pgfscope}%
\begin{pgfscope}%
\pgfsetbuttcap%
\pgfsetmiterjoin%
\pgfsetlinewidth{1.003750pt}%
\definecolor{currentstroke}{rgb}{0.400000,0.400000,0.400000}%
\pgfsetstrokecolor{currentstroke}%
\pgfsetdash{}{0pt}%
\pgfpathmoveto{\pgfqpoint{2.989294in}{0.330514in}}%
\pgfpathlineto{\pgfqpoint{2.989294in}{0.900349in}}%
\pgfpathlineto{\pgfqpoint{2.989294in}{2.040019in}}%
\pgfpathlineto{\pgfqpoint{2.989294in}{2.609854in}}%
\pgfpathlineto{\pgfqpoint{3.103261in}{2.609854in}}%
\pgfpathlineto{\pgfqpoint{3.103261in}{2.040019in}}%
\pgfpathlineto{\pgfqpoint{3.103261in}{0.900349in}}%
\pgfpathlineto{\pgfqpoint{3.103261in}{0.330514in}}%
\pgfpathclose%
\pgfusepath{stroke}%
\end{pgfscope}%
\end{pgfpicture}%
\makeatother%
\endgroup%


        \caption{Exemplo das curvas em níveis específicos}
        \label{fig:contorno:niveis}
    \end{figure}

    \begin{nota}
        Devido às imperfeições da técnica de triangulação, é possível que as escolha manual dos níveis pode gerar alguns problemas no desenho, em especial nas extremidades da figura. Por exemplo, com \\\pyline{niveis = [0.33, 0.66, 1.00, 1.33, 1.66]}, o código \ref{code:contorno:niveis} gera a curva de maior potencial com várias semi-retas sem sentido e a curva de menor potencial termina antes das outras.

        \begin{figure}[H]
            \centering
            %% Creator: Matplotlib, PGF backend
%%
%% To include the figure in your LaTeX document, write
%%   \input{<filename>.pgf}
%%
%% Make sure the required packages are loaded in your preamble
%%   \usepackage{pgf}
%%
%% Figures using additional raster images can only be included by \input if
%% they are in the same directory as the main LaTeX file. For loading figures
%% from other directories you can use the `import` package
%%   \usepackage{import}
%% and then include the figures with
%%   \import{<path to file>}{<filename>.pgf}
%%
%% Matplotlib used the following preamble
%%   
%%       \usepackage[portuguese]{babel}
%%       \usepackage[T1]{fontenc}
%%       \usepackage[utf8]{inputenc}
%%   \usepackage{fontspec}
%%
\begingroup%
\makeatletter%
\begin{pgfpicture}%
\pgfpathrectangle{\pgfpointorigin}{\pgfqpoint{3.150000in}{2.450000in}}%
\pgfusepath{use as bounding box, clip}%
\begin{pgfscope}%
\pgfsetbuttcap%
\pgfsetmiterjoin%
\pgfsetlinewidth{0.000000pt}%
\definecolor{currentstroke}{rgb}{0.000000,0.000000,0.000000}%
\pgfsetstrokecolor{currentstroke}%
\pgfsetstrokeopacity{0.000000}%
\pgfsetdash{}{0pt}%
\pgfpathmoveto{\pgfqpoint{0.000000in}{0.000000in}}%
\pgfpathlineto{\pgfqpoint{3.150000in}{0.000000in}}%
\pgfpathlineto{\pgfqpoint{3.150000in}{2.450000in}}%
\pgfpathlineto{\pgfqpoint{0.000000in}{2.450000in}}%
\pgfpathclose%
\pgfusepath{}%
\end{pgfscope}%
\begin{pgfscope}%
\pgfsetbuttcap%
\pgfsetmiterjoin%
\pgfsetlinewidth{0.000000pt}%
\definecolor{currentstroke}{rgb}{0.000000,0.000000,0.000000}%
\pgfsetstrokecolor{currentstroke}%
\pgfsetstrokeopacity{0.000000}%
\pgfsetdash{}{0pt}%
\pgfpathmoveto{\pgfqpoint{0.378629in}{0.330514in}}%
\pgfpathlineto{\pgfqpoint{2.463090in}{0.330514in}}%
\pgfpathlineto{\pgfqpoint{2.463090in}{2.259854in}}%
\pgfpathlineto{\pgfqpoint{0.378629in}{2.259854in}}%
\pgfpathclose%
\pgfusepath{}%
\end{pgfscope}%
\begin{pgfscope}%
\pgfpathrectangle{\pgfqpoint{0.378629in}{0.330514in}}{\pgfqpoint{2.084461in}{1.929340in}}%
\pgfusepath{clip}%
\pgfsetbuttcap%
\pgfsetroundjoin%
\pgfsetlinewidth{0.803000pt}%
\definecolor{currentstroke}{rgb}{0.800000,0.800000,0.800000}%
\pgfsetstrokecolor{currentstroke}%
\pgfsetdash{{2.960000pt}{1.280000pt}}{0.000000pt}%
\pgfpathmoveto{\pgfqpoint{0.552334in}{0.330514in}}%
\pgfpathlineto{\pgfqpoint{0.552334in}{2.259854in}}%
\pgfusepath{stroke}%
\end{pgfscope}%
\begin{pgfscope}%
\definecolor{textcolor}{rgb}{0.150000,0.150000,0.150000}%
\pgfsetstrokecolor{textcolor}%
\pgfsetfillcolor{textcolor}%
\pgftext[x=0.552334in,y=0.252737in,,top]{\color{textcolor}\rmfamily\fontsize{8.330000}{9.996000}\selectfont \(\displaystyle -10\)}%
\end{pgfscope}%
\begin{pgfscope}%
\pgfpathrectangle{\pgfqpoint{0.378629in}{0.330514in}}{\pgfqpoint{2.084461in}{1.929340in}}%
\pgfusepath{clip}%
\pgfsetbuttcap%
\pgfsetroundjoin%
\pgfsetlinewidth{0.803000pt}%
\definecolor{currentstroke}{rgb}{0.800000,0.800000,0.800000}%
\pgfsetstrokecolor{currentstroke}%
\pgfsetdash{{2.960000pt}{1.280000pt}}{0.000000pt}%
\pgfpathmoveto{\pgfqpoint{0.986597in}{0.330514in}}%
\pgfpathlineto{\pgfqpoint{0.986597in}{2.259854in}}%
\pgfusepath{stroke}%
\end{pgfscope}%
\begin{pgfscope}%
\definecolor{textcolor}{rgb}{0.150000,0.150000,0.150000}%
\pgfsetstrokecolor{textcolor}%
\pgfsetfillcolor{textcolor}%
\pgftext[x=0.986597in,y=0.252737in,,top]{\color{textcolor}\rmfamily\fontsize{8.330000}{9.996000}\selectfont \(\displaystyle -5\)}%
\end{pgfscope}%
\begin{pgfscope}%
\pgfpathrectangle{\pgfqpoint{0.378629in}{0.330514in}}{\pgfqpoint{2.084461in}{1.929340in}}%
\pgfusepath{clip}%
\pgfsetbuttcap%
\pgfsetroundjoin%
\pgfsetlinewidth{0.803000pt}%
\definecolor{currentstroke}{rgb}{0.800000,0.800000,0.800000}%
\pgfsetstrokecolor{currentstroke}%
\pgfsetdash{{2.960000pt}{1.280000pt}}{0.000000pt}%
\pgfpathmoveto{\pgfqpoint{1.420859in}{0.330514in}}%
\pgfpathlineto{\pgfqpoint{1.420859in}{2.259854in}}%
\pgfusepath{stroke}%
\end{pgfscope}%
\begin{pgfscope}%
\definecolor{textcolor}{rgb}{0.150000,0.150000,0.150000}%
\pgfsetstrokecolor{textcolor}%
\pgfsetfillcolor{textcolor}%
\pgftext[x=1.420859in,y=0.252737in,,top]{\color{textcolor}\rmfamily\fontsize{8.330000}{9.996000}\selectfont \(\displaystyle 0\)}%
\end{pgfscope}%
\begin{pgfscope}%
\pgfpathrectangle{\pgfqpoint{0.378629in}{0.330514in}}{\pgfqpoint{2.084461in}{1.929340in}}%
\pgfusepath{clip}%
\pgfsetbuttcap%
\pgfsetroundjoin%
\pgfsetlinewidth{0.803000pt}%
\definecolor{currentstroke}{rgb}{0.800000,0.800000,0.800000}%
\pgfsetstrokecolor{currentstroke}%
\pgfsetdash{{2.960000pt}{1.280000pt}}{0.000000pt}%
\pgfpathmoveto{\pgfqpoint{1.855122in}{0.330514in}}%
\pgfpathlineto{\pgfqpoint{1.855122in}{2.259854in}}%
\pgfusepath{stroke}%
\end{pgfscope}%
\begin{pgfscope}%
\definecolor{textcolor}{rgb}{0.150000,0.150000,0.150000}%
\pgfsetstrokecolor{textcolor}%
\pgfsetfillcolor{textcolor}%
\pgftext[x=1.855122in,y=0.252737in,,top]{\color{textcolor}\rmfamily\fontsize{8.330000}{9.996000}\selectfont \(\displaystyle 5\)}%
\end{pgfscope}%
\begin{pgfscope}%
\pgfpathrectangle{\pgfqpoint{0.378629in}{0.330514in}}{\pgfqpoint{2.084461in}{1.929340in}}%
\pgfusepath{clip}%
\pgfsetbuttcap%
\pgfsetroundjoin%
\pgfsetlinewidth{0.803000pt}%
\definecolor{currentstroke}{rgb}{0.800000,0.800000,0.800000}%
\pgfsetstrokecolor{currentstroke}%
\pgfsetdash{{2.960000pt}{1.280000pt}}{0.000000pt}%
\pgfpathmoveto{\pgfqpoint{2.289385in}{0.330514in}}%
\pgfpathlineto{\pgfqpoint{2.289385in}{2.259854in}}%
\pgfusepath{stroke}%
\end{pgfscope}%
\begin{pgfscope}%
\definecolor{textcolor}{rgb}{0.150000,0.150000,0.150000}%
\pgfsetstrokecolor{textcolor}%
\pgfsetfillcolor{textcolor}%
\pgftext[x=2.289385in,y=0.252737in,,top]{\color{textcolor}\rmfamily\fontsize{8.330000}{9.996000}\selectfont \(\displaystyle 10\)}%
\end{pgfscope}%
\begin{pgfscope}%
\pgfpathrectangle{\pgfqpoint{0.378629in}{0.330514in}}{\pgfqpoint{2.084461in}{1.929340in}}%
\pgfusepath{clip}%
\pgfsetbuttcap%
\pgfsetroundjoin%
\pgfsetlinewidth{0.803000pt}%
\definecolor{currentstroke}{rgb}{0.800000,0.800000,0.800000}%
\pgfsetstrokecolor{currentstroke}%
\pgfsetdash{{2.960000pt}{1.280000pt}}{0.000000pt}%
\pgfpathmoveto{\pgfqpoint{0.378629in}{0.515519in}}%
\pgfpathlineto{\pgfqpoint{2.463090in}{0.515519in}}%
\pgfusepath{stroke}%
\end{pgfscope}%
\begin{pgfscope}%
\definecolor{textcolor}{rgb}{0.150000,0.150000,0.150000}%
\pgfsetstrokecolor{textcolor}%
\pgfsetfillcolor{textcolor}%
\pgftext[x=0.150000in,y=0.475374in,left,base]{\color{textcolor}\rmfamily\fontsize{8.330000}{9.996000}\selectfont \(\displaystyle -6\)}%
\end{pgfscope}%
\begin{pgfscope}%
\pgfpathrectangle{\pgfqpoint{0.378629in}{0.330514in}}{\pgfqpoint{2.084461in}{1.929340in}}%
\pgfusepath{clip}%
\pgfsetbuttcap%
\pgfsetroundjoin%
\pgfsetlinewidth{0.803000pt}%
\definecolor{currentstroke}{rgb}{0.800000,0.800000,0.800000}%
\pgfsetstrokecolor{currentstroke}%
\pgfsetdash{{2.960000pt}{1.280000pt}}{0.000000pt}%
\pgfpathmoveto{\pgfqpoint{0.378629in}{0.779813in}}%
\pgfpathlineto{\pgfqpoint{2.463090in}{0.779813in}}%
\pgfusepath{stroke}%
\end{pgfscope}%
\begin{pgfscope}%
\definecolor{textcolor}{rgb}{0.150000,0.150000,0.150000}%
\pgfsetstrokecolor{textcolor}%
\pgfsetfillcolor{textcolor}%
\pgftext[x=0.150000in,y=0.739667in,left,base]{\color{textcolor}\rmfamily\fontsize{8.330000}{9.996000}\selectfont \(\displaystyle -4\)}%
\end{pgfscope}%
\begin{pgfscope}%
\pgfpathrectangle{\pgfqpoint{0.378629in}{0.330514in}}{\pgfqpoint{2.084461in}{1.929340in}}%
\pgfusepath{clip}%
\pgfsetbuttcap%
\pgfsetroundjoin%
\pgfsetlinewidth{0.803000pt}%
\definecolor{currentstroke}{rgb}{0.800000,0.800000,0.800000}%
\pgfsetstrokecolor{currentstroke}%
\pgfsetdash{{2.960000pt}{1.280000pt}}{0.000000pt}%
\pgfpathmoveto{\pgfqpoint{0.378629in}{1.044106in}}%
\pgfpathlineto{\pgfqpoint{2.463090in}{1.044106in}}%
\pgfusepath{stroke}%
\end{pgfscope}%
\begin{pgfscope}%
\definecolor{textcolor}{rgb}{0.150000,0.150000,0.150000}%
\pgfsetstrokecolor{textcolor}%
\pgfsetfillcolor{textcolor}%
\pgftext[x=0.150000in,y=1.003960in,left,base]{\color{textcolor}\rmfamily\fontsize{8.330000}{9.996000}\selectfont \(\displaystyle -2\)}%
\end{pgfscope}%
\begin{pgfscope}%
\pgfpathrectangle{\pgfqpoint{0.378629in}{0.330514in}}{\pgfqpoint{2.084461in}{1.929340in}}%
\pgfusepath{clip}%
\pgfsetbuttcap%
\pgfsetroundjoin%
\pgfsetlinewidth{0.803000pt}%
\definecolor{currentstroke}{rgb}{0.800000,0.800000,0.800000}%
\pgfsetstrokecolor{currentstroke}%
\pgfsetdash{{2.960000pt}{1.280000pt}}{0.000000pt}%
\pgfpathmoveto{\pgfqpoint{0.378629in}{1.308399in}}%
\pgfpathlineto{\pgfqpoint{2.463090in}{1.308399in}}%
\pgfusepath{stroke}%
\end{pgfscope}%
\begin{pgfscope}%
\definecolor{textcolor}{rgb}{0.150000,0.150000,0.150000}%
\pgfsetstrokecolor{textcolor}%
\pgfsetfillcolor{textcolor}%
\pgftext[x=0.241822in,y=1.268253in,left,base]{\color{textcolor}\rmfamily\fontsize{8.330000}{9.996000}\selectfont \(\displaystyle 0\)}%
\end{pgfscope}%
\begin{pgfscope}%
\pgfpathrectangle{\pgfqpoint{0.378629in}{0.330514in}}{\pgfqpoint{2.084461in}{1.929340in}}%
\pgfusepath{clip}%
\pgfsetbuttcap%
\pgfsetroundjoin%
\pgfsetlinewidth{0.803000pt}%
\definecolor{currentstroke}{rgb}{0.800000,0.800000,0.800000}%
\pgfsetstrokecolor{currentstroke}%
\pgfsetdash{{2.960000pt}{1.280000pt}}{0.000000pt}%
\pgfpathmoveto{\pgfqpoint{0.378629in}{1.572692in}}%
\pgfpathlineto{\pgfqpoint{2.463090in}{1.572692in}}%
\pgfusepath{stroke}%
\end{pgfscope}%
\begin{pgfscope}%
\definecolor{textcolor}{rgb}{0.150000,0.150000,0.150000}%
\pgfsetstrokecolor{textcolor}%
\pgfsetfillcolor{textcolor}%
\pgftext[x=0.241822in,y=1.532546in,left,base]{\color{textcolor}\rmfamily\fontsize{8.330000}{9.996000}\selectfont \(\displaystyle 2\)}%
\end{pgfscope}%
\begin{pgfscope}%
\pgfpathrectangle{\pgfqpoint{0.378629in}{0.330514in}}{\pgfqpoint{2.084461in}{1.929340in}}%
\pgfusepath{clip}%
\pgfsetbuttcap%
\pgfsetroundjoin%
\pgfsetlinewidth{0.803000pt}%
\definecolor{currentstroke}{rgb}{0.800000,0.800000,0.800000}%
\pgfsetstrokecolor{currentstroke}%
\pgfsetdash{{2.960000pt}{1.280000pt}}{0.000000pt}%
\pgfpathmoveto{\pgfqpoint{0.378629in}{1.836985in}}%
\pgfpathlineto{\pgfqpoint{2.463090in}{1.836985in}}%
\pgfusepath{stroke}%
\end{pgfscope}%
\begin{pgfscope}%
\definecolor{textcolor}{rgb}{0.150000,0.150000,0.150000}%
\pgfsetstrokecolor{textcolor}%
\pgfsetfillcolor{textcolor}%
\pgftext[x=0.241822in,y=1.796839in,left,base]{\color{textcolor}\rmfamily\fontsize{8.330000}{9.996000}\selectfont \(\displaystyle 4\)}%
\end{pgfscope}%
\begin{pgfscope}%
\pgfpathrectangle{\pgfqpoint{0.378629in}{0.330514in}}{\pgfqpoint{2.084461in}{1.929340in}}%
\pgfusepath{clip}%
\pgfsetbuttcap%
\pgfsetroundjoin%
\pgfsetlinewidth{0.803000pt}%
\definecolor{currentstroke}{rgb}{0.800000,0.800000,0.800000}%
\pgfsetstrokecolor{currentstroke}%
\pgfsetdash{{2.960000pt}{1.280000pt}}{0.000000pt}%
\pgfpathmoveto{\pgfqpoint{0.378629in}{2.101278in}}%
\pgfpathlineto{\pgfqpoint{2.463090in}{2.101278in}}%
\pgfusepath{stroke}%
\end{pgfscope}%
\begin{pgfscope}%
\definecolor{textcolor}{rgb}{0.150000,0.150000,0.150000}%
\pgfsetstrokecolor{textcolor}%
\pgfsetfillcolor{textcolor}%
\pgftext[x=0.241822in,y=2.061132in,left,base]{\color{textcolor}\rmfamily\fontsize{8.330000}{9.996000}\selectfont \(\displaystyle 6\)}%
\end{pgfscope}%
\begin{pgfscope}%
\pgfpathrectangle{\pgfqpoint{0.378629in}{0.330514in}}{\pgfqpoint{2.084461in}{1.929340in}}%
\pgfusepath{clip}%
\pgfsetbuttcap%
\pgfsetroundjoin%
\pgfsetlinewidth{1.405250pt}%
\definecolor{currentstroke}{rgb}{0.000000,1.000000,0.500000}%
\pgfsetstrokecolor{currentstroke}%
\pgfsetdash{}{0pt}%
\pgfpathmoveto{\pgfqpoint{0.378629in}{0.383373in}}%
\pgfpathlineto{\pgfqpoint{0.726039in}{0.344843in}}%
\pgfpathlineto{\pgfqpoint{0.727079in}{0.344797in}}%
\pgfpathlineto{\pgfqpoint{1.073449in}{0.343729in}}%
\pgfpathlineto{\pgfqpoint{1.420859in}{0.332677in}}%
\pgfpathlineto{\pgfqpoint{1.765121in}{0.333748in}}%
\pgfpathlineto{\pgfqpoint{1.768269in}{0.333748in}}%
\pgfpathlineto{\pgfqpoint{2.113574in}{0.345811in}}%
\pgfpathlineto{\pgfqpoint{2.115680in}{0.345805in}}%
\pgfpathlineto{\pgfqpoint{2.117779in}{0.345565in}}%
\pgfpathlineto{\pgfqpoint{2.289385in}{0.337122in}}%
\pgfpathlineto{\pgfqpoint{2.185162in}{0.330514in}}%
\pgfusepath{stroke}%
\end{pgfscope}%
\begin{pgfscope}%
\pgfpathrectangle{\pgfqpoint{0.378629in}{0.330514in}}{\pgfqpoint{2.084461in}{1.929340in}}%
\pgfusepath{clip}%
\pgfsetbuttcap%
\pgfsetroundjoin%
\pgfsetlinewidth{1.405250pt}%
\definecolor{currentstroke}{rgb}{0.000000,0.752941,0.623529}%
\pgfsetstrokecolor{currentstroke}%
\pgfsetdash{}{0pt}%
\pgfpathmoveto{\pgfqpoint{0.378629in}{0.699567in}}%
\pgfpathlineto{\pgfqpoint{0.724989in}{0.712741in}}%
\pgfpathlineto{\pgfqpoint{0.726039in}{0.712625in}}%
\pgfpathlineto{\pgfqpoint{1.070329in}{0.697320in}}%
\pgfpathlineto{\pgfqpoint{1.073449in}{0.697310in}}%
\pgfpathlineto{\pgfqpoint{1.281895in}{0.692596in}}%
\pgfpathlineto{\pgfqpoint{1.420859in}{0.689325in}}%
\pgfpathlineto{\pgfqpoint{1.766232in}{0.689325in}}%
\pgfpathlineto{\pgfqpoint{1.768269in}{0.689474in}}%
\pgfpathlineto{\pgfqpoint{2.114661in}{0.688395in}}%
\pgfpathlineto{\pgfqpoint{2.115680in}{0.688565in}}%
\pgfpathlineto{\pgfqpoint{2.116684in}{0.688494in}}%
\pgfpathlineto{\pgfqpoint{2.463090in}{0.648965in}}%
\pgfusepath{stroke}%
\end{pgfscope}%
\begin{pgfscope}%
\pgfpathrectangle{\pgfqpoint{0.378629in}{0.330514in}}{\pgfqpoint{2.084461in}{1.929340in}}%
\pgfusepath{clip}%
\pgfsetbuttcap%
\pgfsetroundjoin%
\pgfsetlinewidth{1.405250pt}%
\definecolor{currentstroke}{rgb}{0.000000,0.498039,0.750980}%
\pgfsetstrokecolor{currentstroke}%
\pgfsetdash{}{0pt}%
\pgfpathmoveto{\pgfqpoint{0.378629in}{1.115960in}}%
\pgfpathlineto{\pgfqpoint{0.381721in}{1.115355in}}%
\pgfpathlineto{\pgfqpoint{0.726039in}{1.098739in}}%
\pgfpathlineto{\pgfqpoint{0.727105in}{1.098505in}}%
\pgfpathlineto{\pgfqpoint{1.073449in}{1.073768in}}%
\pgfpathlineto{\pgfqpoint{1.075555in}{1.072858in}}%
\pgfpathlineto{\pgfqpoint{1.420859in}{1.032176in}}%
\pgfpathlineto{\pgfqpoint{1.768269in}{1.057320in}}%
\pgfpathlineto{\pgfqpoint{2.111519in}{1.115560in}}%
\pgfpathlineto{\pgfqpoint{2.115680in}{1.117322in}}%
\pgfpathlineto{\pgfqpoint{2.270084in}{1.104306in}}%
\pgfpathlineto{\pgfqpoint{2.458069in}{1.091044in}}%
\pgfpathlineto{\pgfqpoint{2.463090in}{1.090472in}}%
\pgfusepath{stroke}%
\end{pgfscope}%
\begin{pgfscope}%
\pgfpathrectangle{\pgfqpoint{0.378629in}{0.330514in}}{\pgfqpoint{2.084461in}{1.929340in}}%
\pgfusepath{clip}%
\pgfsetbuttcap%
\pgfsetroundjoin%
\pgfsetlinewidth{1.405250pt}%
\definecolor{currentstroke}{rgb}{0.000000,0.247059,0.876471}%
\pgfsetstrokecolor{currentstroke}%
\pgfsetdash{}{0pt}%
\pgfpathmoveto{\pgfqpoint{0.378629in}{1.751916in}}%
\pgfpathlineto{\pgfqpoint{0.721915in}{1.684722in}}%
\pgfpathlineto{\pgfqpoint{0.726039in}{1.684523in}}%
\pgfpathlineto{\pgfqpoint{0.880443in}{1.650512in}}%
\pgfpathlineto{\pgfqpoint{1.073449in}{1.606180in}}%
\pgfpathlineto{\pgfqpoint{1.078619in}{1.602071in}}%
\pgfpathlineto{\pgfqpoint{1.420859in}{1.455824in}}%
\pgfpathlineto{\pgfqpoint{1.768269in}{1.559477in}}%
\pgfpathlineto{\pgfqpoint{2.114617in}{1.705647in}}%
\pgfpathlineto{\pgfqpoint{2.115680in}{1.705933in}}%
\pgfpathlineto{\pgfqpoint{2.202532in}{1.708142in}}%
\pgfpathlineto{\pgfqpoint{2.463090in}{1.712372in}}%
\pgfusepath{stroke}%
\end{pgfscope}%
\begin{pgfscope}%
\pgfpathrectangle{\pgfqpoint{0.378629in}{0.330514in}}{\pgfqpoint{2.084461in}{1.929340in}}%
\pgfusepath{clip}%
\pgfsetbuttcap%
\pgfsetroundjoin%
\pgfsetlinewidth{1.405250pt}%
\definecolor{currentstroke}{rgb}{0.000000,0.000000,1.000000}%
\pgfsetstrokecolor{currentstroke}%
\pgfsetdash{}{0pt}%
\pgfpathmoveto{\pgfqpoint{2.463090in}{2.101278in}}%
\pgfpathlineto{\pgfqpoint{0.726039in}{2.193781in}}%
\pgfpathlineto{\pgfqpoint{1.071331in}{2.072512in}}%
\pgfpathlineto{\pgfqpoint{1.073449in}{2.072026in}}%
\pgfpathlineto{\pgfqpoint{1.419825in}{1.796751in}}%
\pgfpathlineto{\pgfqpoint{1.420859in}{1.796309in}}%
\pgfpathlineto{\pgfqpoint{1.421909in}{1.796622in}}%
\pgfpathlineto{\pgfqpoint{1.490341in}{1.831699in}}%
\pgfpathlineto{\pgfqpoint{1.305056in}{2.039610in}}%
\pgfpathlineto{\pgfqpoint{1.371229in}{2.071073in}}%
\pgfpathlineto{\pgfqpoint{0.726039in}{2.193781in}}%
\pgfpathlineto{\pgfqpoint{2.463090in}{2.101278in}}%
\pgfusepath{stroke}%
\end{pgfscope}%
\begin{pgfscope}%
\pgfsetrectcap%
\pgfsetmiterjoin%
\pgfsetlinewidth{1.003750pt}%
\definecolor{currentstroke}{rgb}{0.400000,0.400000,0.400000}%
\pgfsetstrokecolor{currentstroke}%
\pgfsetdash{}{0pt}%
\pgfpathmoveto{\pgfqpoint{0.378629in}{0.330514in}}%
\pgfpathlineto{\pgfqpoint{0.378629in}{2.259854in}}%
\pgfusepath{stroke}%
\end{pgfscope}%
\begin{pgfscope}%
\pgfsetrectcap%
\pgfsetmiterjoin%
\pgfsetlinewidth{1.003750pt}%
\definecolor{currentstroke}{rgb}{0.400000,0.400000,0.400000}%
\pgfsetstrokecolor{currentstroke}%
\pgfsetdash{}{0pt}%
\pgfpathmoveto{\pgfqpoint{0.378629in}{0.330514in}}%
\pgfpathlineto{\pgfqpoint{2.463090in}{0.330514in}}%
\pgfusepath{stroke}%
\end{pgfscope}%
\begin{pgfscope}%
\pgfpathrectangle{\pgfqpoint{2.593369in}{0.330514in}}{\pgfqpoint{0.096467in}{1.929340in}}%
\pgfusepath{clip}%
\pgfsetbuttcap%
\pgfsetmiterjoin%
\definecolor{currentfill}{rgb}{1.000000,1.000000,1.000000}%
\pgfsetfillcolor{currentfill}%
\pgfsetlinewidth{0.010037pt}%
\definecolor{currentstroke}{rgb}{1.000000,1.000000,1.000000}%
\pgfsetstrokecolor{currentstroke}%
\pgfsetdash{}{0pt}%
\pgfpathmoveto{\pgfqpoint{2.593369in}{0.330514in}}%
\pgfpathlineto{\pgfqpoint{2.593369in}{0.812849in}}%
\pgfpathlineto{\pgfqpoint{2.593369in}{1.777519in}}%
\pgfpathlineto{\pgfqpoint{2.593369in}{2.259854in}}%
\pgfpathlineto{\pgfqpoint{2.689836in}{2.259854in}}%
\pgfpathlineto{\pgfqpoint{2.689836in}{1.777519in}}%
\pgfpathlineto{\pgfqpoint{2.689836in}{0.812849in}}%
\pgfpathlineto{\pgfqpoint{2.689836in}{0.330514in}}%
\pgfpathclose%
\pgfusepath{stroke,fill}%
\end{pgfscope}%
\begin{pgfscope}%
\pgfpathrectangle{\pgfqpoint{2.593369in}{0.330514in}}{\pgfqpoint{0.096467in}{1.929340in}}%
\pgfusepath{clip}%
\pgfsetbuttcap%
\pgfsetroundjoin%
\pgfsetlinewidth{0.803000pt}%
\definecolor{currentstroke}{rgb}{0.800000,0.800000,0.800000}%
\pgfsetstrokecolor{currentstroke}%
\pgfsetdash{{2.960000pt}{1.280000pt}}{0.000000pt}%
\pgfpathmoveto{\pgfqpoint{2.593369in}{0.330514in}}%
\pgfpathlineto{\pgfqpoint{2.689836in}{0.330514in}}%
\pgfusepath{stroke}%
\end{pgfscope}%
\begin{pgfscope}%
\definecolor{textcolor}{rgb}{0.150000,0.150000,0.150000}%
\pgfsetstrokecolor{textcolor}%
\pgfsetfillcolor{textcolor}%
\pgftext[x=2.767613in,y=0.290368in,left,base]{\color{textcolor}\rmfamily\fontsize{8.330000}{9.996000}\selectfont \(\displaystyle 0.33\)}%
\end{pgfscope}%
\begin{pgfscope}%
\pgfpathrectangle{\pgfqpoint{2.593369in}{0.330514in}}{\pgfqpoint{0.096467in}{1.929340in}}%
\pgfusepath{clip}%
\pgfsetbuttcap%
\pgfsetroundjoin%
\pgfsetlinewidth{0.803000pt}%
\definecolor{currentstroke}{rgb}{0.800000,0.800000,0.800000}%
\pgfsetstrokecolor{currentstroke}%
\pgfsetdash{{2.960000pt}{1.280000pt}}{0.000000pt}%
\pgfpathmoveto{\pgfqpoint{2.593369in}{0.812849in}}%
\pgfpathlineto{\pgfqpoint{2.689836in}{0.812849in}}%
\pgfusepath{stroke}%
\end{pgfscope}%
\begin{pgfscope}%
\definecolor{textcolor}{rgb}{0.150000,0.150000,0.150000}%
\pgfsetstrokecolor{textcolor}%
\pgfsetfillcolor{textcolor}%
\pgftext[x=2.767613in,y=0.772703in,left,base]{\color{textcolor}\rmfamily\fontsize{8.330000}{9.996000}\selectfont \(\displaystyle 0.66\)}%
\end{pgfscope}%
\begin{pgfscope}%
\pgfpathrectangle{\pgfqpoint{2.593369in}{0.330514in}}{\pgfqpoint{0.096467in}{1.929340in}}%
\pgfusepath{clip}%
\pgfsetbuttcap%
\pgfsetroundjoin%
\pgfsetlinewidth{0.803000pt}%
\definecolor{currentstroke}{rgb}{0.800000,0.800000,0.800000}%
\pgfsetstrokecolor{currentstroke}%
\pgfsetdash{{2.960000pt}{1.280000pt}}{0.000000pt}%
\pgfpathmoveto{\pgfqpoint{2.593369in}{1.295184in}}%
\pgfpathlineto{\pgfqpoint{2.689836in}{1.295184in}}%
\pgfusepath{stroke}%
\end{pgfscope}%
\begin{pgfscope}%
\definecolor{textcolor}{rgb}{0.150000,0.150000,0.150000}%
\pgfsetstrokecolor{textcolor}%
\pgfsetfillcolor{textcolor}%
\pgftext[x=2.767613in,y=1.255038in,left,base]{\color{textcolor}\rmfamily\fontsize{8.330000}{9.996000}\selectfont \(\displaystyle 1.00\)}%
\end{pgfscope}%
\begin{pgfscope}%
\pgfpathrectangle{\pgfqpoint{2.593369in}{0.330514in}}{\pgfqpoint{0.096467in}{1.929340in}}%
\pgfusepath{clip}%
\pgfsetbuttcap%
\pgfsetroundjoin%
\pgfsetlinewidth{0.803000pt}%
\definecolor{currentstroke}{rgb}{0.800000,0.800000,0.800000}%
\pgfsetstrokecolor{currentstroke}%
\pgfsetdash{{2.960000pt}{1.280000pt}}{0.000000pt}%
\pgfpathmoveto{\pgfqpoint{2.593369in}{1.777519in}}%
\pgfpathlineto{\pgfqpoint{2.689836in}{1.777519in}}%
\pgfusepath{stroke}%
\end{pgfscope}%
\begin{pgfscope}%
\definecolor{textcolor}{rgb}{0.150000,0.150000,0.150000}%
\pgfsetstrokecolor{textcolor}%
\pgfsetfillcolor{textcolor}%
\pgftext[x=2.767613in,y=1.737373in,left,base]{\color{textcolor}\rmfamily\fontsize{8.330000}{9.996000}\selectfont \(\displaystyle 1.33\)}%
\end{pgfscope}%
\begin{pgfscope}%
\pgfpathrectangle{\pgfqpoint{2.593369in}{0.330514in}}{\pgfqpoint{0.096467in}{1.929340in}}%
\pgfusepath{clip}%
\pgfsetbuttcap%
\pgfsetroundjoin%
\pgfsetlinewidth{0.803000pt}%
\definecolor{currentstroke}{rgb}{0.800000,0.800000,0.800000}%
\pgfsetstrokecolor{currentstroke}%
\pgfsetdash{{2.960000pt}{1.280000pt}}{0.000000pt}%
\pgfpathmoveto{\pgfqpoint{2.593369in}{2.259854in}}%
\pgfpathlineto{\pgfqpoint{2.689836in}{2.259854in}}%
\pgfusepath{stroke}%
\end{pgfscope}%
\begin{pgfscope}%
\definecolor{textcolor}{rgb}{0.150000,0.150000,0.150000}%
\pgfsetstrokecolor{textcolor}%
\pgfsetfillcolor{textcolor}%
\pgftext[x=2.767613in,y=2.219708in,left,base]{\color{textcolor}\rmfamily\fontsize{8.330000}{9.996000}\selectfont \(\displaystyle 1.66\)}%
\end{pgfscope}%
\begin{pgfscope}%
\pgfpathrectangle{\pgfqpoint{2.593369in}{0.330514in}}{\pgfqpoint{0.096467in}{1.929340in}}%
\pgfusepath{clip}%
\pgfsetbuttcap%
\pgfsetroundjoin%
\pgfsetlinewidth{1.405250pt}%
\definecolor{currentstroke}{rgb}{0.000000,1.000000,0.500000}%
\pgfsetstrokecolor{currentstroke}%
\pgfsetdash{}{0pt}%
\pgfpathmoveto{\pgfqpoint{2.593369in}{0.330514in}}%
\pgfpathlineto{\pgfqpoint{2.689836in}{0.330514in}}%
\pgfusepath{stroke}%
\end{pgfscope}%
\begin{pgfscope}%
\pgfpathrectangle{\pgfqpoint{2.593369in}{0.330514in}}{\pgfqpoint{0.096467in}{1.929340in}}%
\pgfusepath{clip}%
\pgfsetbuttcap%
\pgfsetroundjoin%
\pgfsetlinewidth{1.405250pt}%
\definecolor{currentstroke}{rgb}{0.000000,0.752941,0.623529}%
\pgfsetstrokecolor{currentstroke}%
\pgfsetdash{}{0pt}%
\pgfpathmoveto{\pgfqpoint{2.593369in}{0.812849in}}%
\pgfpathlineto{\pgfqpoint{2.689836in}{0.812849in}}%
\pgfusepath{stroke}%
\end{pgfscope}%
\begin{pgfscope}%
\pgfpathrectangle{\pgfqpoint{2.593369in}{0.330514in}}{\pgfqpoint{0.096467in}{1.929340in}}%
\pgfusepath{clip}%
\pgfsetbuttcap%
\pgfsetroundjoin%
\pgfsetlinewidth{1.405250pt}%
\definecolor{currentstroke}{rgb}{0.000000,0.498039,0.750980}%
\pgfsetstrokecolor{currentstroke}%
\pgfsetdash{}{0pt}%
\pgfpathmoveto{\pgfqpoint{2.593369in}{1.295184in}}%
\pgfpathlineto{\pgfqpoint{2.689836in}{1.295184in}}%
\pgfusepath{stroke}%
\end{pgfscope}%
\begin{pgfscope}%
\pgfpathrectangle{\pgfqpoint{2.593369in}{0.330514in}}{\pgfqpoint{0.096467in}{1.929340in}}%
\pgfusepath{clip}%
\pgfsetbuttcap%
\pgfsetroundjoin%
\pgfsetlinewidth{1.405250pt}%
\definecolor{currentstroke}{rgb}{0.000000,0.247059,0.876471}%
\pgfsetstrokecolor{currentstroke}%
\pgfsetdash{}{0pt}%
\pgfpathmoveto{\pgfqpoint{2.593369in}{1.777519in}}%
\pgfpathlineto{\pgfqpoint{2.689836in}{1.777519in}}%
\pgfusepath{stroke}%
\end{pgfscope}%
\begin{pgfscope}%
\pgfpathrectangle{\pgfqpoint{2.593369in}{0.330514in}}{\pgfqpoint{0.096467in}{1.929340in}}%
\pgfusepath{clip}%
\pgfsetbuttcap%
\pgfsetroundjoin%
\pgfsetlinewidth{1.405250pt}%
\definecolor{currentstroke}{rgb}{0.000000,0.000000,1.000000}%
\pgfsetstrokecolor{currentstroke}%
\pgfsetdash{}{0pt}%
\pgfpathmoveto{\pgfqpoint{2.593369in}{2.259854in}}%
\pgfpathlineto{\pgfqpoint{2.689836in}{2.259854in}}%
\pgfusepath{stroke}%
\end{pgfscope}%
\begin{pgfscope}%
\pgfsetbuttcap%
\pgfsetmiterjoin%
\pgfsetlinewidth{1.003750pt}%
\definecolor{currentstroke}{rgb}{0.400000,0.400000,0.400000}%
\pgfsetstrokecolor{currentstroke}%
\pgfsetdash{}{0pt}%
\pgfpathmoveto{\pgfqpoint{2.593369in}{0.330514in}}%
\pgfpathlineto{\pgfqpoint{2.593369in}{0.812849in}}%
\pgfpathlineto{\pgfqpoint{2.593369in}{1.777519in}}%
\pgfpathlineto{\pgfqpoint{2.593369in}{2.259854in}}%
\pgfpathlineto{\pgfqpoint{2.689836in}{2.259854in}}%
\pgfpathlineto{\pgfqpoint{2.689836in}{1.777519in}}%
\pgfpathlineto{\pgfqpoint{2.689836in}{0.812849in}}%
\pgfpathlineto{\pgfqpoint{2.689836in}{0.330514in}}%
\pgfpathclose%
\pgfusepath{stroke}%
\end{pgfscope}%
\end{pgfpicture}%
\makeatother%
\endgroup%


            \caption{Exemplo de níveis com problemas}
            \label{fig:contorno:niveis:problema}
        \end{figure}
    \end{nota}


\subsection{Formatação da Escala de Cores}

    Para colocar os textos do gráfico, como títulos e rótulos, o procedimento é como nos casos anteriores, com as funções \pyline{title}, \pyline{xlabel} e \pyline{ylabel}. O único problema aqui é o com a escala de cores, que, até mesmo pelo modo como funciona, não é possível de se fazer uma função diretamente no \pyplot para descrevê-la. No entanto, as funções do \pyplot, em geral, retornam o objeto que foi criado, então a função \pyline{colorbar} retorna o objeto \pyref{https://matplotlib.org/3.1.0/api/colorbar_api.html\#matplotlib.colorbar.Colorbar}{Colorbar} que foi criado. Esse objeto tem um método próprio para mudar sua descrição, o \pyref{https://matplotlib.org/3.1.0/api/colorbar_api.html\#matplotlib.colorbar.ColorbarBase.set_label}{set\_label}, que é o que pode ser visto no código \ref{code:contorno:completo}.

    \begin{listing}[H]
        \caption{Exemplo de formatação da escala de cores e das curvas de nível}
        \label{code:contorno:completo}

        \pyinclude[firstline=70, lastline=85]{recursos/contorno/contorno.py}
    \end{listing}

    Além disso, podemos ver nas escalas de cores anteriores, como da seção \nameref{sec:contorno:cmap}, que os níveis mais extremos não aparecem muito bem. Para resolver isso, podemos extender um pouco a escala para mostrar valores além dos limites de potencial do gráfico. O argumento para isso é \pyline{extend='both'} em \pyline{tricontour}, que faz extensão em ambos os extremos. Outro argumento poderia ser \pyline{extendrect=True}, que faz a extensão ser retangular, em vez de triangular.


\subsection{Resultado}

    \begin{figure}[H]
        \centering
        %% Creator: Matplotlib, PGF backend
%%
%% To include the figure in your LaTeX document, write
%%   \input{<filename>.pgf}
%%
%% Make sure the required packages are loaded in your preamble
%%   \usepackage{pgf}
%%
%% Figures using additional raster images can only be included by \input if
%% they are in the same directory as the main LaTeX file. For loading figures
%% from other directories you can use the `import` package
%%   \usepackage{import}
%% and then include the figures with
%%   \import{<path to file>}{<filename>.pgf}
%%
%% Matplotlib used the following preamble
%%   
%%       \usepackage[portuguese]{babel}
%%       \usepackage[T1]{fontenc}
%%       \usepackage[utf8]{inputenc}
%%   \usepackage{fontspec}
%%
\begingroup%
\makeatletter%
\begin{pgfpicture}%
\pgfpathrectangle{\pgfpointorigin}{\pgfqpoint{4.500000in}{3.500000in}}%
\pgfusepath{use as bounding box, clip}%
\begin{pgfscope}%
\pgfsetbuttcap%
\pgfsetmiterjoin%
\pgfsetlinewidth{0.000000pt}%
\definecolor{currentstroke}{rgb}{0.000000,0.000000,0.000000}%
\pgfsetstrokecolor{currentstroke}%
\pgfsetstrokeopacity{0.000000}%
\pgfsetdash{}{0pt}%
\pgfpathmoveto{\pgfqpoint{0.000000in}{0.000000in}}%
\pgfpathlineto{\pgfqpoint{4.500000in}{0.000000in}}%
\pgfpathlineto{\pgfqpoint{4.500000in}{3.500000in}}%
\pgfpathlineto{\pgfqpoint{0.000000in}{3.500000in}}%
\pgfpathclose%
\pgfusepath{}%
\end{pgfscope}%
\begin{pgfscope}%
\pgfsetbuttcap%
\pgfsetmiterjoin%
\pgfsetlinewidth{0.000000pt}%
\definecolor{currentstroke}{rgb}{0.000000,0.000000,0.000000}%
\pgfsetstrokecolor{currentstroke}%
\pgfsetstrokeopacity{0.000000}%
\pgfsetdash{}{0pt}%
\pgfpathmoveto{\pgfqpoint{0.719917in}{0.524958in}}%
\pgfpathlineto{\pgfqpoint{3.558802in}{0.524958in}}%
\pgfpathlineto{\pgfqpoint{3.558802in}{3.151000in}}%
\pgfpathlineto{\pgfqpoint{0.719917in}{3.151000in}}%
\pgfpathclose%
\pgfusepath{}%
\end{pgfscope}%
\begin{pgfscope}%
\pgfpathrectangle{\pgfqpoint{0.719917in}{0.524958in}}{\pgfqpoint{2.838885in}{2.626042in}}%
\pgfusepath{clip}%
\pgfsetbuttcap%
\pgfsetroundjoin%
\pgfsetlinewidth{0.803000pt}%
\definecolor{currentstroke}{rgb}{0.800000,0.800000,0.800000}%
\pgfsetstrokecolor{currentstroke}%
\pgfsetdash{{2.960000pt}{1.280000pt}}{0.000000pt}%
\pgfpathmoveto{\pgfqpoint{0.956491in}{0.524958in}}%
\pgfpathlineto{\pgfqpoint{0.956491in}{3.151000in}}%
\pgfusepath{stroke}%
\end{pgfscope}%
\begin{pgfscope}%
\definecolor{textcolor}{rgb}{0.150000,0.150000,0.150000}%
\pgfsetstrokecolor{textcolor}%
\pgfsetfillcolor{textcolor}%
\pgftext[x=0.956491in,y=0.447181in,,top]{\color{textcolor}\rmfamily\fontsize{8.330000}{9.996000}\selectfont \(\displaystyle -10\)}%
\end{pgfscope}%
\begin{pgfscope}%
\pgfpathrectangle{\pgfqpoint{0.719917in}{0.524958in}}{\pgfqpoint{2.838885in}{2.626042in}}%
\pgfusepath{clip}%
\pgfsetbuttcap%
\pgfsetroundjoin%
\pgfsetlinewidth{0.803000pt}%
\definecolor{currentstroke}{rgb}{0.800000,0.800000,0.800000}%
\pgfsetstrokecolor{currentstroke}%
\pgfsetdash{{2.960000pt}{1.280000pt}}{0.000000pt}%
\pgfpathmoveto{\pgfqpoint{1.547925in}{0.524958in}}%
\pgfpathlineto{\pgfqpoint{1.547925in}{3.151000in}}%
\pgfusepath{stroke}%
\end{pgfscope}%
\begin{pgfscope}%
\definecolor{textcolor}{rgb}{0.150000,0.150000,0.150000}%
\pgfsetstrokecolor{textcolor}%
\pgfsetfillcolor{textcolor}%
\pgftext[x=1.547925in,y=0.447181in,,top]{\color{textcolor}\rmfamily\fontsize{8.330000}{9.996000}\selectfont \(\displaystyle -5\)}%
\end{pgfscope}%
\begin{pgfscope}%
\pgfpathrectangle{\pgfqpoint{0.719917in}{0.524958in}}{\pgfqpoint{2.838885in}{2.626042in}}%
\pgfusepath{clip}%
\pgfsetbuttcap%
\pgfsetroundjoin%
\pgfsetlinewidth{0.803000pt}%
\definecolor{currentstroke}{rgb}{0.800000,0.800000,0.800000}%
\pgfsetstrokecolor{currentstroke}%
\pgfsetdash{{2.960000pt}{1.280000pt}}{0.000000pt}%
\pgfpathmoveto{\pgfqpoint{2.139359in}{0.524958in}}%
\pgfpathlineto{\pgfqpoint{2.139359in}{3.151000in}}%
\pgfusepath{stroke}%
\end{pgfscope}%
\begin{pgfscope}%
\definecolor{textcolor}{rgb}{0.150000,0.150000,0.150000}%
\pgfsetstrokecolor{textcolor}%
\pgfsetfillcolor{textcolor}%
\pgftext[x=2.139359in,y=0.447181in,,top]{\color{textcolor}\rmfamily\fontsize{8.330000}{9.996000}\selectfont \(\displaystyle 0\)}%
\end{pgfscope}%
\begin{pgfscope}%
\pgfpathrectangle{\pgfqpoint{0.719917in}{0.524958in}}{\pgfqpoint{2.838885in}{2.626042in}}%
\pgfusepath{clip}%
\pgfsetbuttcap%
\pgfsetroundjoin%
\pgfsetlinewidth{0.803000pt}%
\definecolor{currentstroke}{rgb}{0.800000,0.800000,0.800000}%
\pgfsetstrokecolor{currentstroke}%
\pgfsetdash{{2.960000pt}{1.280000pt}}{0.000000pt}%
\pgfpathmoveto{\pgfqpoint{2.730794in}{0.524958in}}%
\pgfpathlineto{\pgfqpoint{2.730794in}{3.151000in}}%
\pgfusepath{stroke}%
\end{pgfscope}%
\begin{pgfscope}%
\definecolor{textcolor}{rgb}{0.150000,0.150000,0.150000}%
\pgfsetstrokecolor{textcolor}%
\pgfsetfillcolor{textcolor}%
\pgftext[x=2.730794in,y=0.447181in,,top]{\color{textcolor}\rmfamily\fontsize{8.330000}{9.996000}\selectfont \(\displaystyle 5\)}%
\end{pgfscope}%
\begin{pgfscope}%
\pgfpathrectangle{\pgfqpoint{0.719917in}{0.524958in}}{\pgfqpoint{2.838885in}{2.626042in}}%
\pgfusepath{clip}%
\pgfsetbuttcap%
\pgfsetroundjoin%
\pgfsetlinewidth{0.803000pt}%
\definecolor{currentstroke}{rgb}{0.800000,0.800000,0.800000}%
\pgfsetstrokecolor{currentstroke}%
\pgfsetdash{{2.960000pt}{1.280000pt}}{0.000000pt}%
\pgfpathmoveto{\pgfqpoint{3.322228in}{0.524958in}}%
\pgfpathlineto{\pgfqpoint{3.322228in}{3.151000in}}%
\pgfusepath{stroke}%
\end{pgfscope}%
\begin{pgfscope}%
\definecolor{textcolor}{rgb}{0.150000,0.150000,0.150000}%
\pgfsetstrokecolor{textcolor}%
\pgfsetfillcolor{textcolor}%
\pgftext[x=3.322228in,y=0.447181in,,top]{\color{textcolor}\rmfamily\fontsize{8.330000}{9.996000}\selectfont \(\displaystyle 10\)}%
\end{pgfscope}%
\begin{pgfscope}%
\definecolor{textcolor}{rgb}{0.000000,0.000000,0.000000}%
\pgfsetstrokecolor{textcolor}%
\pgfsetfillcolor{textcolor}%
\pgftext[x=2.139359in,y=0.288889in,,top]{\color{textcolor}\rmfamily\fontsize{10.000000}{12.000000}\selectfont Posição X [\(\displaystyle cm\)]}%
\end{pgfscope}%
\begin{pgfscope}%
\pgfpathrectangle{\pgfqpoint{0.719917in}{0.524958in}}{\pgfqpoint{2.838885in}{2.626042in}}%
\pgfusepath{clip}%
\pgfsetbuttcap%
\pgfsetroundjoin%
\pgfsetlinewidth{0.803000pt}%
\definecolor{currentstroke}{rgb}{0.800000,0.800000,0.800000}%
\pgfsetstrokecolor{currentstroke}%
\pgfsetdash{{2.960000pt}{1.280000pt}}{0.000000pt}%
\pgfpathmoveto{\pgfqpoint{0.719917in}{0.776771in}}%
\pgfpathlineto{\pgfqpoint{3.558802in}{0.776771in}}%
\pgfusepath{stroke}%
\end{pgfscope}%
\begin{pgfscope}%
\definecolor{textcolor}{rgb}{0.150000,0.150000,0.150000}%
\pgfsetstrokecolor{textcolor}%
\pgfsetfillcolor{textcolor}%
\pgftext[x=0.491288in,y=0.736625in,left,base]{\color{textcolor}\rmfamily\fontsize{8.330000}{9.996000}\selectfont \(\displaystyle -6\)}%
\end{pgfscope}%
\begin{pgfscope}%
\pgfpathrectangle{\pgfqpoint{0.719917in}{0.524958in}}{\pgfqpoint{2.838885in}{2.626042in}}%
\pgfusepath{clip}%
\pgfsetbuttcap%
\pgfsetroundjoin%
\pgfsetlinewidth{0.803000pt}%
\definecolor{currentstroke}{rgb}{0.800000,0.800000,0.800000}%
\pgfsetstrokecolor{currentstroke}%
\pgfsetdash{{2.960000pt}{1.280000pt}}{0.000000pt}%
\pgfpathmoveto{\pgfqpoint{0.719917in}{1.136502in}}%
\pgfpathlineto{\pgfqpoint{3.558802in}{1.136502in}}%
\pgfusepath{stroke}%
\end{pgfscope}%
\begin{pgfscope}%
\definecolor{textcolor}{rgb}{0.150000,0.150000,0.150000}%
\pgfsetstrokecolor{textcolor}%
\pgfsetfillcolor{textcolor}%
\pgftext[x=0.491288in,y=1.096356in,left,base]{\color{textcolor}\rmfamily\fontsize{8.330000}{9.996000}\selectfont \(\displaystyle -4\)}%
\end{pgfscope}%
\begin{pgfscope}%
\pgfpathrectangle{\pgfqpoint{0.719917in}{0.524958in}}{\pgfqpoint{2.838885in}{2.626042in}}%
\pgfusepath{clip}%
\pgfsetbuttcap%
\pgfsetroundjoin%
\pgfsetlinewidth{0.803000pt}%
\definecolor{currentstroke}{rgb}{0.800000,0.800000,0.800000}%
\pgfsetstrokecolor{currentstroke}%
\pgfsetdash{{2.960000pt}{1.280000pt}}{0.000000pt}%
\pgfpathmoveto{\pgfqpoint{0.719917in}{1.496234in}}%
\pgfpathlineto{\pgfqpoint{3.558802in}{1.496234in}}%
\pgfusepath{stroke}%
\end{pgfscope}%
\begin{pgfscope}%
\definecolor{textcolor}{rgb}{0.150000,0.150000,0.150000}%
\pgfsetstrokecolor{textcolor}%
\pgfsetfillcolor{textcolor}%
\pgftext[x=0.491288in,y=1.456088in,left,base]{\color{textcolor}\rmfamily\fontsize{8.330000}{9.996000}\selectfont \(\displaystyle -2\)}%
\end{pgfscope}%
\begin{pgfscope}%
\pgfpathrectangle{\pgfqpoint{0.719917in}{0.524958in}}{\pgfqpoint{2.838885in}{2.626042in}}%
\pgfusepath{clip}%
\pgfsetbuttcap%
\pgfsetroundjoin%
\pgfsetlinewidth{0.803000pt}%
\definecolor{currentstroke}{rgb}{0.800000,0.800000,0.800000}%
\pgfsetstrokecolor{currentstroke}%
\pgfsetdash{{2.960000pt}{1.280000pt}}{0.000000pt}%
\pgfpathmoveto{\pgfqpoint{0.719917in}{1.855966in}}%
\pgfpathlineto{\pgfqpoint{3.558802in}{1.855966in}}%
\pgfusepath{stroke}%
\end{pgfscope}%
\begin{pgfscope}%
\definecolor{textcolor}{rgb}{0.150000,0.150000,0.150000}%
\pgfsetstrokecolor{textcolor}%
\pgfsetfillcolor{textcolor}%
\pgftext[x=0.583110in,y=1.815820in,left,base]{\color{textcolor}\rmfamily\fontsize{8.330000}{9.996000}\selectfont \(\displaystyle 0\)}%
\end{pgfscope}%
\begin{pgfscope}%
\pgfpathrectangle{\pgfqpoint{0.719917in}{0.524958in}}{\pgfqpoint{2.838885in}{2.626042in}}%
\pgfusepath{clip}%
\pgfsetbuttcap%
\pgfsetroundjoin%
\pgfsetlinewidth{0.803000pt}%
\definecolor{currentstroke}{rgb}{0.800000,0.800000,0.800000}%
\pgfsetstrokecolor{currentstroke}%
\pgfsetdash{{2.960000pt}{1.280000pt}}{0.000000pt}%
\pgfpathmoveto{\pgfqpoint{0.719917in}{2.215698in}}%
\pgfpathlineto{\pgfqpoint{3.558802in}{2.215698in}}%
\pgfusepath{stroke}%
\end{pgfscope}%
\begin{pgfscope}%
\definecolor{textcolor}{rgb}{0.150000,0.150000,0.150000}%
\pgfsetstrokecolor{textcolor}%
\pgfsetfillcolor{textcolor}%
\pgftext[x=0.583110in,y=2.175552in,left,base]{\color{textcolor}\rmfamily\fontsize{8.330000}{9.996000}\selectfont \(\displaystyle 2\)}%
\end{pgfscope}%
\begin{pgfscope}%
\pgfpathrectangle{\pgfqpoint{0.719917in}{0.524958in}}{\pgfqpoint{2.838885in}{2.626042in}}%
\pgfusepath{clip}%
\pgfsetbuttcap%
\pgfsetroundjoin%
\pgfsetlinewidth{0.803000pt}%
\definecolor{currentstroke}{rgb}{0.800000,0.800000,0.800000}%
\pgfsetstrokecolor{currentstroke}%
\pgfsetdash{{2.960000pt}{1.280000pt}}{0.000000pt}%
\pgfpathmoveto{\pgfqpoint{0.719917in}{2.575429in}}%
\pgfpathlineto{\pgfqpoint{3.558802in}{2.575429in}}%
\pgfusepath{stroke}%
\end{pgfscope}%
\begin{pgfscope}%
\definecolor{textcolor}{rgb}{0.150000,0.150000,0.150000}%
\pgfsetstrokecolor{textcolor}%
\pgfsetfillcolor{textcolor}%
\pgftext[x=0.583110in,y=2.535283in,left,base]{\color{textcolor}\rmfamily\fontsize{8.330000}{9.996000}\selectfont \(\displaystyle 4\)}%
\end{pgfscope}%
\begin{pgfscope}%
\pgfpathrectangle{\pgfqpoint{0.719917in}{0.524958in}}{\pgfqpoint{2.838885in}{2.626042in}}%
\pgfusepath{clip}%
\pgfsetbuttcap%
\pgfsetroundjoin%
\pgfsetlinewidth{0.803000pt}%
\definecolor{currentstroke}{rgb}{0.800000,0.800000,0.800000}%
\pgfsetstrokecolor{currentstroke}%
\pgfsetdash{{2.960000pt}{1.280000pt}}{0.000000pt}%
\pgfpathmoveto{\pgfqpoint{0.719917in}{2.935161in}}%
\pgfpathlineto{\pgfqpoint{3.558802in}{2.935161in}}%
\pgfusepath{stroke}%
\end{pgfscope}%
\begin{pgfscope}%
\definecolor{textcolor}{rgb}{0.150000,0.150000,0.150000}%
\pgfsetstrokecolor{textcolor}%
\pgfsetfillcolor{textcolor}%
\pgftext[x=0.583110in,y=2.895015in,left,base]{\color{textcolor}\rmfamily\fontsize{8.330000}{9.996000}\selectfont \(\displaystyle 6\)}%
\end{pgfscope}%
\begin{pgfscope}%
\definecolor{textcolor}{rgb}{0.000000,0.000000,0.000000}%
\pgfsetstrokecolor{textcolor}%
\pgfsetfillcolor{textcolor}%
\pgftext[x=0.435732in,y=1.837979in,,bottom,rotate=90.000000]{\color{textcolor}\rmfamily\fontsize{10.000000}{12.000000}\selectfont Posição Y [\(\displaystyle cm\)]}%
\end{pgfscope}%
\begin{pgfscope}%
\pgfpathrectangle{\pgfqpoint{0.719917in}{0.524958in}}{\pgfqpoint{2.838885in}{2.626042in}}%
\pgfusepath{clip}%
\pgfsetbuttcap%
\pgfsetroundjoin%
\pgfsetlinewidth{1.405250pt}%
\definecolor{currentstroke}{rgb}{0.000000,1.000000,0.500000}%
\pgfsetstrokecolor{currentstroke}%
\pgfsetdash{}{0pt}%
\pgfpathmoveto{\pgfqpoint{0.719917in}{0.609946in}}%
\pgfpathlineto{\pgfqpoint{0.734211in}{0.610490in}}%
\pgfpathlineto{\pgfqpoint{1.193064in}{0.559631in}}%
\pgfpathlineto{\pgfqpoint{1.208647in}{0.558939in}}%
\pgfpathlineto{\pgfqpoint{1.666212in}{0.557529in}}%
\pgfpathlineto{\pgfqpoint{1.680637in}{0.557203in}}%
\pgfpathlineto{\pgfqpoint{2.139359in}{0.542618in}}%
\pgfpathlineto{\pgfqpoint{2.593924in}{0.544032in}}%
\pgfpathlineto{\pgfqpoint{2.612507in}{0.544032in}}%
\pgfpathlineto{\pgfqpoint{3.068449in}{0.559951in}}%
\pgfpathlineto{\pgfqpoint{3.085655in}{0.559899in}}%
\pgfpathlineto{\pgfqpoint{3.102808in}{0.557943in}}%
\pgfpathlineto{\pgfqpoint{3.558802in}{0.535519in}}%
\pgfusepath{stroke}%
\end{pgfscope}%
\begin{pgfscope}%
\pgfpathrectangle{\pgfqpoint{0.719917in}{0.524958in}}{\pgfqpoint{2.838885in}{2.626042in}}%
\pgfusepath{clip}%
\pgfsetbuttcap%
\pgfsetroundjoin%
\pgfsetlinewidth{1.405250pt}%
\definecolor{currentstroke}{rgb}{0.000000,0.756863,0.621569}%
\pgfsetstrokecolor{currentstroke}%
\pgfsetdash{}{0pt}%
\pgfpathmoveto{\pgfqpoint{0.719917in}{1.027279in}}%
\pgfpathlineto{\pgfqpoint{1.191635in}{1.045211in}}%
\pgfpathlineto{\pgfqpoint{1.193064in}{1.045053in}}%
\pgfpathlineto{\pgfqpoint{1.661962in}{1.024221in}}%
\pgfpathlineto{\pgfqpoint{1.666212in}{1.024208in}}%
\pgfpathlineto{\pgfqpoint{1.950100in}{1.017791in}}%
\pgfpathlineto{\pgfqpoint{2.139359in}{1.013339in}}%
\pgfpathlineto{\pgfqpoint{2.609732in}{1.013339in}}%
\pgfpathlineto{\pgfqpoint{2.612507in}{1.013541in}}%
\pgfpathlineto{\pgfqpoint{3.084267in}{1.012073in}}%
\pgfpathlineto{\pgfqpoint{3.085655in}{1.012304in}}%
\pgfpathlineto{\pgfqpoint{3.087022in}{1.012208in}}%
\pgfpathlineto{\pgfqpoint{3.558802in}{0.958404in}}%
\pgfusepath{stroke}%
\end{pgfscope}%
\begin{pgfscope}%
\pgfpathrectangle{\pgfqpoint{0.719917in}{0.524958in}}{\pgfqpoint{2.838885in}{2.626042in}}%
\pgfusepath{clip}%
\pgfsetbuttcap%
\pgfsetroundjoin%
\pgfsetlinewidth{1.405250pt}%
\definecolor{currentstroke}{rgb}{0.000000,0.498039,0.750980}%
\pgfsetstrokecolor{currentstroke}%
\pgfsetdash{}{0pt}%
\pgfpathmoveto{\pgfqpoint{0.719917in}{1.594036in}}%
\pgfpathlineto{\pgfqpoint{0.724129in}{1.593212in}}%
\pgfpathlineto{\pgfqpoint{1.193064in}{1.570597in}}%
\pgfpathlineto{\pgfqpoint{1.194516in}{1.570277in}}%
\pgfpathlineto{\pgfqpoint{1.666212in}{1.536608in}}%
\pgfpathlineto{\pgfqpoint{1.669079in}{1.535369in}}%
\pgfpathlineto{\pgfqpoint{2.139359in}{1.479997in}}%
\pgfpathlineto{\pgfqpoint{2.612507in}{1.514221in}}%
\pgfpathlineto{\pgfqpoint{3.079988in}{1.593491in}}%
\pgfpathlineto{\pgfqpoint{3.085655in}{1.595890in}}%
\pgfpathlineto{\pgfqpoint{3.295942in}{1.578173in}}%
\pgfpathlineto{\pgfqpoint{3.551965in}{1.560123in}}%
\pgfpathlineto{\pgfqpoint{3.558802in}{1.559343in}}%
\pgfusepath{stroke}%
\end{pgfscope}%
\begin{pgfscope}%
\pgfpathrectangle{\pgfqpoint{0.719917in}{0.524958in}}{\pgfqpoint{2.838885in}{2.626042in}}%
\pgfusepath{clip}%
\pgfsetbuttcap%
\pgfsetroundjoin%
\pgfsetlinewidth{1.405250pt}%
\definecolor{currentstroke}{rgb}{0.000000,0.243137,0.878431}%
\pgfsetstrokecolor{currentstroke}%
\pgfsetdash{}{0pt}%
\pgfpathmoveto{\pgfqpoint{0.719917in}{2.459641in}}%
\pgfpathlineto{\pgfqpoint{1.187448in}{2.368183in}}%
\pgfpathlineto{\pgfqpoint{1.193064in}{2.367912in}}%
\pgfpathlineto{\pgfqpoint{1.403352in}{2.321619in}}%
\pgfpathlineto{\pgfqpoint{1.666212in}{2.261278in}}%
\pgfpathlineto{\pgfqpoint{1.673253in}{2.255686in}}%
\pgfpathlineto{\pgfqpoint{2.139359in}{2.056627in}}%
\pgfpathlineto{\pgfqpoint{2.612507in}{2.197711in}}%
\pgfpathlineto{\pgfqpoint{3.084208in}{2.396663in}}%
\pgfpathlineto{\pgfqpoint{3.085655in}{2.397053in}}%
\pgfpathlineto{\pgfqpoint{3.203941in}{2.400060in}}%
\pgfpathlineto{\pgfqpoint{3.558802in}{2.405818in}}%
\pgfusepath{stroke}%
\end{pgfscope}%
\begin{pgfscope}%
\pgfpathrectangle{\pgfqpoint{0.719917in}{0.524958in}}{\pgfqpoint{2.838885in}{2.626042in}}%
\pgfusepath{clip}%
\pgfsetbuttcap%
\pgfsetroundjoin%
\pgfsetlinewidth{1.405250pt}%
\definecolor{currentstroke}{rgb}{0.000000,0.000000,1.000000}%
\pgfsetstrokecolor{currentstroke}%
\pgfsetdash{}{0pt}%
\pgfpathmoveto{\pgfqpoint{0.719917in}{3.138304in}}%
\pgfpathlineto{\pgfqpoint{1.178595in}{3.042915in}}%
\pgfpathlineto{\pgfqpoint{1.193064in}{3.040101in}}%
\pgfpathlineto{\pgfqpoint{1.648902in}{2.880104in}}%
\pgfpathlineto{\pgfqpoint{1.666212in}{2.876131in}}%
\pgfpathlineto{\pgfqpoint{2.123869in}{2.512637in}}%
\pgfpathlineto{\pgfqpoint{2.139359in}{2.506022in}}%
\pgfpathlineto{\pgfqpoint{2.155083in}{2.510710in}}%
\pgfpathlineto{\pgfqpoint{2.612507in}{2.745033in}}%
\pgfpathlineto{\pgfqpoint{2.621189in}{2.748695in}}%
\pgfpathlineto{\pgfqpoint{3.085655in}{2.873753in}}%
\pgfpathlineto{\pgfqpoint{3.093002in}{2.873940in}}%
\pgfpathlineto{\pgfqpoint{3.558802in}{2.919210in}}%
\pgfusepath{stroke}%
\end{pgfscope}%
\begin{pgfscope}%
\pgfsetrectcap%
\pgfsetmiterjoin%
\pgfsetlinewidth{1.003750pt}%
\definecolor{currentstroke}{rgb}{0.400000,0.400000,0.400000}%
\pgfsetstrokecolor{currentstroke}%
\pgfsetdash{}{0pt}%
\pgfpathmoveto{\pgfqpoint{0.719917in}{0.524958in}}%
\pgfpathlineto{\pgfqpoint{0.719917in}{3.151000in}}%
\pgfusepath{stroke}%
\end{pgfscope}%
\begin{pgfscope}%
\pgfsetrectcap%
\pgfsetmiterjoin%
\pgfsetlinewidth{1.003750pt}%
\definecolor{currentstroke}{rgb}{0.400000,0.400000,0.400000}%
\pgfsetstrokecolor{currentstroke}%
\pgfsetdash{}{0pt}%
\pgfpathmoveto{\pgfqpoint{0.719917in}{0.524958in}}%
\pgfpathlineto{\pgfqpoint{3.558802in}{0.524958in}}%
\pgfusepath{stroke}%
\end{pgfscope}%
\begin{pgfscope}%
\definecolor{textcolor}{rgb}{0.000000,0.000000,0.000000}%
\pgfsetstrokecolor{textcolor}%
\pgfsetfillcolor{textcolor}%
\pgftext[x=2.139359in,y=3.234333in,,base]{\color{textcolor}\rmfamily\fontsize{12.000000}{14.400000}\selectfont Potencial na Cuba com uma Ponta em uma das Placas}%
\end{pgfscope}%
\begin{pgfscope}%
\pgfpathrectangle{\pgfqpoint{3.736232in}{0.524958in}}{\pgfqpoint{0.119366in}{2.626042in}}%
\pgfusepath{clip}%
\pgfsetbuttcap%
\pgfsetmiterjoin%
\definecolor{currentfill}{rgb}{1.000000,1.000000,1.000000}%
\pgfsetfillcolor{currentfill}%
\pgfsetlinewidth{0.010037pt}%
\definecolor{currentstroke}{rgb}{1.000000,1.000000,1.000000}%
\pgfsetstrokecolor{currentstroke}%
\pgfsetdash{}{0pt}%
\pgfpathmoveto{\pgfqpoint{3.795915in}{0.524958in}}%
\pgfpathlineto{\pgfqpoint{3.736232in}{0.644324in}}%
\pgfpathlineto{\pgfqpoint{3.736232in}{3.031634in}}%
\pgfpathlineto{\pgfqpoint{3.795915in}{3.151000in}}%
\pgfpathlineto{\pgfqpoint{3.795915in}{3.151000in}}%
\pgfpathlineto{\pgfqpoint{3.855598in}{3.031634in}}%
\pgfpathlineto{\pgfqpoint{3.855598in}{0.644324in}}%
\pgfpathlineto{\pgfqpoint{3.795915in}{0.524958in}}%
\pgfpathclose%
\pgfusepath{stroke,fill}%
\end{pgfscope}%
\begin{pgfscope}%
\pgfpathrectangle{\pgfqpoint{3.736232in}{0.524958in}}{\pgfqpoint{0.119366in}{2.626042in}}%
\pgfusepath{clip}%
\pgfsetbuttcap%
\pgfsetroundjoin%
\pgfsetlinewidth{0.803000pt}%
\definecolor{currentstroke}{rgb}{0.800000,0.800000,0.800000}%
\pgfsetstrokecolor{currentstroke}%
\pgfsetdash{{2.960000pt}{1.280000pt}}{0.000000pt}%
\pgfpathmoveto{\pgfqpoint{3.736232in}{0.644324in}}%
\pgfpathlineto{\pgfqpoint{3.855598in}{0.644324in}}%
\pgfusepath{stroke}%
\end{pgfscope}%
\begin{pgfscope}%
\definecolor{textcolor}{rgb}{0.150000,0.150000,0.150000}%
\pgfsetstrokecolor{textcolor}%
\pgfsetfillcolor{textcolor}%
\pgftext[x=3.933376in,y=0.604178in,left,base]{\color{textcolor}\rmfamily\fontsize{8.330000}{9.996000}\selectfont \(\displaystyle 0.34\)}%
\end{pgfscope}%
\begin{pgfscope}%
\pgfpathrectangle{\pgfqpoint{3.736232in}{0.524958in}}{\pgfqpoint{0.119366in}{2.626042in}}%
\pgfusepath{clip}%
\pgfsetbuttcap%
\pgfsetroundjoin%
\pgfsetlinewidth{0.803000pt}%
\definecolor{currentstroke}{rgb}{0.800000,0.800000,0.800000}%
\pgfsetstrokecolor{currentstroke}%
\pgfsetdash{{2.960000pt}{1.280000pt}}{0.000000pt}%
\pgfpathmoveto{\pgfqpoint{3.736232in}{1.241152in}}%
\pgfpathlineto{\pgfqpoint{3.855598in}{1.241152in}}%
\pgfusepath{stroke}%
\end{pgfscope}%
\begin{pgfscope}%
\definecolor{textcolor}{rgb}{0.150000,0.150000,0.150000}%
\pgfsetstrokecolor{textcolor}%
\pgfsetfillcolor{textcolor}%
\pgftext[x=3.933376in,y=1.201006in,left,base]{\color{textcolor}\rmfamily\fontsize{8.330000}{9.996000}\selectfont \(\displaystyle 0.66\)}%
\end{pgfscope}%
\begin{pgfscope}%
\pgfpathrectangle{\pgfqpoint{3.736232in}{0.524958in}}{\pgfqpoint{0.119366in}{2.626042in}}%
\pgfusepath{clip}%
\pgfsetbuttcap%
\pgfsetroundjoin%
\pgfsetlinewidth{0.803000pt}%
\definecolor{currentstroke}{rgb}{0.800000,0.800000,0.800000}%
\pgfsetstrokecolor{currentstroke}%
\pgfsetdash{{2.960000pt}{1.280000pt}}{0.000000pt}%
\pgfpathmoveto{\pgfqpoint{3.736232in}{1.837979in}}%
\pgfpathlineto{\pgfqpoint{3.855598in}{1.837979in}}%
\pgfusepath{stroke}%
\end{pgfscope}%
\begin{pgfscope}%
\definecolor{textcolor}{rgb}{0.150000,0.150000,0.150000}%
\pgfsetstrokecolor{textcolor}%
\pgfsetfillcolor{textcolor}%
\pgftext[x=3.933376in,y=1.797833in,left,base]{\color{textcolor}\rmfamily\fontsize{8.330000}{9.996000}\selectfont \(\displaystyle 1.00\)}%
\end{pgfscope}%
\begin{pgfscope}%
\pgfpathrectangle{\pgfqpoint{3.736232in}{0.524958in}}{\pgfqpoint{0.119366in}{2.626042in}}%
\pgfusepath{clip}%
\pgfsetbuttcap%
\pgfsetroundjoin%
\pgfsetlinewidth{0.803000pt}%
\definecolor{currentstroke}{rgb}{0.800000,0.800000,0.800000}%
\pgfsetstrokecolor{currentstroke}%
\pgfsetdash{{2.960000pt}{1.280000pt}}{0.000000pt}%
\pgfpathmoveto{\pgfqpoint{3.736232in}{2.434807in}}%
\pgfpathlineto{\pgfqpoint{3.855598in}{2.434807in}}%
\pgfusepath{stroke}%
\end{pgfscope}%
\begin{pgfscope}%
\definecolor{textcolor}{rgb}{0.150000,0.150000,0.150000}%
\pgfsetstrokecolor{textcolor}%
\pgfsetfillcolor{textcolor}%
\pgftext[x=3.933376in,y=2.394661in,left,base]{\color{textcolor}\rmfamily\fontsize{8.330000}{9.996000}\selectfont \(\displaystyle 1.33\)}%
\end{pgfscope}%
\begin{pgfscope}%
\pgfpathrectangle{\pgfqpoint{3.736232in}{0.524958in}}{\pgfqpoint{0.119366in}{2.626042in}}%
\pgfusepath{clip}%
\pgfsetbuttcap%
\pgfsetroundjoin%
\pgfsetlinewidth{0.803000pt}%
\definecolor{currentstroke}{rgb}{0.800000,0.800000,0.800000}%
\pgfsetstrokecolor{currentstroke}%
\pgfsetdash{{2.960000pt}{1.280000pt}}{0.000000pt}%
\pgfpathmoveto{\pgfqpoint{3.736232in}{3.031634in}}%
\pgfpathlineto{\pgfqpoint{3.855598in}{3.031634in}}%
\pgfusepath{stroke}%
\end{pgfscope}%
\begin{pgfscope}%
\definecolor{textcolor}{rgb}{0.150000,0.150000,0.150000}%
\pgfsetstrokecolor{textcolor}%
\pgfsetfillcolor{textcolor}%
\pgftext[x=3.933376in,y=2.991489in,left,base]{\color{textcolor}\rmfamily\fontsize{8.330000}{9.996000}\selectfont \(\displaystyle 1.65\)}%
\end{pgfscope}%
\begin{pgfscope}%
\definecolor{textcolor}{rgb}{0.000000,0.000000,0.000000}%
\pgfsetstrokecolor{textcolor}%
\pgfsetfillcolor{textcolor}%
\pgftext[x=4.198811in,y=1.837979in,,top,rotate=90.000000]{\color{textcolor}\rmfamily\fontsize{10.000000}{12.000000}\selectfont Tensão [\(\displaystyle V\)]}%
\end{pgfscope}%
\begin{pgfscope}%
\pgfpathrectangle{\pgfqpoint{3.736232in}{0.524958in}}{\pgfqpoint{0.119366in}{2.626042in}}%
\pgfusepath{clip}%
\pgfsetbuttcap%
\pgfsetroundjoin%
\pgfsetlinewidth{1.405250pt}%
\definecolor{currentstroke}{rgb}{0.000000,1.000000,0.500000}%
\pgfsetstrokecolor{currentstroke}%
\pgfsetdash{}{0pt}%
\pgfpathmoveto{\pgfqpoint{3.730264in}{0.644324in}}%
\pgfpathlineto{\pgfqpoint{3.861566in}{0.644324in}}%
\pgfusepath{stroke}%
\end{pgfscope}%
\begin{pgfscope}%
\pgfpathrectangle{\pgfqpoint{3.736232in}{0.524958in}}{\pgfqpoint{0.119366in}{2.626042in}}%
\pgfusepath{clip}%
\pgfsetbuttcap%
\pgfsetroundjoin%
\pgfsetlinewidth{1.405250pt}%
\definecolor{currentstroke}{rgb}{0.000000,0.756863,0.621569}%
\pgfsetstrokecolor{currentstroke}%
\pgfsetdash{}{0pt}%
\pgfpathmoveto{\pgfqpoint{3.730264in}{1.241152in}}%
\pgfpathlineto{\pgfqpoint{3.861566in}{1.241152in}}%
\pgfusepath{stroke}%
\end{pgfscope}%
\begin{pgfscope}%
\pgfpathrectangle{\pgfqpoint{3.736232in}{0.524958in}}{\pgfqpoint{0.119366in}{2.626042in}}%
\pgfusepath{clip}%
\pgfsetbuttcap%
\pgfsetroundjoin%
\pgfsetlinewidth{1.405250pt}%
\definecolor{currentstroke}{rgb}{0.000000,0.498039,0.750980}%
\pgfsetstrokecolor{currentstroke}%
\pgfsetdash{}{0pt}%
\pgfpathmoveto{\pgfqpoint{3.730264in}{1.837979in}}%
\pgfpathlineto{\pgfqpoint{3.861566in}{1.837979in}}%
\pgfusepath{stroke}%
\end{pgfscope}%
\begin{pgfscope}%
\pgfpathrectangle{\pgfqpoint{3.736232in}{0.524958in}}{\pgfqpoint{0.119366in}{2.626042in}}%
\pgfusepath{clip}%
\pgfsetbuttcap%
\pgfsetroundjoin%
\pgfsetlinewidth{1.405250pt}%
\definecolor{currentstroke}{rgb}{0.000000,0.243137,0.878431}%
\pgfsetstrokecolor{currentstroke}%
\pgfsetdash{}{0pt}%
\pgfpathmoveto{\pgfqpoint{3.730264in}{2.434807in}}%
\pgfpathlineto{\pgfqpoint{3.861566in}{2.434807in}}%
\pgfusepath{stroke}%
\end{pgfscope}%
\begin{pgfscope}%
\pgfpathrectangle{\pgfqpoint{3.736232in}{0.524958in}}{\pgfqpoint{0.119366in}{2.626042in}}%
\pgfusepath{clip}%
\pgfsetbuttcap%
\pgfsetroundjoin%
\pgfsetlinewidth{1.405250pt}%
\definecolor{currentstroke}{rgb}{0.000000,0.000000,1.000000}%
\pgfsetstrokecolor{currentstroke}%
\pgfsetdash{}{0pt}%
\pgfpathmoveto{\pgfqpoint{3.730264in}{3.031634in}}%
\pgfpathlineto{\pgfqpoint{3.861566in}{3.031634in}}%
\pgfusepath{stroke}%
\end{pgfscope}%
\begin{pgfscope}%
\pgfsetbuttcap%
\pgfsetmiterjoin%
\pgfsetlinewidth{1.003750pt}%
\definecolor{currentstroke}{rgb}{0.400000,0.400000,0.400000}%
\pgfsetstrokecolor{currentstroke}%
\pgfsetdash{}{0pt}%
\pgfpathmoveto{\pgfqpoint{3.795915in}{0.524958in}}%
\pgfpathlineto{\pgfqpoint{3.736232in}{0.644324in}}%
\pgfpathlineto{\pgfqpoint{3.736232in}{3.031634in}}%
\pgfpathlineto{\pgfqpoint{3.795915in}{3.151000in}}%
\pgfpathlineto{\pgfqpoint{3.795915in}{3.151000in}}%
\pgfpathlineto{\pgfqpoint{3.855598in}{3.031634in}}%
\pgfpathlineto{\pgfqpoint{3.855598in}{0.644324in}}%
\pgfpathlineto{\pgfqpoint{3.795915in}{0.524958in}}%
\pgfpathclose%
\pgfusepath{stroke}%
\end{pgfscope}%
\end{pgfpicture}%
\makeatother%
\endgroup%


        \caption{Exemplo completo de curvas de nível}
        \label{fig:contorno:completo}
    \end{figure}
