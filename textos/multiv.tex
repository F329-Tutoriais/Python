\edef\indentacao{\the\parindent}

\noindent
\begin{minipage}[t]{0.55\textwidth}\setlength{\parindent}{\indentacao}

    Para os casos em que é necessário apresentar dados com mais de uma váriavel dependente de um mesmo dado $x$, existe a opção de gráficos múltiplos. Eles servem para comparar as relações do tipo $y_1 = f(x)$ e $y_2 = g(x)$, quando $x$, $y_1$ e $y_2$ são medidos em conjunto.

    Em experimentos com circuitos, esse tipo de dado aparece, por exemplo, na medição de tensão em nós diferentes para a comparação de seus comportamentos no tempo. É o caso do circuito da figura \ref{fig:multiv:circuito}, cujos dados foram os da tabela \ref{tab:multiv:dados}.

    \begin{table}[H]
        \centering
        \begin{tabular}{cccc}
\toprule
        Tempo &           $V_1$ &           $V_2$ &      Corrente \\
\midrule
 0.0 \si{\ms} &  0.0 \si{\volt} &  2.0 \si{\volt} &  -20 \si{\mA} \\
 0.4 \si{\ms} &  1.9 \si{\volt} &  2.3 \si{\volt} &   -3 \si{\mA} \\
 0.8 \si{\ms} &  3.6 \si{\volt} &  2.2 \si{\volt} &   14 \si{\mA} \\
 1.2 \si{\ms} &  4.7 \si{\volt} &  1.8 \si{\volt} &   29 \si{\mA} \\
 1.6 \si{\ms} &  5.0 \si{\volt} &  1.1 \si{\volt} &   39 \si{\mA} \\
 2.0 \si{\ms} &  4.5 \si{\volt} &  0.2 \si{\volt} &   43 \si{\mA} \\
\multicolumn{4}{c}{\dots} \\
\bottomrule
\end{tabular}

        \caption{Dados gerados com simulador. Primeiros 6 valores.}
        \label{tab:multiv:dados}
    \end{table}

\end{minipage}\vspace{0.05\textwidth}%
\begin{minipage}[t]{0.4\textwidth}
    \begin{figure}[H]
        \centering
        \begin{circuitikz}[scale=1.2]

    \draw (2, 0)
    to [short] ++(-2,0)
    to [vsourcesin, l=$E$] ++(0,4)
    to [short] ++(2,0)
    to [resistor, l_=$R$] ++(0,-2)
    to [capacitor, l_=$C$] ++(0,-2)
    node[ground] {};

    \draw [->, thick] (3,4) to node[above] {$V_1$} (2.1,4);

    \draw [->, thick] (3,2) to node[above] {$V_2$} (2.1,2);

\end{circuitikz}


        \caption{Circuito de defasagem da tensão por um capacitor}
        \label{fig:multiv:circuito}
    \end{figure}
\end{minipage}


\subsection{Gráficos de Eixos Compartilhados} \label{sec:multiv:juntos}

    \begin{listing}[H]
        \caption{Montagem completa do gráfico de duas varíaveis com eixos compartilhados}
        \label{code:multiv:juntos}

        \pyinclude[firstline=13, lastline=31]{recursos/multiv/multiv.py}
    \end{listing}

    Esse método é melhor para visualizar a diferença de escala entre os dados, mas, se algum dos dados tem uma escala muito diferente, o gráfico pode acabar pecando na percepção dos dados. Um dos maiores limites para esse método, no entanto, é que as variáveis dependentes precisam ter a mesma motivação física e, por causa disso, a mesma gradeza, caso contrário, o eixo compartilhado entre elas perde completamente o sentido.

    \begin{figure}[htbp]
        \centering
        %% Creator: Matplotlib, PGF backend
%%
%% To include the figure in your LaTeX document, write
%%   \input{<filename>.pgf}
%%
%% Make sure the required packages are loaded in your preamble
%%   \usepackage{pgf}
%%
%% Figures using additional raster images can only be included by \input if
%% they are in the same directory as the main LaTeX file. For loading figures
%% from other directories you can use the `import` package
%%   \usepackage{import}
%% and then include the figures with
%%   \import{<path to file>}{<filename>.pgf}
%%
%% Matplotlib used the following preamble
%%   
%%       \usepackage[portuguese]{babel}
%%       \usepackage[T1]{fontenc}
%%       \usepackage[utf8]{inputenc}
%%   \usepackage{fontspec}
%%
\begingroup%
\makeatletter%
\begin{pgfpicture}%
\pgfpathrectangle{\pgfpointorigin}{\pgfqpoint{4.500000in}{3.500000in}}%
\pgfusepath{use as bounding box, clip}%
\begin{pgfscope}%
\pgfsetbuttcap%
\pgfsetmiterjoin%
\pgfsetlinewidth{0.000000pt}%
\definecolor{currentstroke}{rgb}{0.000000,0.000000,0.000000}%
\pgfsetstrokecolor{currentstroke}%
\pgfsetstrokeopacity{0.000000}%
\pgfsetdash{}{0pt}%
\pgfpathmoveto{\pgfqpoint{0.000000in}{0.000000in}}%
\pgfpathlineto{\pgfqpoint{4.500000in}{0.000000in}}%
\pgfpathlineto{\pgfqpoint{4.500000in}{3.500000in}}%
\pgfpathlineto{\pgfqpoint{0.000000in}{3.500000in}}%
\pgfpathclose%
\pgfusepath{}%
\end{pgfscope}%
\begin{pgfscope}%
\pgfsetbuttcap%
\pgfsetmiterjoin%
\pgfsetlinewidth{0.000000pt}%
\definecolor{currentstroke}{rgb}{0.000000,0.000000,0.000000}%
\pgfsetstrokecolor{currentstroke}%
\pgfsetstrokeopacity{0.000000}%
\pgfsetdash{}{0pt}%
\pgfpathmoveto{\pgfqpoint{0.573073in}{0.524958in}}%
\pgfpathlineto{\pgfqpoint{4.350000in}{0.524958in}}%
\pgfpathlineto{\pgfqpoint{4.350000in}{3.149333in}}%
\pgfpathlineto{\pgfqpoint{0.573073in}{3.149333in}}%
\pgfpathclose%
\pgfusepath{}%
\end{pgfscope}%
\begin{pgfscope}%
\pgfpathrectangle{\pgfqpoint{0.573073in}{0.524958in}}{\pgfqpoint{3.776927in}{2.624375in}}%
\pgfusepath{clip}%
\pgfsetbuttcap%
\pgfsetroundjoin%
\pgfsetlinewidth{0.803000pt}%
\definecolor{currentstroke}{rgb}{0.800000,0.800000,0.800000}%
\pgfsetstrokecolor{currentstroke}%
\pgfsetdash{{2.960000pt}{1.280000pt}}{0.000000pt}%
\pgfpathmoveto{\pgfqpoint{0.744751in}{0.524958in}}%
\pgfpathlineto{\pgfqpoint{0.744751in}{3.149333in}}%
\pgfusepath{stroke}%
\end{pgfscope}%
\begin{pgfscope}%
\definecolor{textcolor}{rgb}{0.150000,0.150000,0.150000}%
\pgfsetstrokecolor{textcolor}%
\pgfsetfillcolor{textcolor}%
\pgftext[x=0.744751in,y=0.447181in,,top]{\color{textcolor}\rmfamily\fontsize{8.330000}{9.996000}\selectfont \(\displaystyle 0\)}%
\end{pgfscope}%
\begin{pgfscope}%
\pgfpathrectangle{\pgfqpoint{0.573073in}{0.524958in}}{\pgfqpoint{3.776927in}{2.624375in}}%
\pgfusepath{clip}%
\pgfsetbuttcap%
\pgfsetroundjoin%
\pgfsetlinewidth{0.803000pt}%
\definecolor{currentstroke}{rgb}{0.800000,0.800000,0.800000}%
\pgfsetstrokecolor{currentstroke}%
\pgfsetdash{{2.960000pt}{1.280000pt}}{0.000000pt}%
\pgfpathmoveto{\pgfqpoint{1.298553in}{0.524958in}}%
\pgfpathlineto{\pgfqpoint{1.298553in}{3.149333in}}%
\pgfusepath{stroke}%
\end{pgfscope}%
\begin{pgfscope}%
\definecolor{textcolor}{rgb}{0.150000,0.150000,0.150000}%
\pgfsetstrokecolor{textcolor}%
\pgfsetfillcolor{textcolor}%
\pgftext[x=1.298553in,y=0.447181in,,top]{\color{textcolor}\rmfamily\fontsize{8.330000}{9.996000}\selectfont \(\displaystyle 2\)}%
\end{pgfscope}%
\begin{pgfscope}%
\pgfpathrectangle{\pgfqpoint{0.573073in}{0.524958in}}{\pgfqpoint{3.776927in}{2.624375in}}%
\pgfusepath{clip}%
\pgfsetbuttcap%
\pgfsetroundjoin%
\pgfsetlinewidth{0.803000pt}%
\definecolor{currentstroke}{rgb}{0.800000,0.800000,0.800000}%
\pgfsetstrokecolor{currentstroke}%
\pgfsetdash{{2.960000pt}{1.280000pt}}{0.000000pt}%
\pgfpathmoveto{\pgfqpoint{1.852355in}{0.524958in}}%
\pgfpathlineto{\pgfqpoint{1.852355in}{3.149333in}}%
\pgfusepath{stroke}%
\end{pgfscope}%
\begin{pgfscope}%
\definecolor{textcolor}{rgb}{0.150000,0.150000,0.150000}%
\pgfsetstrokecolor{textcolor}%
\pgfsetfillcolor{textcolor}%
\pgftext[x=1.852355in,y=0.447181in,,top]{\color{textcolor}\rmfamily\fontsize{8.330000}{9.996000}\selectfont \(\displaystyle 4\)}%
\end{pgfscope}%
\begin{pgfscope}%
\pgfpathrectangle{\pgfqpoint{0.573073in}{0.524958in}}{\pgfqpoint{3.776927in}{2.624375in}}%
\pgfusepath{clip}%
\pgfsetbuttcap%
\pgfsetroundjoin%
\pgfsetlinewidth{0.803000pt}%
\definecolor{currentstroke}{rgb}{0.800000,0.800000,0.800000}%
\pgfsetstrokecolor{currentstroke}%
\pgfsetdash{{2.960000pt}{1.280000pt}}{0.000000pt}%
\pgfpathmoveto{\pgfqpoint{2.406156in}{0.524958in}}%
\pgfpathlineto{\pgfqpoint{2.406156in}{3.149333in}}%
\pgfusepath{stroke}%
\end{pgfscope}%
\begin{pgfscope}%
\definecolor{textcolor}{rgb}{0.150000,0.150000,0.150000}%
\pgfsetstrokecolor{textcolor}%
\pgfsetfillcolor{textcolor}%
\pgftext[x=2.406156in,y=0.447181in,,top]{\color{textcolor}\rmfamily\fontsize{8.330000}{9.996000}\selectfont \(\displaystyle 6\)}%
\end{pgfscope}%
\begin{pgfscope}%
\pgfpathrectangle{\pgfqpoint{0.573073in}{0.524958in}}{\pgfqpoint{3.776927in}{2.624375in}}%
\pgfusepath{clip}%
\pgfsetbuttcap%
\pgfsetroundjoin%
\pgfsetlinewidth{0.803000pt}%
\definecolor{currentstroke}{rgb}{0.800000,0.800000,0.800000}%
\pgfsetstrokecolor{currentstroke}%
\pgfsetdash{{2.960000pt}{1.280000pt}}{0.000000pt}%
\pgfpathmoveto{\pgfqpoint{2.959958in}{0.524958in}}%
\pgfpathlineto{\pgfqpoint{2.959958in}{3.149333in}}%
\pgfusepath{stroke}%
\end{pgfscope}%
\begin{pgfscope}%
\definecolor{textcolor}{rgb}{0.150000,0.150000,0.150000}%
\pgfsetstrokecolor{textcolor}%
\pgfsetfillcolor{textcolor}%
\pgftext[x=2.959958in,y=0.447181in,,top]{\color{textcolor}\rmfamily\fontsize{8.330000}{9.996000}\selectfont \(\displaystyle 8\)}%
\end{pgfscope}%
\begin{pgfscope}%
\pgfpathrectangle{\pgfqpoint{0.573073in}{0.524958in}}{\pgfqpoint{3.776927in}{2.624375in}}%
\pgfusepath{clip}%
\pgfsetbuttcap%
\pgfsetroundjoin%
\pgfsetlinewidth{0.803000pt}%
\definecolor{currentstroke}{rgb}{0.800000,0.800000,0.800000}%
\pgfsetstrokecolor{currentstroke}%
\pgfsetdash{{2.960000pt}{1.280000pt}}{0.000000pt}%
\pgfpathmoveto{\pgfqpoint{3.513760in}{0.524958in}}%
\pgfpathlineto{\pgfqpoint{3.513760in}{3.149333in}}%
\pgfusepath{stroke}%
\end{pgfscope}%
\begin{pgfscope}%
\definecolor{textcolor}{rgb}{0.150000,0.150000,0.150000}%
\pgfsetstrokecolor{textcolor}%
\pgfsetfillcolor{textcolor}%
\pgftext[x=3.513760in,y=0.447181in,,top]{\color{textcolor}\rmfamily\fontsize{8.330000}{9.996000}\selectfont \(\displaystyle 10\)}%
\end{pgfscope}%
\begin{pgfscope}%
\pgfpathrectangle{\pgfqpoint{0.573073in}{0.524958in}}{\pgfqpoint{3.776927in}{2.624375in}}%
\pgfusepath{clip}%
\pgfsetbuttcap%
\pgfsetroundjoin%
\pgfsetlinewidth{0.803000pt}%
\definecolor{currentstroke}{rgb}{0.800000,0.800000,0.800000}%
\pgfsetstrokecolor{currentstroke}%
\pgfsetdash{{2.960000pt}{1.280000pt}}{0.000000pt}%
\pgfpathmoveto{\pgfqpoint{4.067561in}{0.524958in}}%
\pgfpathlineto{\pgfqpoint{4.067561in}{3.149333in}}%
\pgfusepath{stroke}%
\end{pgfscope}%
\begin{pgfscope}%
\definecolor{textcolor}{rgb}{0.150000,0.150000,0.150000}%
\pgfsetstrokecolor{textcolor}%
\pgfsetfillcolor{textcolor}%
\pgftext[x=4.067561in,y=0.447181in,,top]{\color{textcolor}\rmfamily\fontsize{8.330000}{9.996000}\selectfont \(\displaystyle 12\)}%
\end{pgfscope}%
\begin{pgfscope}%
\definecolor{textcolor}{rgb}{0.000000,0.000000,0.000000}%
\pgfsetstrokecolor{textcolor}%
\pgfsetfillcolor{textcolor}%
\pgftext[x=2.461536in,y=0.288889in,,top]{\color{textcolor}\rmfamily\fontsize{10.000000}{12.000000}\selectfont Tempo [\(\displaystyle ms\)]}%
\end{pgfscope}%
\begin{pgfscope}%
\pgfpathrectangle{\pgfqpoint{0.573073in}{0.524958in}}{\pgfqpoint{3.776927in}{2.624375in}}%
\pgfusepath{clip}%
\pgfsetbuttcap%
\pgfsetroundjoin%
\pgfsetlinewidth{0.803000pt}%
\definecolor{currentstroke}{rgb}{0.800000,0.800000,0.800000}%
\pgfsetstrokecolor{currentstroke}%
\pgfsetdash{{2.960000pt}{1.280000pt}}{0.000000pt}%
\pgfpathmoveto{\pgfqpoint{0.573073in}{0.878525in}}%
\pgfpathlineto{\pgfqpoint{4.350000in}{0.878525in}}%
\pgfusepath{stroke}%
\end{pgfscope}%
\begin{pgfscope}%
\definecolor{textcolor}{rgb}{0.150000,0.150000,0.150000}%
\pgfsetstrokecolor{textcolor}%
\pgfsetfillcolor{textcolor}%
\pgftext[x=0.344444in,y=0.838379in,left,base]{\color{textcolor}\rmfamily\fontsize{8.330000}{9.996000}\selectfont \(\displaystyle -4\)}%
\end{pgfscope}%
\begin{pgfscope}%
\pgfpathrectangle{\pgfqpoint{0.573073in}{0.524958in}}{\pgfqpoint{3.776927in}{2.624375in}}%
\pgfusepath{clip}%
\pgfsetbuttcap%
\pgfsetroundjoin%
\pgfsetlinewidth{0.803000pt}%
\definecolor{currentstroke}{rgb}{0.800000,0.800000,0.800000}%
\pgfsetstrokecolor{currentstroke}%
\pgfsetdash{{2.960000pt}{1.280000pt}}{0.000000pt}%
\pgfpathmoveto{\pgfqpoint{0.573073in}{1.356640in}}%
\pgfpathlineto{\pgfqpoint{4.350000in}{1.356640in}}%
\pgfusepath{stroke}%
\end{pgfscope}%
\begin{pgfscope}%
\definecolor{textcolor}{rgb}{0.150000,0.150000,0.150000}%
\pgfsetstrokecolor{textcolor}%
\pgfsetfillcolor{textcolor}%
\pgftext[x=0.344444in,y=1.316494in,left,base]{\color{textcolor}\rmfamily\fontsize{8.330000}{9.996000}\selectfont \(\displaystyle -2\)}%
\end{pgfscope}%
\begin{pgfscope}%
\pgfpathrectangle{\pgfqpoint{0.573073in}{0.524958in}}{\pgfqpoint{3.776927in}{2.624375in}}%
\pgfusepath{clip}%
\pgfsetbuttcap%
\pgfsetroundjoin%
\pgfsetlinewidth{0.803000pt}%
\definecolor{currentstroke}{rgb}{0.800000,0.800000,0.800000}%
\pgfsetstrokecolor{currentstroke}%
\pgfsetdash{{2.960000pt}{1.280000pt}}{0.000000pt}%
\pgfpathmoveto{\pgfqpoint{0.573073in}{1.834755in}}%
\pgfpathlineto{\pgfqpoint{4.350000in}{1.834755in}}%
\pgfusepath{stroke}%
\end{pgfscope}%
\begin{pgfscope}%
\definecolor{textcolor}{rgb}{0.150000,0.150000,0.150000}%
\pgfsetstrokecolor{textcolor}%
\pgfsetfillcolor{textcolor}%
\pgftext[x=0.436267in,y=1.794609in,left,base]{\color{textcolor}\rmfamily\fontsize{8.330000}{9.996000}\selectfont \(\displaystyle 0\)}%
\end{pgfscope}%
\begin{pgfscope}%
\pgfpathrectangle{\pgfqpoint{0.573073in}{0.524958in}}{\pgfqpoint{3.776927in}{2.624375in}}%
\pgfusepath{clip}%
\pgfsetbuttcap%
\pgfsetroundjoin%
\pgfsetlinewidth{0.803000pt}%
\definecolor{currentstroke}{rgb}{0.800000,0.800000,0.800000}%
\pgfsetstrokecolor{currentstroke}%
\pgfsetdash{{2.960000pt}{1.280000pt}}{0.000000pt}%
\pgfpathmoveto{\pgfqpoint{0.573073in}{2.312871in}}%
\pgfpathlineto{\pgfqpoint{4.350000in}{2.312871in}}%
\pgfusepath{stroke}%
\end{pgfscope}%
\begin{pgfscope}%
\definecolor{textcolor}{rgb}{0.150000,0.150000,0.150000}%
\pgfsetstrokecolor{textcolor}%
\pgfsetfillcolor{textcolor}%
\pgftext[x=0.436267in,y=2.272725in,left,base]{\color{textcolor}\rmfamily\fontsize{8.330000}{9.996000}\selectfont \(\displaystyle 2\)}%
\end{pgfscope}%
\begin{pgfscope}%
\pgfpathrectangle{\pgfqpoint{0.573073in}{0.524958in}}{\pgfqpoint{3.776927in}{2.624375in}}%
\pgfusepath{clip}%
\pgfsetbuttcap%
\pgfsetroundjoin%
\pgfsetlinewidth{0.803000pt}%
\definecolor{currentstroke}{rgb}{0.800000,0.800000,0.800000}%
\pgfsetstrokecolor{currentstroke}%
\pgfsetdash{{2.960000pt}{1.280000pt}}{0.000000pt}%
\pgfpathmoveto{\pgfqpoint{0.573073in}{2.790986in}}%
\pgfpathlineto{\pgfqpoint{4.350000in}{2.790986in}}%
\pgfusepath{stroke}%
\end{pgfscope}%
\begin{pgfscope}%
\definecolor{textcolor}{rgb}{0.150000,0.150000,0.150000}%
\pgfsetstrokecolor{textcolor}%
\pgfsetfillcolor{textcolor}%
\pgftext[x=0.436267in,y=2.750840in,left,base]{\color{textcolor}\rmfamily\fontsize{8.330000}{9.996000}\selectfont \(\displaystyle 4\)}%
\end{pgfscope}%
\begin{pgfscope}%
\definecolor{textcolor}{rgb}{0.000000,0.000000,0.000000}%
\pgfsetstrokecolor{textcolor}%
\pgfsetfillcolor{textcolor}%
\pgftext[x=0.288889in,y=1.837146in,,bottom,rotate=90.000000]{\color{textcolor}\rmfamily\fontsize{10.000000}{12.000000}\selectfont Tensão [\(\displaystyle V\)]}%
\end{pgfscope}%
\begin{pgfscope}%
\pgfpathrectangle{\pgfqpoint{0.573073in}{0.524958in}}{\pgfqpoint{3.776927in}{2.624375in}}%
\pgfusepath{clip}%
\pgfsetroundcap%
\pgfsetroundjoin%
\pgfsetlinewidth{1.405250pt}%
\definecolor{currentstroke}{rgb}{1.000000,0.000000,0.000000}%
\pgfsetstrokecolor{currentstroke}%
\pgfsetstrokeopacity{0.600000}%
\pgfsetdash{}{0pt}%
\pgfpathmoveto{\pgfqpoint{0.744751in}{1.834755in}}%
\pgfpathlineto{\pgfqpoint{0.855512in}{2.300918in}}%
\pgfpathlineto{\pgfqpoint{0.966272in}{2.692972in}}%
\pgfpathlineto{\pgfqpoint{1.077032in}{2.948764in}}%
\pgfpathlineto{\pgfqpoint{1.187793in}{3.030044in}}%
\pgfpathlineto{\pgfqpoint{1.298553in}{2.922468in}}%
\pgfpathlineto{\pgfqpoint{1.409313in}{2.642770in}}%
\pgfpathlineto{\pgfqpoint{1.520074in}{2.233982in}}%
\pgfpathlineto{\pgfqpoint{1.630834in}{1.765429in}}%
\pgfpathlineto{\pgfqpoint{1.741594in}{1.306438in}}%
\pgfpathlineto{\pgfqpoint{1.852355in}{0.931117in}}%
\pgfpathlineto{\pgfqpoint{1.963115in}{0.696841in}}%
\pgfpathlineto{\pgfqpoint{2.073875in}{0.644248in}}%
\pgfpathlineto{\pgfqpoint{2.184636in}{0.778121in}}%
\pgfpathlineto{\pgfqpoint{2.295396in}{1.079333in}}%
\pgfpathlineto{\pgfqpoint{2.406156in}{1.500075in}}%
\pgfpathlineto{\pgfqpoint{2.516917in}{1.973409in}}%
\pgfpathlineto{\pgfqpoint{2.627677in}{2.425228in}}%
\pgfpathlineto{\pgfqpoint{2.738437in}{2.783814in}}%
\pgfpathlineto{\pgfqpoint{2.849198in}{2.991794in}}%
\pgfpathlineto{\pgfqpoint{2.959958in}{3.018091in}}%
\pgfpathlineto{\pgfqpoint{3.070718in}{2.855531in}}%
\pgfpathlineto{\pgfqpoint{3.181479in}{2.532804in}}%
\pgfpathlineto{\pgfqpoint{3.292239in}{2.100109in}}%
\pgfpathlineto{\pgfqpoint{3.402999in}{1.626775in}}%
\pgfpathlineto{\pgfqpoint{3.513760in}{1.184519in}}%
\pgfpathlineto{\pgfqpoint{3.624520in}{0.845057in}}%
\pgfpathlineto{\pgfqpoint{3.735280in}{0.663373in}}%
\pgfpathlineto{\pgfqpoint{3.846041in}{0.663373in}}%
\pgfpathlineto{\pgfqpoint{3.956801in}{0.852228in}}%
\pgfpathlineto{\pgfqpoint{4.067561in}{1.194081in}}%
\pgfpathlineto{\pgfqpoint{4.178321in}{1.636337in}}%
\pgfusepath{stroke}%
\end{pgfscope}%
\begin{pgfscope}%
\pgfpathrectangle{\pgfqpoint{0.573073in}{0.524958in}}{\pgfqpoint{3.776927in}{2.624375in}}%
\pgfusepath{clip}%
\pgfsetroundcap%
\pgfsetroundjoin%
\pgfsetlinewidth{1.405250pt}%
\definecolor{currentstroke}{rgb}{0.000000,0.000000,1.000000}%
\pgfsetstrokecolor{currentstroke}%
\pgfsetstrokeopacity{0.600000}%
\pgfsetdash{}{0pt}%
\pgfpathmoveto{\pgfqpoint{0.744751in}{2.310480in}}%
\pgfpathlineto{\pgfqpoint{0.855512in}{2.379807in}}%
\pgfpathlineto{\pgfqpoint{0.966272in}{2.363073in}}%
\pgfpathlineto{\pgfqpoint{1.077032in}{2.262669in}}%
\pgfpathlineto{\pgfqpoint{1.187793in}{2.095328in}}%
\pgfpathlineto{\pgfqpoint{1.298553in}{1.887348in}}%
\pgfpathlineto{\pgfqpoint{1.409313in}{1.669806in}}%
\pgfpathlineto{\pgfqpoint{1.520074in}{1.478559in}}%
\pgfpathlineto{\pgfqpoint{1.630834in}{1.342297in}}%
\pgfpathlineto{\pgfqpoint{1.741594in}{1.284923in}}%
\pgfpathlineto{\pgfqpoint{1.852355in}{1.316000in}}%
\pgfpathlineto{\pgfqpoint{1.963115in}{1.425967in}}%
\pgfpathlineto{\pgfqpoint{2.073875in}{1.602869in}}%
\pgfpathlineto{\pgfqpoint{2.184636in}{1.815631in}}%
\pgfpathlineto{\pgfqpoint{2.295396in}{2.030783in}}%
\pgfpathlineto{\pgfqpoint{2.406156in}{2.214857in}}%
\pgfpathlineto{\pgfqpoint{2.516917in}{2.339167in}}%
\pgfpathlineto{\pgfqpoint{2.627677in}{2.384588in}}%
\pgfpathlineto{\pgfqpoint{2.738437in}{2.341558in}}%
\pgfpathlineto{\pgfqpoint{2.849198in}{2.219638in}}%
\pgfpathlineto{\pgfqpoint{2.959958in}{2.037954in}}%
\pgfpathlineto{\pgfqpoint{3.070718in}{1.822802in}}%
\pgfpathlineto{\pgfqpoint{3.181479in}{1.610041in}}%
\pgfpathlineto{\pgfqpoint{3.292239in}{1.430748in}}%
\pgfpathlineto{\pgfqpoint{3.402999in}{1.318391in}}%
\pgfpathlineto{\pgfqpoint{3.513760in}{1.284923in}}%
\pgfpathlineto{\pgfqpoint{3.624520in}{1.339906in}}%
\pgfpathlineto{\pgfqpoint{3.735280in}{1.471388in}}%
\pgfpathlineto{\pgfqpoint{3.846041in}{1.662634in}}%
\pgfpathlineto{\pgfqpoint{3.956801in}{1.880176in}}%
\pgfpathlineto{\pgfqpoint{4.067561in}{2.088156in}}%
\pgfpathlineto{\pgfqpoint{4.178321in}{2.257887in}}%
\pgfusepath{stroke}%
\end{pgfscope}%
\begin{pgfscope}%
\pgfsetrectcap%
\pgfsetmiterjoin%
\pgfsetlinewidth{1.003750pt}%
\definecolor{currentstroke}{rgb}{0.400000,0.400000,0.400000}%
\pgfsetstrokecolor{currentstroke}%
\pgfsetdash{}{0pt}%
\pgfpathmoveto{\pgfqpoint{0.573073in}{0.524958in}}%
\pgfpathlineto{\pgfqpoint{0.573073in}{3.149333in}}%
\pgfusepath{stroke}%
\end{pgfscope}%
\begin{pgfscope}%
\pgfsetrectcap%
\pgfsetmiterjoin%
\pgfsetlinewidth{1.003750pt}%
\definecolor{currentstroke}{rgb}{0.400000,0.400000,0.400000}%
\pgfsetstrokecolor{currentstroke}%
\pgfsetdash{}{0pt}%
\pgfpathmoveto{\pgfqpoint{0.573073in}{0.524958in}}%
\pgfpathlineto{\pgfqpoint{4.350000in}{0.524958in}}%
\pgfusepath{stroke}%
\end{pgfscope}%
\begin{pgfscope}%
\definecolor{textcolor}{rgb}{0.000000,0.000000,0.000000}%
\pgfsetstrokecolor{textcolor}%
\pgfsetfillcolor{textcolor}%
\pgftext[x=2.461536in,y=3.232667in,,base]{\color{textcolor}\rmfamily\fontsize{12.000000}{14.400000}\selectfont Comportamento da Tensão em um Circuito RC}%
\end{pgfscope}%
\begin{pgfscope}%
\pgfsetbuttcap%
\pgfsetmiterjoin%
\definecolor{currentfill}{rgb}{0.900000,0.900000,0.900000}%
\pgfsetfillcolor{currentfill}%
\pgfsetfillopacity{0.800000}%
\pgfsetlinewidth{0.240900pt}%
\definecolor{currentstroke}{rgb}{0.800000,0.800000,0.800000}%
\pgfsetstrokecolor{currentstroke}%
\pgfsetstrokeopacity{0.800000}%
\pgfsetdash{}{0pt}%
\pgfpathmoveto{\pgfqpoint{0.650851in}{0.580514in}}%
\pgfpathlineto{\pgfqpoint{1.978962in}{0.580514in}}%
\pgfpathquadraticcurveto{\pgfqpoint{2.001184in}{0.580514in}}{\pgfqpoint{2.001184in}{0.602736in}}%
\pgfpathlineto{\pgfqpoint{2.001184in}{0.901403in}}%
\pgfpathquadraticcurveto{\pgfqpoint{2.001184in}{0.923625in}}{\pgfqpoint{1.978962in}{0.923625in}}%
\pgfpathlineto{\pgfqpoint{0.650851in}{0.923625in}}%
\pgfpathquadraticcurveto{\pgfqpoint{0.628628in}{0.923625in}}{\pgfqpoint{0.628628in}{0.901403in}}%
\pgfpathlineto{\pgfqpoint{0.628628in}{0.602736in}}%
\pgfpathquadraticcurveto{\pgfqpoint{0.628628in}{0.580514in}}{\pgfqpoint{0.650851in}{0.580514in}}%
\pgfpathclose%
\pgfusepath{stroke,fill}%
\end{pgfscope}%
\begin{pgfscope}%
\pgfsetroundcap%
\pgfsetroundjoin%
\pgfsetlinewidth{1.405250pt}%
\definecolor{currentstroke}{rgb}{1.000000,0.000000,0.000000}%
\pgfsetstrokecolor{currentstroke}%
\pgfsetstrokeopacity{0.600000}%
\pgfsetdash{}{0pt}%
\pgfpathmoveto{\pgfqpoint{0.673073in}{0.840292in}}%
\pgfpathlineto{\pgfqpoint{0.895295in}{0.840292in}}%
\pgfusepath{stroke}%
\end{pgfscope}%
\begin{pgfscope}%
\definecolor{textcolor}{rgb}{0.000000,0.000000,0.000000}%
\pgfsetstrokecolor{textcolor}%
\pgfsetfillcolor{textcolor}%
\pgftext[x=0.984184in,y=0.801403in,left,base]{\color{textcolor}\rmfamily\fontsize{8.000000}{9.600000}\selectfont Tensão de Entrada}%
\end{pgfscope}%
\begin{pgfscope}%
\pgfsetroundcap%
\pgfsetroundjoin%
\pgfsetlinewidth{1.405250pt}%
\definecolor{currentstroke}{rgb}{0.000000,0.000000,1.000000}%
\pgfsetstrokecolor{currentstroke}%
\pgfsetstrokeopacity{0.600000}%
\pgfsetdash{}{0pt}%
\pgfpathmoveto{\pgfqpoint{0.673073in}{0.685403in}}%
\pgfpathlineto{\pgfqpoint{0.895295in}{0.685403in}}%
\pgfusepath{stroke}%
\end{pgfscope}%
\begin{pgfscope}%
\definecolor{textcolor}{rgb}{0.000000,0.000000,0.000000}%
\pgfsetstrokecolor{textcolor}%
\pgfsetfillcolor{textcolor}%
\pgftext[x=0.984184in,y=0.646514in,left,base]{\color{textcolor}\rmfamily\fontsize{8.000000}{9.600000}\selectfont Tensão de Saída}%
\end{pgfscope}%
\end{pgfpicture}%
\makeatother%
\endgroup%


        \caption{Exemplo de gráfico com as curvas das duas tensões $V_1$ e $V_2$}
        \label{fig:multiv:juntos}
    \end{figure}


\subsection{Gráficos com Apenas a Abscissa Comum}

    Se as escalas entre $y_1$ e $y_2$ forem muito diferentes ou se as grandezas forem diferentes, uma opção viável é montar um gráfico de três eixos. Pra fazer isso no \matplotlib é preciso tratar diretamente dos \emph{eixos} do gráfico, que são instâncias da classe \pyref{https://matplotlib.org/3.1.0/api/axes_api.html\#matplotlib.axes.Axes}{Axes}. Normalmente, a interface \pyplot gera esse objetos automaticamente, quando necessário, mas aqui vamos precisar tratar da construção deles também, que pode ser feito com a função \pyref{https://matplotlib.org/3.1.0/api/_as_gen/matplotlib.pyplot.subplot.html}{subplot}, como no código \ref{code:multiv:duplo}.

    O outro objeto, com o terceiro eixo, é um tipo de eixo chamado de \emph{gêmeo}, já que o eixo $x$ dele é o mesmo que o do primeiro. Para criar um eixo gêmeo em $x$ é com o método \pyref{https://matplotlib.org/3.1.0/api/_as_gen/matplotlib.axes.Axes.twinx.html}{twinx}. Para desenhar em cada eixo, as \href{https://matplotlib.org/3.1.0/api/axes_api.html\#plotting}{funções} são parecidas com as do \pyplot, como função \pyref{https://matplotlib.org/3.1.0/api/_as_gen/matplotlib.axes.Axes.plot.html}{plot}, mas aqui se deve tomar cuidado com qual eixo se deseja desenhar.

    \begin{listing}[H]
        \caption{Montagem completa do gráfico de duas varíaveis com abscissa compartilhada}
        \label{code:multiv:duplo}

        \pyinclude[firstline=36, lastline=66]{recursos/multiv/multiv.py}
    \end{listing}

    Os métodos de formatação dos \pyline{Axes} também são semelhantes às funções de formatação do \pyplot, como \pyref{https://matplotlib.org/3.1.0/api/_as_gen/matplotlib.axes.Axes.grid.html}{grid} e \pyref{https://matplotlib.org/3.1.0/api/_as_gen/matplotlib.axes.Axes.set_ylabel.html}{set\_ylabel}. Os métodos \pyref{https://matplotlib.org/3.1.0/api/_as_gen/matplotlib.axes.Axes.set_xlabel.html}{set\_xlabel} e \pyref{https://matplotlib.org/3.1.0/api/_as_gen/matplotlib.axes.Axes.set_title.html}{set\_title} também são idênticos aos do \pyplot, mas nesse gráfico tem um detalhe a mais: como o segundo eixo é um gêmeo em $x$ do primeiro eixo, o título e o rótulo do eixo $x$ não podem ser modificados pelo eixo gêmeo.

    Apesar de ter gerado o eixo gêmeo, o \matplotlib está configurado para não mostrar a coluna (\textit{spine}) na parte direita do gráfico. Isso pode ser alterado mexendo diretamente com as instâncias da classe \pyref{https://matplotlib.org/3.1.0/api/spines_api.html\#matplotlib.spines.Spine}{Spine}, que podem ser acessadas pelo eixo, através de um dicionário que relaciona o nome da coluna com o objeto dela. No caso, vamos acessar a coluna de nome \pyline{'right'} e mudar sua visibilidade com \pyref{https://matplotlib.org/3.1.0/gallery/ticks_and_spines/spines.html}{set\_visible}.

    Para deixar os eixos mais reconhecíveis, é possível colocar legenda neles, mas elas seriam redundantes aqui. Uma outra opção, é o reconhecimento por cor, que pode ser feito na coluna do eixo com os métodos \pyref{https://matplotlib.org/3.1.0/api/spines_api.html?highlight=spine\#matplotlib.spines.Spine.set_color}{set\_color} e \pyref{https://matplotlib.org/3.1.0/api/_as_gen/matplotlib.patches.Patch.html\#matplotlib.patches.Patch.set_alpha}{set\_alpha}. Também foi removido a coluna esquerda do eixo gêmeo para evitar sobreposição visual das colunas.

    \begin{nota}
        As vezes, os gráficos com múltiplas curvas podem ficar sobrecarregados de informação. Quando isso acontece, o melhor é separar os dados em gráficos distintos pra manter a legibilidade. Gráficos de três eixos podem ficar complicados com facilidade.
    \end{nota}

    \begin{figure}[H]
        \centering
        %% Creator: Matplotlib, PGF backend
%%
%% To include the figure in your LaTeX document, write
%%   \input{<filename>.pgf}
%%
%% Make sure the required packages are loaded in your preamble
%%   \usepackage{pgf}
%%
%% Figures using additional raster images can only be included by \input if
%% they are in the same directory as the main LaTeX file. For loading figures
%% from other directories you can use the `import` package
%%   \usepackage{import}
%% and then include the figures with
%%   \import{<path to file>}{<filename>.pgf}
%%
%% Matplotlib used the following preamble
%%   
%%       \usepackage[portuguese]{babel}
%%       \usepackage[T1]{fontenc}
%%       \usepackage[utf8]{inputenc}
%%   \usepackage{fontspec}
%%
\begingroup%
\makeatletter%
\begin{pgfpicture}%
\pgfpathrectangle{\pgfpointorigin}{\pgfqpoint{4.500000in}{3.500000in}}%
\pgfusepath{use as bounding box, clip}%
\begin{pgfscope}%
\pgfsetbuttcap%
\pgfsetmiterjoin%
\definecolor{currentfill}{rgb}{1.000000,1.000000,1.000000}%
\pgfsetfillcolor{currentfill}%
\pgfsetlinewidth{0.000000pt}%
\definecolor{currentstroke}{rgb}{1.000000,1.000000,1.000000}%
\pgfsetstrokecolor{currentstroke}%
\pgfsetdash{}{0pt}%
\pgfpathmoveto{\pgfqpoint{0.000000in}{0.000000in}}%
\pgfpathlineto{\pgfqpoint{4.500000in}{0.000000in}}%
\pgfpathlineto{\pgfqpoint{4.500000in}{3.500000in}}%
\pgfpathlineto{\pgfqpoint{0.000000in}{3.500000in}}%
\pgfpathclose%
\pgfusepath{fill}%
\end{pgfscope}%
\begin{pgfscope}%
\pgfsetbuttcap%
\pgfsetmiterjoin%
\definecolor{currentfill}{rgb}{1.000000,1.000000,1.000000}%
\pgfsetfillcolor{currentfill}%
\pgfsetlinewidth{0.000000pt}%
\definecolor{currentstroke}{rgb}{0.000000,0.000000,0.000000}%
\pgfsetstrokecolor{currentstroke}%
\pgfsetstrokeopacity{0.000000}%
\pgfsetdash{}{0pt}%
\pgfpathmoveto{\pgfqpoint{0.573073in}{0.524958in}}%
\pgfpathlineto{\pgfqpoint{3.867898in}{0.524958in}}%
\pgfpathlineto{\pgfqpoint{3.867898in}{3.149333in}}%
\pgfpathlineto{\pgfqpoint{0.573073in}{3.149333in}}%
\pgfpathclose%
\pgfusepath{fill}%
\end{pgfscope}%
\begin{pgfscope}%
\definecolor{textcolor}{rgb}{0.150000,0.150000,0.150000}%
\pgfsetstrokecolor{textcolor}%
\pgfsetfillcolor{textcolor}%
\pgftext[x=0.722838in,y=0.447181in,,top]{\color{textcolor}\rmfamily\fontsize{8.330000}{9.996000}\selectfont \(\displaystyle 0\)}%
\end{pgfscope}%
\begin{pgfscope}%
\definecolor{textcolor}{rgb}{0.150000,0.150000,0.150000}%
\pgfsetstrokecolor{textcolor}%
\pgfsetfillcolor{textcolor}%
\pgftext[x=1.205950in,y=0.447181in,,top]{\color{textcolor}\rmfamily\fontsize{8.330000}{9.996000}\selectfont \(\displaystyle 2\)}%
\end{pgfscope}%
\begin{pgfscope}%
\definecolor{textcolor}{rgb}{0.150000,0.150000,0.150000}%
\pgfsetstrokecolor{textcolor}%
\pgfsetfillcolor{textcolor}%
\pgftext[x=1.689062in,y=0.447181in,,top]{\color{textcolor}\rmfamily\fontsize{8.330000}{9.996000}\selectfont \(\displaystyle 4\)}%
\end{pgfscope}%
\begin{pgfscope}%
\definecolor{textcolor}{rgb}{0.150000,0.150000,0.150000}%
\pgfsetstrokecolor{textcolor}%
\pgfsetfillcolor{textcolor}%
\pgftext[x=2.172174in,y=0.447181in,,top]{\color{textcolor}\rmfamily\fontsize{8.330000}{9.996000}\selectfont \(\displaystyle 6\)}%
\end{pgfscope}%
\begin{pgfscope}%
\definecolor{textcolor}{rgb}{0.150000,0.150000,0.150000}%
\pgfsetstrokecolor{textcolor}%
\pgfsetfillcolor{textcolor}%
\pgftext[x=2.655287in,y=0.447181in,,top]{\color{textcolor}\rmfamily\fontsize{8.330000}{9.996000}\selectfont \(\displaystyle 8\)}%
\end{pgfscope}%
\begin{pgfscope}%
\definecolor{textcolor}{rgb}{0.150000,0.150000,0.150000}%
\pgfsetstrokecolor{textcolor}%
\pgfsetfillcolor{textcolor}%
\pgftext[x=3.138399in,y=0.447181in,,top]{\color{textcolor}\rmfamily\fontsize{8.330000}{9.996000}\selectfont \(\displaystyle 10\)}%
\end{pgfscope}%
\begin{pgfscope}%
\definecolor{textcolor}{rgb}{0.150000,0.150000,0.150000}%
\pgfsetstrokecolor{textcolor}%
\pgfsetfillcolor{textcolor}%
\pgftext[x=3.621511in,y=0.447181in,,top]{\color{textcolor}\rmfamily\fontsize{8.330000}{9.996000}\selectfont \(\displaystyle 12\)}%
\end{pgfscope}%
\begin{pgfscope}%
\definecolor{textcolor}{rgb}{0.000000,0.000000,0.000000}%
\pgfsetstrokecolor{textcolor}%
\pgfsetfillcolor{textcolor}%
\pgftext[x=2.220486in,y=0.288889in,,top]{\color{textcolor}\rmfamily\fontsize{10.000000}{12.000000}\selectfont Tempo [\(\displaystyle ms\)]}%
\end{pgfscope}%
\begin{pgfscope}%
\definecolor{textcolor}{rgb}{0.150000,0.150000,0.150000}%
\pgfsetstrokecolor{textcolor}%
\pgfsetfillcolor{textcolor}%
\pgftext[x=0.344444in,y=0.759698in,left,base]{\color{textcolor}\rmfamily\fontsize{8.330000}{9.996000}\selectfont \(\displaystyle -2\)}%
\end{pgfscope}%
\begin{pgfscope}%
\definecolor{textcolor}{rgb}{0.150000,0.150000,0.150000}%
\pgfsetstrokecolor{textcolor}%
\pgfsetfillcolor{textcolor}%
\pgftext[x=0.344444in,y=1.278349in,left,base]{\color{textcolor}\rmfamily\fontsize{8.330000}{9.996000}\selectfont \(\displaystyle -1\)}%
\end{pgfscope}%
\begin{pgfscope}%
\definecolor{textcolor}{rgb}{0.150000,0.150000,0.150000}%
\pgfsetstrokecolor{textcolor}%
\pgfsetfillcolor{textcolor}%
\pgftext[x=0.436267in,y=1.797000in,left,base]{\color{textcolor}\rmfamily\fontsize{8.330000}{9.996000}\selectfont \(\displaystyle 0\)}%
\end{pgfscope}%
\begin{pgfscope}%
\definecolor{textcolor}{rgb}{0.150000,0.150000,0.150000}%
\pgfsetstrokecolor{textcolor}%
\pgfsetfillcolor{textcolor}%
\pgftext[x=0.436267in,y=2.315651in,left,base]{\color{textcolor}\rmfamily\fontsize{8.330000}{9.996000}\selectfont \(\displaystyle 1\)}%
\end{pgfscope}%
\begin{pgfscope}%
\definecolor{textcolor}{rgb}{0.150000,0.150000,0.150000}%
\pgfsetstrokecolor{textcolor}%
\pgfsetfillcolor{textcolor}%
\pgftext[x=0.436267in,y=2.834302in,left,base]{\color{textcolor}\rmfamily\fontsize{8.330000}{9.996000}\selectfont \(\displaystyle 2\)}%
\end{pgfscope}%
\begin{pgfscope}%
\definecolor{textcolor}{rgb}{0.000000,0.000000,0.000000}%
\pgfsetstrokecolor{textcolor}%
\pgfsetfillcolor{textcolor}%
\pgftext[x=0.288889in,y=1.837146in,,bottom,rotate=90.000000]{\color{textcolor}\rmfamily\fontsize{10.000000}{12.000000}\selectfont Tensão [\(\displaystyle V\)]}%
\end{pgfscope}%
\begin{pgfscope}%
\pgfpathrectangle{\pgfqpoint{0.573073in}{0.524958in}}{\pgfqpoint{3.294826in}{2.624375in}}%
\pgfusepath{clip}%
\pgfsetroundcap%
\pgfsetroundjoin%
\pgfsetlinewidth{1.405250pt}%
\definecolor{currentstroke}{rgb}{1.000000,0.000000,0.000000}%
\pgfsetstrokecolor{currentstroke}%
\pgfsetstrokeopacity{0.600000}%
\pgfsetdash{}{0pt}%
\pgfpathmoveto{\pgfqpoint{0.722838in}{2.869262in}}%
\pgfpathlineto{\pgfqpoint{0.819460in}{3.019671in}}%
\pgfpathlineto{\pgfqpoint{0.916083in}{2.983365in}}%
\pgfpathlineto{\pgfqpoint{1.012705in}{2.765531in}}%
\pgfpathlineto{\pgfqpoint{1.109328in}{2.402476in}}%
\pgfpathlineto{\pgfqpoint{1.205950in}{1.951249in}}%
\pgfpathlineto{\pgfqpoint{1.302572in}{1.479277in}}%
\pgfpathlineto{\pgfqpoint{1.399195in}{1.064356in}}%
\pgfpathlineto{\pgfqpoint{1.495817in}{0.768725in}}%
\pgfpathlineto{\pgfqpoint{1.592440in}{0.644248in}}%
\pgfpathlineto{\pgfqpoint{1.689062in}{0.711673in}}%
\pgfpathlineto{\pgfqpoint{1.785685in}{0.950252in}}%
\pgfpathlineto{\pgfqpoint{1.882307in}{1.334054in}}%
\pgfpathlineto{\pgfqpoint{1.978930in}{1.795654in}}%
\pgfpathlineto{\pgfqpoint{2.075552in}{2.262440in}}%
\pgfpathlineto{\pgfqpoint{2.172174in}{2.661801in}}%
\pgfpathlineto{\pgfqpoint{2.268797in}{2.931500in}}%
\pgfpathlineto{\pgfqpoint{2.365419in}{3.030044in}}%
\pgfpathlineto{\pgfqpoint{2.462042in}{2.936686in}}%
\pgfpathlineto{\pgfqpoint{2.558664in}{2.672174in}}%
\pgfpathlineto{\pgfqpoint{2.655287in}{2.277999in}}%
\pgfpathlineto{\pgfqpoint{2.751909in}{1.811213in}}%
\pgfpathlineto{\pgfqpoint{2.848532in}{1.349614in}}%
\pgfpathlineto{\pgfqpoint{2.945154in}{0.960625in}}%
\pgfpathlineto{\pgfqpoint{3.041777in}{0.716859in}}%
\pgfpathlineto{\pgfqpoint{3.138399in}{0.644248in}}%
\pgfpathlineto{\pgfqpoint{3.235021in}{0.763538in}}%
\pgfpathlineto{\pgfqpoint{3.331644in}{1.048796in}}%
\pgfpathlineto{\pgfqpoint{3.428266in}{1.463717in}}%
\pgfpathlineto{\pgfqpoint{3.524889in}{1.935690in}}%
\pgfpathlineto{\pgfqpoint{3.621511in}{2.386916in}}%
\pgfpathlineto{\pgfqpoint{3.718134in}{2.755158in}}%
\pgfusepath{stroke}%
\end{pgfscope}%
\begin{pgfscope}%
\pgfsetrectcap%
\pgfsetmiterjoin%
\pgfsetlinewidth{1.003750pt}%
\definecolor{currentstroke}{rgb}{1.000000,0.000000,0.000000}%
\pgfsetstrokecolor{currentstroke}%
\pgfsetstrokeopacity{0.600000}%
\pgfsetdash{}{0pt}%
\pgfpathmoveto{\pgfqpoint{0.573073in}{0.524958in}}%
\pgfpathlineto{\pgfqpoint{0.573073in}{3.149333in}}%
\pgfusepath{stroke}%
\end{pgfscope}%
\begin{pgfscope}%
\pgfsetrectcap%
\pgfsetmiterjoin%
\pgfsetlinewidth{1.003750pt}%
\definecolor{currentstroke}{rgb}{0.400000,0.400000,0.400000}%
\pgfsetstrokecolor{currentstroke}%
\pgfsetdash{}{0pt}%
\pgfpathmoveto{\pgfqpoint{0.573073in}{0.524958in}}%
\pgfpathlineto{\pgfqpoint{3.867898in}{0.524958in}}%
\pgfusepath{stroke}%
\end{pgfscope}%
\begin{pgfscope}%
\definecolor{textcolor}{rgb}{0.000000,0.000000,0.000000}%
\pgfsetstrokecolor{textcolor}%
\pgfsetfillcolor{textcolor}%
\pgftext[x=2.220486in,y=3.232667in,,base]{\color{textcolor}\rmfamily\fontsize{12.000000}{14.400000}\selectfont Relação de Corrente e Tensão em um Capacitor}%
\end{pgfscope}%
\begin{pgfscope}%
\definecolor{textcolor}{rgb}{0.150000,0.150000,0.150000}%
\pgfsetstrokecolor{textcolor}%
\pgfsetfillcolor{textcolor}%
\pgftext[x=3.945676in,y=0.696216in,left,base]{\color{textcolor}\rmfamily\fontsize{8.330000}{9.996000}\selectfont \(\displaystyle -40\)}%
\end{pgfscope}%
\begin{pgfscope}%
\definecolor{textcolor}{rgb}{0.150000,0.150000,0.150000}%
\pgfsetstrokecolor{textcolor}%
\pgfsetfillcolor{textcolor}%
\pgftext[x=3.945676in,y=1.246940in,left,base]{\color{textcolor}\rmfamily\fontsize{8.330000}{9.996000}\selectfont \(\displaystyle -20\)}%
\end{pgfscope}%
\begin{pgfscope}%
\definecolor{textcolor}{rgb}{0.150000,0.150000,0.150000}%
\pgfsetstrokecolor{textcolor}%
\pgfsetfillcolor{textcolor}%
\pgftext[x=3.945676in,y=1.797664in,left,base]{\color{textcolor}\rmfamily\fontsize{8.330000}{9.996000}\selectfont \(\displaystyle 0\)}%
\end{pgfscope}%
\begin{pgfscope}%
\definecolor{textcolor}{rgb}{0.150000,0.150000,0.150000}%
\pgfsetstrokecolor{textcolor}%
\pgfsetfillcolor{textcolor}%
\pgftext[x=3.945676in,y=2.348387in,left,base]{\color{textcolor}\rmfamily\fontsize{8.330000}{9.996000}\selectfont \(\displaystyle 20\)}%
\end{pgfscope}%
\begin{pgfscope}%
\definecolor{textcolor}{rgb}{0.150000,0.150000,0.150000}%
\pgfsetstrokecolor{textcolor}%
\pgfsetfillcolor{textcolor}%
\pgftext[x=3.945676in,y=2.899111in,left,base]{\color{textcolor}\rmfamily\fontsize{8.330000}{9.996000}\selectfont \(\displaystyle 40\)}%
\end{pgfscope}%
\begin{pgfscope}%
\definecolor{textcolor}{rgb}{0.000000,0.000000,0.000000}%
\pgfsetstrokecolor{textcolor}%
\pgfsetfillcolor{textcolor}%
\pgftext[x=4.211111in,y=1.837146in,,top,rotate=90.000000]{\color{textcolor}\rmfamily\fontsize{10.000000}{12.000000}\selectfont Corrente [\(\displaystyle mA\)]}%
\end{pgfscope}%
\begin{pgfscope}%
\pgfpathrectangle{\pgfqpoint{0.573073in}{0.524958in}}{\pgfqpoint{3.294826in}{2.624375in}}%
\pgfusepath{clip}%
\pgfsetroundcap%
\pgfsetroundjoin%
\pgfsetlinewidth{1.405250pt}%
\definecolor{currentstroke}{rgb}{0.000000,0.000000,1.000000}%
\pgfsetstrokecolor{currentstroke}%
\pgfsetstrokeopacity{0.600000}%
\pgfsetdash{}{0pt}%
\pgfpathmoveto{\pgfqpoint{0.722838in}{1.289327in}}%
\pgfpathlineto{\pgfqpoint{0.819460in}{1.745464in}}%
\pgfpathlineto{\pgfqpoint{0.916083in}{2.216179in}}%
\pgfpathlineto{\pgfqpoint{1.012705in}{2.627159in}}%
\pgfpathlineto{\pgfqpoint{1.109328in}{2.913516in}}%
\pgfpathlineto{\pgfqpoint{1.205950in}{3.030044in}}%
\pgfpathlineto{\pgfqpoint{1.302572in}{2.958345in}}%
\pgfpathlineto{\pgfqpoint{1.399195in}{2.709737in}}%
\pgfpathlineto{\pgfqpoint{1.495817in}{2.323471in}}%
\pgfpathlineto{\pgfqpoint{1.592440in}{1.860530in}}%
\pgfpathlineto{\pgfqpoint{1.689062in}{1.394003in}}%
\pgfpathlineto{\pgfqpoint{1.785685in}{0.997543in}}%
\pgfpathlineto{\pgfqpoint{1.882307in}{0.733741in}}%
\pgfpathlineto{\pgfqpoint{1.978930in}{0.644248in}}%
\pgfpathlineto{\pgfqpoint{2.075552in}{0.743194in}}%
\pgfpathlineto{\pgfqpoint{2.172174in}{1.014954in}}%
\pgfpathlineto{\pgfqpoint{2.268797in}{1.416624in}}%
\pgfpathlineto{\pgfqpoint{2.365419in}{1.884792in}}%
\pgfpathlineto{\pgfqpoint{2.462042in}{2.345541in}}%
\pgfpathlineto{\pgfqpoint{2.558664in}{2.726132in}}%
\pgfpathlineto{\pgfqpoint{2.655287in}{2.966474in}}%
\pgfpathlineto{\pgfqpoint{2.751909in}{3.028628in}}%
\pgfpathlineto{\pgfqpoint{2.848532in}{2.902774in}}%
\pgfpathlineto{\pgfqpoint{2.945154in}{2.608790in}}%
\pgfpathlineto{\pgfqpoint{3.041777in}{2.193081in}}%
\pgfpathlineto{\pgfqpoint{3.138399in}{1.721285in}}%
\pgfpathlineto{\pgfqpoint{3.235021in}{1.267885in}}%
\pgfpathlineto{\pgfqpoint{3.331644in}{0.904462in}}%
\pgfpathlineto{\pgfqpoint{3.428266in}{0.688397in}}%
\pgfpathlineto{\pgfqpoint{3.524889in}{0.653798in}}%
\pgfpathlineto{\pgfqpoint{3.621511in}{0.806128in}}%
\pgfpathlineto{\pgfqpoint{3.718134in}{1.121337in}}%
\pgfusepath{stroke}%
\end{pgfscope}%
\begin{pgfscope}%
\pgfsetrectcap%
\pgfsetmiterjoin%
\pgfsetlinewidth{1.003750pt}%
\definecolor{currentstroke}{rgb}{0.000000,0.000000,1.000000}%
\pgfsetstrokecolor{currentstroke}%
\pgfsetstrokeopacity{0.600000}%
\pgfsetdash{}{0pt}%
\pgfpathmoveto{\pgfqpoint{3.867898in}{0.524958in}}%
\pgfpathlineto{\pgfqpoint{3.867898in}{3.149333in}}%
\pgfusepath{stroke}%
\end{pgfscope}%
\begin{pgfscope}%
\pgfsetrectcap%
\pgfsetmiterjoin%
\pgfsetlinewidth{1.003750pt}%
\definecolor{currentstroke}{rgb}{0.400000,0.400000,0.400000}%
\pgfsetstrokecolor{currentstroke}%
\pgfsetdash{}{0pt}%
\pgfpathmoveto{\pgfqpoint{0.573073in}{0.524958in}}%
\pgfpathlineto{\pgfqpoint{3.867898in}{0.524958in}}%
\pgfusepath{stroke}%
\end{pgfscope}%
\end{pgfpicture}%
\makeatother%
\endgroup%


        \caption{Exemplo de gráfico de três eixos para a corrente e a tensão em cada tempo}
        \label{fig:multiv:duplo}
    \end{figure}


\subsection{Gráficos de Eixos Separados}

    A opção mais genérica para mostrar os dois canais ao mesmo tempo é colocar cada um em seu próprio gŕafico com seus próprios eixos. Normalmente, isso é feito com duas imagens diferentes, mas para facilitar a comparação entre os gráficos, eles podem ser feitos em uma mesma figura. No \matplotlib, isso pode ser feito com a ideia de \textit{subplots}.

    A função do \pyplot para isso é \pyref{https://matplotlib.org/3.1.0/api/_as_gen/matplotlib.pyplot.subplots.html}{subplots}, que retorna a instância da figura, um objeto \pyref{https://matplotlib.org/3.1.0/api/_as_gen/matplotlib.figure.Figure.html}{Figure}, e as instâncias dos eixos, montados em uma matriz de eixos dada pelo número de linhas e de colunas, que no código \ref{code:multiv:paineis} é \pyline{nrows=2} e \pyline{ncols=1}. No exemplo, os eixos $x$ de cada gráfico são os mesmo, então faz sentido eles serem representados do mesmo jeito, que foi pra isso o argumento \pyline{sharex=True}.

    Além disso, os eixos foram separados em superior e inferior, sendo que o título foi colocado apenas no superior e o rótulo do eixo $x$, só no inferior.

    \begin{listing}[H]
        \caption{Montagem completa do gráfico de duas varíaveis com abscissa compartilhada}
        \label{code:multiv:paineis}

        \pyinclude[firstline=71, lastline=89]{recursos/multiv/multiv.py}
    \end{listing}

    \begin{figure}[H]
        \centering
        %% Creator: Matplotlib, PGF backend
%%
%% To include the figure in your LaTeX document, write
%%   \input{<filename>.pgf}
%%
%% Make sure the required packages are loaded in your preamble
%%   \usepackage{pgf}
%%
%% Figures using additional raster images can only be included by \input if
%% they are in the same directory as the main LaTeX file. For loading figures
%% from other directories you can use the `import` package
%%   \usepackage{import}
%% and then include the figures with
%%   \import{<path to file>}{<filename>.pgf}
%%
%% Matplotlib used the following preamble
%%   
%%       \usepackage[portuguese]{babel}
%%       \usepackage[T1]{fontenc}
%%       \usepackage[utf8]{inputenc}
%%   \usepackage{fontspec}
%%
\begingroup%
\makeatletter%
\begin{pgfpicture}%
\pgfpathrectangle{\pgfpointorigin}{\pgfqpoint{4.500000in}{3.500000in}}%
\pgfusepath{use as bounding box, clip}%
\begin{pgfscope}%
\pgfsetbuttcap%
\pgfsetmiterjoin%
\pgfsetlinewidth{0.000000pt}%
\definecolor{currentstroke}{rgb}{0.000000,0.000000,0.000000}%
\pgfsetstrokecolor{currentstroke}%
\pgfsetstrokeopacity{0.000000}%
\pgfsetdash{}{0pt}%
\pgfpathmoveto{\pgfqpoint{0.000000in}{0.000000in}}%
\pgfpathlineto{\pgfqpoint{4.500000in}{0.000000in}}%
\pgfpathlineto{\pgfqpoint{4.500000in}{3.500000in}}%
\pgfpathlineto{\pgfqpoint{0.000000in}{3.500000in}}%
\pgfpathclose%
\pgfusepath{}%
\end{pgfscope}%
\begin{pgfscope}%
\pgfsetbuttcap%
\pgfsetmiterjoin%
\pgfsetlinewidth{0.000000pt}%
\definecolor{currentstroke}{rgb}{0.000000,0.000000,0.000000}%
\pgfsetstrokecolor{currentstroke}%
\pgfsetstrokeopacity{0.000000}%
\pgfsetdash{}{0pt}%
\pgfpathmoveto{\pgfqpoint{0.632102in}{1.912146in}}%
\pgfpathlineto{\pgfqpoint{4.350000in}{1.912146in}}%
\pgfpathlineto{\pgfqpoint{4.350000in}{3.149333in}}%
\pgfpathlineto{\pgfqpoint{0.632102in}{3.149333in}}%
\pgfpathclose%
\pgfusepath{}%
\end{pgfscope}%
\begin{pgfscope}%
\pgfpathrectangle{\pgfqpoint{0.632102in}{1.912146in}}{\pgfqpoint{3.717898in}{1.237187in}}%
\pgfusepath{clip}%
\pgfsetbuttcap%
\pgfsetroundjoin%
\pgfsetlinewidth{0.803000pt}%
\definecolor{currentstroke}{rgb}{0.800000,0.800000,0.800000}%
\pgfsetstrokecolor{currentstroke}%
\pgfsetdash{{2.960000pt}{1.280000pt}}{0.000000pt}%
\pgfpathmoveto{\pgfqpoint{0.801097in}{1.912146in}}%
\pgfpathlineto{\pgfqpoint{0.801097in}{3.149333in}}%
\pgfusepath{stroke}%
\end{pgfscope}%
\begin{pgfscope}%
\pgfpathrectangle{\pgfqpoint{0.632102in}{1.912146in}}{\pgfqpoint{3.717898in}{1.237187in}}%
\pgfusepath{clip}%
\pgfsetbuttcap%
\pgfsetroundjoin%
\pgfsetlinewidth{0.803000pt}%
\definecolor{currentstroke}{rgb}{0.800000,0.800000,0.800000}%
\pgfsetstrokecolor{currentstroke}%
\pgfsetdash{{2.960000pt}{1.280000pt}}{0.000000pt}%
\pgfpathmoveto{\pgfqpoint{1.346243in}{1.912146in}}%
\pgfpathlineto{\pgfqpoint{1.346243in}{3.149333in}}%
\pgfusepath{stroke}%
\end{pgfscope}%
\begin{pgfscope}%
\pgfpathrectangle{\pgfqpoint{0.632102in}{1.912146in}}{\pgfqpoint{3.717898in}{1.237187in}}%
\pgfusepath{clip}%
\pgfsetbuttcap%
\pgfsetroundjoin%
\pgfsetlinewidth{0.803000pt}%
\definecolor{currentstroke}{rgb}{0.800000,0.800000,0.800000}%
\pgfsetstrokecolor{currentstroke}%
\pgfsetdash{{2.960000pt}{1.280000pt}}{0.000000pt}%
\pgfpathmoveto{\pgfqpoint{1.891390in}{1.912146in}}%
\pgfpathlineto{\pgfqpoint{1.891390in}{3.149333in}}%
\pgfusepath{stroke}%
\end{pgfscope}%
\begin{pgfscope}%
\pgfpathrectangle{\pgfqpoint{0.632102in}{1.912146in}}{\pgfqpoint{3.717898in}{1.237187in}}%
\pgfusepath{clip}%
\pgfsetbuttcap%
\pgfsetroundjoin%
\pgfsetlinewidth{0.803000pt}%
\definecolor{currentstroke}{rgb}{0.800000,0.800000,0.800000}%
\pgfsetstrokecolor{currentstroke}%
\pgfsetdash{{2.960000pt}{1.280000pt}}{0.000000pt}%
\pgfpathmoveto{\pgfqpoint{2.436536in}{1.912146in}}%
\pgfpathlineto{\pgfqpoint{2.436536in}{3.149333in}}%
\pgfusepath{stroke}%
\end{pgfscope}%
\begin{pgfscope}%
\pgfpathrectangle{\pgfqpoint{0.632102in}{1.912146in}}{\pgfqpoint{3.717898in}{1.237187in}}%
\pgfusepath{clip}%
\pgfsetbuttcap%
\pgfsetroundjoin%
\pgfsetlinewidth{0.803000pt}%
\definecolor{currentstroke}{rgb}{0.800000,0.800000,0.800000}%
\pgfsetstrokecolor{currentstroke}%
\pgfsetdash{{2.960000pt}{1.280000pt}}{0.000000pt}%
\pgfpathmoveto{\pgfqpoint{2.981683in}{1.912146in}}%
\pgfpathlineto{\pgfqpoint{2.981683in}{3.149333in}}%
\pgfusepath{stroke}%
\end{pgfscope}%
\begin{pgfscope}%
\pgfpathrectangle{\pgfqpoint{0.632102in}{1.912146in}}{\pgfqpoint{3.717898in}{1.237187in}}%
\pgfusepath{clip}%
\pgfsetbuttcap%
\pgfsetroundjoin%
\pgfsetlinewidth{0.803000pt}%
\definecolor{currentstroke}{rgb}{0.800000,0.800000,0.800000}%
\pgfsetstrokecolor{currentstroke}%
\pgfsetdash{{2.960000pt}{1.280000pt}}{0.000000pt}%
\pgfpathmoveto{\pgfqpoint{3.526829in}{1.912146in}}%
\pgfpathlineto{\pgfqpoint{3.526829in}{3.149333in}}%
\pgfusepath{stroke}%
\end{pgfscope}%
\begin{pgfscope}%
\pgfpathrectangle{\pgfqpoint{0.632102in}{1.912146in}}{\pgfqpoint{3.717898in}{1.237187in}}%
\pgfusepath{clip}%
\pgfsetbuttcap%
\pgfsetroundjoin%
\pgfsetlinewidth{0.803000pt}%
\definecolor{currentstroke}{rgb}{0.800000,0.800000,0.800000}%
\pgfsetstrokecolor{currentstroke}%
\pgfsetdash{{2.960000pt}{1.280000pt}}{0.000000pt}%
\pgfpathmoveto{\pgfqpoint{4.071975in}{1.912146in}}%
\pgfpathlineto{\pgfqpoint{4.071975in}{3.149333in}}%
\pgfusepath{stroke}%
\end{pgfscope}%
\begin{pgfscope}%
\pgfpathrectangle{\pgfqpoint{0.632102in}{1.912146in}}{\pgfqpoint{3.717898in}{1.237187in}}%
\pgfusepath{clip}%
\pgfsetbuttcap%
\pgfsetroundjoin%
\pgfsetlinewidth{0.803000pt}%
\definecolor{currentstroke}{rgb}{0.800000,0.800000,0.800000}%
\pgfsetstrokecolor{currentstroke}%
\pgfsetdash{{2.960000pt}{1.280000pt}}{0.000000pt}%
\pgfpathmoveto{\pgfqpoint{0.632102in}{2.041733in}}%
\pgfpathlineto{\pgfqpoint{4.350000in}{2.041733in}}%
\pgfusepath{stroke}%
\end{pgfscope}%
\begin{pgfscope}%
\definecolor{textcolor}{rgb}{0.150000,0.150000,0.150000}%
\pgfsetstrokecolor{textcolor}%
\pgfsetfillcolor{textcolor}%
\pgftext[x=0.403473in,y=2.001587in,left,base]{\color{textcolor}\rmfamily\fontsize{8.330000}{9.996000}\selectfont \(\displaystyle -2\)}%
\end{pgfscope}%
\begin{pgfscope}%
\pgfpathrectangle{\pgfqpoint{0.632102in}{1.912146in}}{\pgfqpoint{3.717898in}{1.237187in}}%
\pgfusepath{clip}%
\pgfsetbuttcap%
\pgfsetroundjoin%
\pgfsetlinewidth{0.803000pt}%
\definecolor{currentstroke}{rgb}{0.800000,0.800000,0.800000}%
\pgfsetstrokecolor{currentstroke}%
\pgfsetdash{{2.960000pt}{1.280000pt}}{0.000000pt}%
\pgfpathmoveto{\pgfqpoint{0.632102in}{2.530740in}}%
\pgfpathlineto{\pgfqpoint{4.350000in}{2.530740in}}%
\pgfusepath{stroke}%
\end{pgfscope}%
\begin{pgfscope}%
\definecolor{textcolor}{rgb}{0.150000,0.150000,0.150000}%
\pgfsetstrokecolor{textcolor}%
\pgfsetfillcolor{textcolor}%
\pgftext[x=0.495295in,y=2.490594in,left,base]{\color{textcolor}\rmfamily\fontsize{8.330000}{9.996000}\selectfont \(\displaystyle 0\)}%
\end{pgfscope}%
\begin{pgfscope}%
\pgfpathrectangle{\pgfqpoint{0.632102in}{1.912146in}}{\pgfqpoint{3.717898in}{1.237187in}}%
\pgfusepath{clip}%
\pgfsetbuttcap%
\pgfsetroundjoin%
\pgfsetlinewidth{0.803000pt}%
\definecolor{currentstroke}{rgb}{0.800000,0.800000,0.800000}%
\pgfsetstrokecolor{currentstroke}%
\pgfsetdash{{2.960000pt}{1.280000pt}}{0.000000pt}%
\pgfpathmoveto{\pgfqpoint{0.632102in}{3.019747in}}%
\pgfpathlineto{\pgfqpoint{4.350000in}{3.019747in}}%
\pgfusepath{stroke}%
\end{pgfscope}%
\begin{pgfscope}%
\definecolor{textcolor}{rgb}{0.150000,0.150000,0.150000}%
\pgfsetstrokecolor{textcolor}%
\pgfsetfillcolor{textcolor}%
\pgftext[x=0.495295in,y=2.979601in,left,base]{\color{textcolor}\rmfamily\fontsize{8.330000}{9.996000}\selectfont \(\displaystyle 2\)}%
\end{pgfscope}%
\begin{pgfscope}%
\definecolor{textcolor}{rgb}{0.000000,0.000000,0.000000}%
\pgfsetstrokecolor{textcolor}%
\pgfsetfillcolor{textcolor}%
\pgftext[x=0.347917in,y=2.530740in,,bottom,rotate=90.000000]{\color{textcolor}\rmfamily\fontsize{10.000000}{12.000000}\selectfont Tensão [\(\displaystyle V\)]}%
\end{pgfscope}%
\begin{pgfscope}%
\pgfpathrectangle{\pgfqpoint{0.632102in}{1.912146in}}{\pgfqpoint{3.717898in}{1.237187in}}%
\pgfusepath{clip}%
\pgfsetroundcap%
\pgfsetroundjoin%
\pgfsetlinewidth{1.405250pt}%
\definecolor{currentstroke}{rgb}{1.000000,0.000000,0.000000}%
\pgfsetstrokecolor{currentstroke}%
\pgfsetstrokeopacity{0.600000}%
\pgfsetdash{}{0pt}%
\pgfpathmoveto{\pgfqpoint{0.801097in}{3.017301in}}%
\pgfpathlineto{\pgfqpoint{0.910126in}{3.088207in}}%
\pgfpathlineto{\pgfqpoint{1.019155in}{3.071092in}}%
\pgfpathlineto{\pgfqpoint{1.128185in}{2.968401in}}%
\pgfpathlineto{\pgfqpoint{1.237214in}{2.797248in}}%
\pgfpathlineto{\pgfqpoint{1.346243in}{2.584530in}}%
\pgfpathlineto{\pgfqpoint{1.455273in}{2.362032in}}%
\pgfpathlineto{\pgfqpoint{1.564302in}{2.166429in}}%
\pgfpathlineto{\pgfqpoint{1.673331in}{2.027063in}}%
\pgfpathlineto{\pgfqpoint{1.782360in}{1.968382in}}%
\pgfpathlineto{\pgfqpoint{1.891390in}{2.000167in}}%
\pgfpathlineto{\pgfqpoint{2.000419in}{2.112639in}}%
\pgfpathlineto{\pgfqpoint{2.109448in}{2.293571in}}%
\pgfpathlineto{\pgfqpoint{2.218478in}{2.511179in}}%
\pgfpathlineto{\pgfqpoint{2.327507in}{2.731232in}}%
\pgfpathlineto{\pgfqpoint{2.436536in}{2.919500in}}%
\pgfpathlineto{\pgfqpoint{2.545565in}{3.046642in}}%
\pgfpathlineto{\pgfqpoint{2.654595in}{3.093098in}}%
\pgfpathlineto{\pgfqpoint{2.763624in}{3.049087in}}%
\pgfpathlineto{\pgfqpoint{2.872653in}{2.924390in}}%
\pgfpathlineto{\pgfqpoint{2.981683in}{2.738568in}}%
\pgfpathlineto{\pgfqpoint{3.090712in}{2.518514in}}%
\pgfpathlineto{\pgfqpoint{3.199741in}{2.300906in}}%
\pgfpathlineto{\pgfqpoint{3.308770in}{2.117529in}}%
\pgfpathlineto{\pgfqpoint{3.417800in}{2.002612in}}%
\pgfpathlineto{\pgfqpoint{3.526829in}{1.968382in}}%
\pgfpathlineto{\pgfqpoint{3.635858in}{2.024617in}}%
\pgfpathlineto{\pgfqpoint{3.744887in}{2.159094in}}%
\pgfpathlineto{\pgfqpoint{3.853917in}{2.354697in}}%
\pgfpathlineto{\pgfqpoint{3.962946in}{2.577195in}}%
\pgfpathlineto{\pgfqpoint{4.071975in}{2.789913in}}%
\pgfpathlineto{\pgfqpoint{4.181005in}{2.963511in}}%
\pgfusepath{stroke}%
\end{pgfscope}%
\begin{pgfscope}%
\pgfsetrectcap%
\pgfsetmiterjoin%
\pgfsetlinewidth{1.003750pt}%
\definecolor{currentstroke}{rgb}{0.400000,0.400000,0.400000}%
\pgfsetstrokecolor{currentstroke}%
\pgfsetdash{}{0pt}%
\pgfpathmoveto{\pgfqpoint{0.632102in}{1.912146in}}%
\pgfpathlineto{\pgfqpoint{0.632102in}{3.149333in}}%
\pgfusepath{stroke}%
\end{pgfscope}%
\begin{pgfscope}%
\pgfsetrectcap%
\pgfsetmiterjoin%
\pgfsetlinewidth{1.003750pt}%
\definecolor{currentstroke}{rgb}{0.400000,0.400000,0.400000}%
\pgfsetstrokecolor{currentstroke}%
\pgfsetdash{}{0pt}%
\pgfpathmoveto{\pgfqpoint{0.632102in}{1.912146in}}%
\pgfpathlineto{\pgfqpoint{4.350000in}{1.912146in}}%
\pgfusepath{stroke}%
\end{pgfscope}%
\begin{pgfscope}%
\definecolor{textcolor}{rgb}{0.000000,0.000000,0.000000}%
\pgfsetstrokecolor{textcolor}%
\pgfsetfillcolor{textcolor}%
\pgftext[x=2.491051in,y=3.232667in,,base]{\color{textcolor}\rmfamily\fontsize{12.000000}{14.400000}\selectfont Relação de Corrente e Tensão em um Capacitor}%
\end{pgfscope}%
\begin{pgfscope}%
\pgfsetbuttcap%
\pgfsetmiterjoin%
\pgfsetlinewidth{0.000000pt}%
\definecolor{currentstroke}{rgb}{0.000000,0.000000,0.000000}%
\pgfsetstrokecolor{currentstroke}%
\pgfsetstrokeopacity{0.000000}%
\pgfsetdash{}{0pt}%
\pgfpathmoveto{\pgfqpoint{0.632102in}{0.524958in}}%
\pgfpathlineto{\pgfqpoint{4.350000in}{0.524958in}}%
\pgfpathlineto{\pgfqpoint{4.350000in}{1.762146in}}%
\pgfpathlineto{\pgfqpoint{0.632102in}{1.762146in}}%
\pgfpathclose%
\pgfusepath{}%
\end{pgfscope}%
\begin{pgfscope}%
\pgfpathrectangle{\pgfqpoint{0.632102in}{0.524958in}}{\pgfqpoint{3.717898in}{1.237187in}}%
\pgfusepath{clip}%
\pgfsetbuttcap%
\pgfsetroundjoin%
\pgfsetlinewidth{0.803000pt}%
\definecolor{currentstroke}{rgb}{0.800000,0.800000,0.800000}%
\pgfsetstrokecolor{currentstroke}%
\pgfsetdash{{2.960000pt}{1.280000pt}}{0.000000pt}%
\pgfpathmoveto{\pgfqpoint{0.801097in}{0.524958in}}%
\pgfpathlineto{\pgfqpoint{0.801097in}{1.762146in}}%
\pgfusepath{stroke}%
\end{pgfscope}%
\begin{pgfscope}%
\definecolor{textcolor}{rgb}{0.150000,0.150000,0.150000}%
\pgfsetstrokecolor{textcolor}%
\pgfsetfillcolor{textcolor}%
\pgftext[x=0.801097in,y=0.447181in,,top]{\color{textcolor}\rmfamily\fontsize{8.330000}{9.996000}\selectfont \(\displaystyle 0\)}%
\end{pgfscope}%
\begin{pgfscope}%
\pgfpathrectangle{\pgfqpoint{0.632102in}{0.524958in}}{\pgfqpoint{3.717898in}{1.237187in}}%
\pgfusepath{clip}%
\pgfsetbuttcap%
\pgfsetroundjoin%
\pgfsetlinewidth{0.803000pt}%
\definecolor{currentstroke}{rgb}{0.800000,0.800000,0.800000}%
\pgfsetstrokecolor{currentstroke}%
\pgfsetdash{{2.960000pt}{1.280000pt}}{0.000000pt}%
\pgfpathmoveto{\pgfqpoint{1.346243in}{0.524958in}}%
\pgfpathlineto{\pgfqpoint{1.346243in}{1.762146in}}%
\pgfusepath{stroke}%
\end{pgfscope}%
\begin{pgfscope}%
\definecolor{textcolor}{rgb}{0.150000,0.150000,0.150000}%
\pgfsetstrokecolor{textcolor}%
\pgfsetfillcolor{textcolor}%
\pgftext[x=1.346243in,y=0.447181in,,top]{\color{textcolor}\rmfamily\fontsize{8.330000}{9.996000}\selectfont \(\displaystyle 2\)}%
\end{pgfscope}%
\begin{pgfscope}%
\pgfpathrectangle{\pgfqpoint{0.632102in}{0.524958in}}{\pgfqpoint{3.717898in}{1.237187in}}%
\pgfusepath{clip}%
\pgfsetbuttcap%
\pgfsetroundjoin%
\pgfsetlinewidth{0.803000pt}%
\definecolor{currentstroke}{rgb}{0.800000,0.800000,0.800000}%
\pgfsetstrokecolor{currentstroke}%
\pgfsetdash{{2.960000pt}{1.280000pt}}{0.000000pt}%
\pgfpathmoveto{\pgfqpoint{1.891390in}{0.524958in}}%
\pgfpathlineto{\pgfqpoint{1.891390in}{1.762146in}}%
\pgfusepath{stroke}%
\end{pgfscope}%
\begin{pgfscope}%
\definecolor{textcolor}{rgb}{0.150000,0.150000,0.150000}%
\pgfsetstrokecolor{textcolor}%
\pgfsetfillcolor{textcolor}%
\pgftext[x=1.891390in,y=0.447181in,,top]{\color{textcolor}\rmfamily\fontsize{8.330000}{9.996000}\selectfont \(\displaystyle 4\)}%
\end{pgfscope}%
\begin{pgfscope}%
\pgfpathrectangle{\pgfqpoint{0.632102in}{0.524958in}}{\pgfqpoint{3.717898in}{1.237187in}}%
\pgfusepath{clip}%
\pgfsetbuttcap%
\pgfsetroundjoin%
\pgfsetlinewidth{0.803000pt}%
\definecolor{currentstroke}{rgb}{0.800000,0.800000,0.800000}%
\pgfsetstrokecolor{currentstroke}%
\pgfsetdash{{2.960000pt}{1.280000pt}}{0.000000pt}%
\pgfpathmoveto{\pgfqpoint{2.436536in}{0.524958in}}%
\pgfpathlineto{\pgfqpoint{2.436536in}{1.762146in}}%
\pgfusepath{stroke}%
\end{pgfscope}%
\begin{pgfscope}%
\definecolor{textcolor}{rgb}{0.150000,0.150000,0.150000}%
\pgfsetstrokecolor{textcolor}%
\pgfsetfillcolor{textcolor}%
\pgftext[x=2.436536in,y=0.447181in,,top]{\color{textcolor}\rmfamily\fontsize{8.330000}{9.996000}\selectfont \(\displaystyle 6\)}%
\end{pgfscope}%
\begin{pgfscope}%
\pgfpathrectangle{\pgfqpoint{0.632102in}{0.524958in}}{\pgfqpoint{3.717898in}{1.237187in}}%
\pgfusepath{clip}%
\pgfsetbuttcap%
\pgfsetroundjoin%
\pgfsetlinewidth{0.803000pt}%
\definecolor{currentstroke}{rgb}{0.800000,0.800000,0.800000}%
\pgfsetstrokecolor{currentstroke}%
\pgfsetdash{{2.960000pt}{1.280000pt}}{0.000000pt}%
\pgfpathmoveto{\pgfqpoint{2.981683in}{0.524958in}}%
\pgfpathlineto{\pgfqpoint{2.981683in}{1.762146in}}%
\pgfusepath{stroke}%
\end{pgfscope}%
\begin{pgfscope}%
\definecolor{textcolor}{rgb}{0.150000,0.150000,0.150000}%
\pgfsetstrokecolor{textcolor}%
\pgfsetfillcolor{textcolor}%
\pgftext[x=2.981683in,y=0.447181in,,top]{\color{textcolor}\rmfamily\fontsize{8.330000}{9.996000}\selectfont \(\displaystyle 8\)}%
\end{pgfscope}%
\begin{pgfscope}%
\pgfpathrectangle{\pgfqpoint{0.632102in}{0.524958in}}{\pgfqpoint{3.717898in}{1.237187in}}%
\pgfusepath{clip}%
\pgfsetbuttcap%
\pgfsetroundjoin%
\pgfsetlinewidth{0.803000pt}%
\definecolor{currentstroke}{rgb}{0.800000,0.800000,0.800000}%
\pgfsetstrokecolor{currentstroke}%
\pgfsetdash{{2.960000pt}{1.280000pt}}{0.000000pt}%
\pgfpathmoveto{\pgfqpoint{3.526829in}{0.524958in}}%
\pgfpathlineto{\pgfqpoint{3.526829in}{1.762146in}}%
\pgfusepath{stroke}%
\end{pgfscope}%
\begin{pgfscope}%
\definecolor{textcolor}{rgb}{0.150000,0.150000,0.150000}%
\pgfsetstrokecolor{textcolor}%
\pgfsetfillcolor{textcolor}%
\pgftext[x=3.526829in,y=0.447181in,,top]{\color{textcolor}\rmfamily\fontsize{8.330000}{9.996000}\selectfont \(\displaystyle 10\)}%
\end{pgfscope}%
\begin{pgfscope}%
\pgfpathrectangle{\pgfqpoint{0.632102in}{0.524958in}}{\pgfqpoint{3.717898in}{1.237187in}}%
\pgfusepath{clip}%
\pgfsetbuttcap%
\pgfsetroundjoin%
\pgfsetlinewidth{0.803000pt}%
\definecolor{currentstroke}{rgb}{0.800000,0.800000,0.800000}%
\pgfsetstrokecolor{currentstroke}%
\pgfsetdash{{2.960000pt}{1.280000pt}}{0.000000pt}%
\pgfpathmoveto{\pgfqpoint{4.071975in}{0.524958in}}%
\pgfpathlineto{\pgfqpoint{4.071975in}{1.762146in}}%
\pgfusepath{stroke}%
\end{pgfscope}%
\begin{pgfscope}%
\definecolor{textcolor}{rgb}{0.150000,0.150000,0.150000}%
\pgfsetstrokecolor{textcolor}%
\pgfsetfillcolor{textcolor}%
\pgftext[x=4.071975in,y=0.447181in,,top]{\color{textcolor}\rmfamily\fontsize{8.330000}{9.996000}\selectfont \(\displaystyle 12\)}%
\end{pgfscope}%
\begin{pgfscope}%
\definecolor{textcolor}{rgb}{0.000000,0.000000,0.000000}%
\pgfsetstrokecolor{textcolor}%
\pgfsetfillcolor{textcolor}%
\pgftext[x=2.491051in,y=0.288889in,,top]{\color{textcolor}\rmfamily\fontsize{10.000000}{12.000000}\selectfont Tempo [\(\displaystyle ms\)]}%
\end{pgfscope}%
\begin{pgfscope}%
\pgfpathrectangle{\pgfqpoint{0.632102in}{0.524958in}}{\pgfqpoint{3.717898in}{1.237187in}}%
\pgfusepath{clip}%
\pgfsetbuttcap%
\pgfsetroundjoin%
\pgfsetlinewidth{0.803000pt}%
\definecolor{currentstroke}{rgb}{0.800000,0.800000,0.800000}%
\pgfsetstrokecolor{currentstroke}%
\pgfsetdash{{2.960000pt}{1.280000pt}}{0.000000pt}%
\pgfpathmoveto{\pgfqpoint{0.632102in}{0.624619in}}%
\pgfpathlineto{\pgfqpoint{4.350000in}{0.624619in}}%
\pgfusepath{stroke}%
\end{pgfscope}%
\begin{pgfscope}%
\definecolor{textcolor}{rgb}{0.150000,0.150000,0.150000}%
\pgfsetstrokecolor{textcolor}%
\pgfsetfillcolor{textcolor}%
\pgftext[x=0.344444in,y=0.584473in,left,base]{\color{textcolor}\rmfamily\fontsize{8.330000}{9.996000}\selectfont \(\displaystyle -40\)}%
\end{pgfscope}%
\begin{pgfscope}%
\pgfpathrectangle{\pgfqpoint{0.632102in}{0.524958in}}{\pgfqpoint{3.717898in}{1.237187in}}%
\pgfusepath{clip}%
\pgfsetbuttcap%
\pgfsetroundjoin%
\pgfsetlinewidth{0.803000pt}%
\definecolor{currentstroke}{rgb}{0.800000,0.800000,0.800000}%
\pgfsetstrokecolor{currentstroke}%
\pgfsetdash{{2.960000pt}{1.280000pt}}{0.000000pt}%
\pgfpathmoveto{\pgfqpoint{0.632102in}{0.884242in}}%
\pgfpathlineto{\pgfqpoint{4.350000in}{0.884242in}}%
\pgfusepath{stroke}%
\end{pgfscope}%
\begin{pgfscope}%
\definecolor{textcolor}{rgb}{0.150000,0.150000,0.150000}%
\pgfsetstrokecolor{textcolor}%
\pgfsetfillcolor{textcolor}%
\pgftext[x=0.344444in,y=0.844096in,left,base]{\color{textcolor}\rmfamily\fontsize{8.330000}{9.996000}\selectfont \(\displaystyle -20\)}%
\end{pgfscope}%
\begin{pgfscope}%
\pgfpathrectangle{\pgfqpoint{0.632102in}{0.524958in}}{\pgfqpoint{3.717898in}{1.237187in}}%
\pgfusepath{clip}%
\pgfsetbuttcap%
\pgfsetroundjoin%
\pgfsetlinewidth{0.803000pt}%
\definecolor{currentstroke}{rgb}{0.800000,0.800000,0.800000}%
\pgfsetstrokecolor{currentstroke}%
\pgfsetdash{{2.960000pt}{1.280000pt}}{0.000000pt}%
\pgfpathmoveto{\pgfqpoint{0.632102in}{1.143865in}}%
\pgfpathlineto{\pgfqpoint{4.350000in}{1.143865in}}%
\pgfusepath{stroke}%
\end{pgfscope}%
\begin{pgfscope}%
\definecolor{textcolor}{rgb}{0.150000,0.150000,0.150000}%
\pgfsetstrokecolor{textcolor}%
\pgfsetfillcolor{textcolor}%
\pgftext[x=0.495295in,y=1.103719in,left,base]{\color{textcolor}\rmfamily\fontsize{8.330000}{9.996000}\selectfont \(\displaystyle 0\)}%
\end{pgfscope}%
\begin{pgfscope}%
\pgfpathrectangle{\pgfqpoint{0.632102in}{0.524958in}}{\pgfqpoint{3.717898in}{1.237187in}}%
\pgfusepath{clip}%
\pgfsetbuttcap%
\pgfsetroundjoin%
\pgfsetlinewidth{0.803000pt}%
\definecolor{currentstroke}{rgb}{0.800000,0.800000,0.800000}%
\pgfsetstrokecolor{currentstroke}%
\pgfsetdash{{2.960000pt}{1.280000pt}}{0.000000pt}%
\pgfpathmoveto{\pgfqpoint{0.632102in}{1.403488in}}%
\pgfpathlineto{\pgfqpoint{4.350000in}{1.403488in}}%
\pgfusepath{stroke}%
\end{pgfscope}%
\begin{pgfscope}%
\definecolor{textcolor}{rgb}{0.150000,0.150000,0.150000}%
\pgfsetstrokecolor{textcolor}%
\pgfsetfillcolor{textcolor}%
\pgftext[x=0.436267in,y=1.363342in,left,base]{\color{textcolor}\rmfamily\fontsize{8.330000}{9.996000}\selectfont \(\displaystyle 20\)}%
\end{pgfscope}%
\begin{pgfscope}%
\pgfpathrectangle{\pgfqpoint{0.632102in}{0.524958in}}{\pgfqpoint{3.717898in}{1.237187in}}%
\pgfusepath{clip}%
\pgfsetbuttcap%
\pgfsetroundjoin%
\pgfsetlinewidth{0.803000pt}%
\definecolor{currentstroke}{rgb}{0.800000,0.800000,0.800000}%
\pgfsetstrokecolor{currentstroke}%
\pgfsetdash{{2.960000pt}{1.280000pt}}{0.000000pt}%
\pgfpathmoveto{\pgfqpoint{0.632102in}{1.663111in}}%
\pgfpathlineto{\pgfqpoint{4.350000in}{1.663111in}}%
\pgfusepath{stroke}%
\end{pgfscope}%
\begin{pgfscope}%
\definecolor{textcolor}{rgb}{0.150000,0.150000,0.150000}%
\pgfsetstrokecolor{textcolor}%
\pgfsetfillcolor{textcolor}%
\pgftext[x=0.436267in,y=1.622965in,left,base]{\color{textcolor}\rmfamily\fontsize{8.330000}{9.996000}\selectfont \(\displaystyle 40\)}%
\end{pgfscope}%
\begin{pgfscope}%
\definecolor{textcolor}{rgb}{0.000000,0.000000,0.000000}%
\pgfsetstrokecolor{textcolor}%
\pgfsetfillcolor{textcolor}%
\pgftext[x=0.288889in,y=1.143552in,,bottom,rotate=90.000000]{\color{textcolor}\rmfamily\fontsize{10.000000}{12.000000}\selectfont Corrente [\(\displaystyle mA\)]}%
\end{pgfscope}%
\begin{pgfscope}%
\pgfpathrectangle{\pgfqpoint{0.632102in}{0.524958in}}{\pgfqpoint{3.717898in}{1.237187in}}%
\pgfusepath{clip}%
\pgfsetroundcap%
\pgfsetroundjoin%
\pgfsetlinewidth{1.405250pt}%
\definecolor{currentstroke}{rgb}{0.000000,0.000000,1.000000}%
\pgfsetstrokecolor{currentstroke}%
\pgfsetstrokeopacity{0.600000}%
\pgfsetdash{}{0pt}%
\pgfpathmoveto{\pgfqpoint{0.801097in}{0.885299in}}%
\pgfpathlineto{\pgfqpoint{0.910126in}{1.100331in}}%
\pgfpathlineto{\pgfqpoint{1.019155in}{1.322236in}}%
\pgfpathlineto{\pgfqpoint{1.128185in}{1.515982in}}%
\pgfpathlineto{\pgfqpoint{1.237214in}{1.650976in}}%
\pgfpathlineto{\pgfqpoint{1.346243in}{1.705910in}}%
\pgfpathlineto{\pgfqpoint{1.455273in}{1.672110in}}%
\pgfpathlineto{\pgfqpoint{1.564302in}{1.554911in}}%
\pgfpathlineto{\pgfqpoint{1.673331in}{1.372816in}}%
\pgfpathlineto{\pgfqpoint{1.782360in}{1.154576in}}%
\pgfpathlineto{\pgfqpoint{1.891390in}{0.934645in}}%
\pgfpathlineto{\pgfqpoint{2.000419in}{0.747745in}}%
\pgfpathlineto{\pgfqpoint{2.109448in}{0.623383in}}%
\pgfpathlineto{\pgfqpoint{2.218478in}{0.581194in}}%
\pgfpathlineto{\pgfqpoint{2.327507in}{0.627839in}}%
\pgfpathlineto{\pgfqpoint{2.436536in}{0.755953in}}%
\pgfpathlineto{\pgfqpoint{2.545565in}{0.945309in}}%
\pgfpathlineto{\pgfqpoint{2.654595in}{1.166013in}}%
\pgfpathlineto{\pgfqpoint{2.763624in}{1.383221in}}%
\pgfpathlineto{\pgfqpoint{2.872653in}{1.562640in}}%
\pgfpathlineto{\pgfqpoint{2.981683in}{1.675942in}}%
\pgfpathlineto{\pgfqpoint{3.090712in}{1.705243in}}%
\pgfpathlineto{\pgfqpoint{3.199741in}{1.645913in}}%
\pgfpathlineto{\pgfqpoint{3.308770in}{1.507322in}}%
\pgfpathlineto{\pgfqpoint{3.417800in}{1.311348in}}%
\pgfpathlineto{\pgfqpoint{3.526829in}{1.088933in}}%
\pgfpathlineto{\pgfqpoint{3.635858in}{0.875190in}}%
\pgfpathlineto{\pgfqpoint{3.744887in}{0.703865in}}%
\pgfpathlineto{\pgfqpoint{3.853917in}{0.602007in}}%
\pgfpathlineto{\pgfqpoint{3.962946in}{0.585696in}}%
\pgfpathlineto{\pgfqpoint{4.071975in}{0.657508in}}%
\pgfpathlineto{\pgfqpoint{4.181005in}{0.806104in}}%
\pgfusepath{stroke}%
\end{pgfscope}%
\begin{pgfscope}%
\pgfsetrectcap%
\pgfsetmiterjoin%
\pgfsetlinewidth{1.003750pt}%
\definecolor{currentstroke}{rgb}{0.400000,0.400000,0.400000}%
\pgfsetstrokecolor{currentstroke}%
\pgfsetdash{}{0pt}%
\pgfpathmoveto{\pgfqpoint{0.632102in}{0.524958in}}%
\pgfpathlineto{\pgfqpoint{0.632102in}{1.762146in}}%
\pgfusepath{stroke}%
\end{pgfscope}%
\begin{pgfscope}%
\pgfsetrectcap%
\pgfsetmiterjoin%
\pgfsetlinewidth{1.003750pt}%
\definecolor{currentstroke}{rgb}{0.400000,0.400000,0.400000}%
\pgfsetstrokecolor{currentstroke}%
\pgfsetdash{}{0pt}%
\pgfpathmoveto{\pgfqpoint{0.632102in}{0.524958in}}%
\pgfpathlineto{\pgfqpoint{4.350000in}{0.524958in}}%
\pgfusepath{stroke}%
\end{pgfscope}%
\end{pgfpicture}%
\makeatother%
\endgroup%


        \caption{Exemplo de imagem com dois gráficos}
        \label{fig:multiv:paineis}
    \end{figure}
