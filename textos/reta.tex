\begin{table}[H]
    \centering
    \begin{tabular}{rr}
\toprule
  Tensão &   Corrente \\
\midrule
 -3.07 V &  -34.38 mA \\
 -2.70 V &  -27.96 mA \\
 -1.69 V &  -15.83 mA \\
 -1.47 V &  -10.79 mA \\
 -0.62 V &   -7.94 mA \\
 -0.04 V &   -0.05 mA \\
  0.72 V &    7.43 mA \\
  1.25 V &   13.37 mA \\
  2.35 V &   21.56 mA \\
  2.48 V &   31.34 mA \\
  3.38 V &   33.32 mA \\
\bottomrule
\end{tabular}

    \caption{Dados de corrente para cada tensão, gerados por computador}
    \label{tab:reta:dados}
\end{table}

Nesta seção, será tomado como exemplo a relação de corrente e tensão em um resistor, dado de forma teórica pela relação (\ref{eq:reta:corrente}). Por mais que os dados usados aqui sejam os da tabela \ref{tab:reta:dados}, essa parte de apresentação de dados é importante para todos os tipos de análise, em especial, para dados coletados manualmente, como é o caso da maioria dos experimentos da disciplina de \texttt{F 329}.

\begin{equacao} \label{eq:reta:corrente}
    I = \frac{1}{R} ~ V
\end{equacao}


\subsection{Dados Pontuais}

    \begin{listing}[H]
        \caption{Gerando um gráfico de dispersão}
        \label{code:reta:scatter}

        \pyinclude[firstline=11, lastline=15]{recursos/reta/reta.py}
    \end{listing}

    Para apresentar os dados coletados, a melhor opção é a função \pyref{https://matplotlib.org/3.1.0/api/_as_gen/matplotlib.pyplot.scatter.html}{scatter} do \pyplot, que recebe uma lista de valores de $x$ e outra lista de $y$ como argumentos e desenha cada ponto $(x, y)$.

    \begin{figure}[htbp]
        \centering
        %% Creator: Matplotlib, PGF backend
%%
%% To include the figure in your LaTeX document, write
%%   \input{<filename>.pgf}
%%
%% Make sure the required packages are loaded in your preamble
%%   \usepackage{pgf}
%%
%% Figures using additional raster images can only be included by \input if
%% they are in the same directory as the main LaTeX file. For loading figures
%% from other directories you can use the `import` package
%%   \usepackage{import}
%% and then include the figures with
%%   \import{<path to file>}{<filename>.pgf}
%%
%% Matplotlib used the following preamble
%%   
%%       \usepackage[portuguese]{babel}
%%       \usepackage[T1]{fontenc}
%%       \usepackage[utf8]{inputenc}
%%   \usepackage{fontspec}
%%
\begingroup%
\makeatletter%
\begin{pgfpicture}%
\pgfpathrectangle{\pgfpointorigin}{\pgfqpoint{4.500000in}{3.500000in}}%
\pgfusepath{use as bounding box, clip}%
\begin{pgfscope}%
\pgfsetbuttcap%
\pgfsetmiterjoin%
\pgfsetlinewidth{0.000000pt}%
\definecolor{currentstroke}{rgb}{0.000000,0.000000,0.000000}%
\pgfsetstrokecolor{currentstroke}%
\pgfsetstrokeopacity{0.000000}%
\pgfsetdash{}{0pt}%
\pgfpathmoveto{\pgfqpoint{0.000000in}{0.000000in}}%
\pgfpathlineto{\pgfqpoint{4.500000in}{0.000000in}}%
\pgfpathlineto{\pgfqpoint{4.500000in}{3.500000in}}%
\pgfpathlineto{\pgfqpoint{0.000000in}{3.500000in}}%
\pgfpathclose%
\pgfusepath{}%
\end{pgfscope}%
\begin{pgfscope}%
\pgfsetbuttcap%
\pgfsetmiterjoin%
\pgfsetlinewidth{0.000000pt}%
\definecolor{currentstroke}{rgb}{0.000000,0.000000,0.000000}%
\pgfsetstrokecolor{currentstroke}%
\pgfsetstrokeopacity{0.000000}%
\pgfsetdash{}{0pt}%
\pgfpathmoveto{\pgfqpoint{0.437657in}{0.330514in}}%
\pgfpathlineto{\pgfqpoint{4.350000in}{0.330514in}}%
\pgfpathlineto{\pgfqpoint{4.350000in}{3.350000in}}%
\pgfpathlineto{\pgfqpoint{0.437657in}{3.350000in}}%
\pgfpathclose%
\pgfusepath{}%
\end{pgfscope}%
\begin{pgfscope}%
\pgfpathrectangle{\pgfqpoint{0.437657in}{0.330514in}}{\pgfqpoint{3.912343in}{3.019486in}}%
\pgfusepath{clip}%
\pgfsetbuttcap%
\pgfsetroundjoin%
\pgfsetlinewidth{0.803000pt}%
\definecolor{currentstroke}{rgb}{0.800000,0.800000,0.800000}%
\pgfsetstrokecolor{currentstroke}%
\pgfsetdash{{2.960000pt}{1.280000pt}}{0.000000pt}%
\pgfpathmoveto{\pgfqpoint{0.658411in}{0.330514in}}%
\pgfpathlineto{\pgfqpoint{0.658411in}{3.350000in}}%
\pgfusepath{stroke}%
\end{pgfscope}%
\begin{pgfscope}%
\definecolor{textcolor}{rgb}{0.150000,0.150000,0.150000}%
\pgfsetstrokecolor{textcolor}%
\pgfsetfillcolor{textcolor}%
\pgftext[x=0.658411in,y=0.252737in,,top]{\color{textcolor}\rmfamily\fontsize{8.330000}{9.996000}\selectfont \(\displaystyle -3\)}%
\end{pgfscope}%
\begin{pgfscope}%
\pgfpathrectangle{\pgfqpoint{0.437657in}{0.330514in}}{\pgfqpoint{3.912343in}{3.019486in}}%
\pgfusepath{clip}%
\pgfsetbuttcap%
\pgfsetroundjoin%
\pgfsetlinewidth{0.803000pt}%
\definecolor{currentstroke}{rgb}{0.800000,0.800000,0.800000}%
\pgfsetstrokecolor{currentstroke}%
\pgfsetdash{{2.960000pt}{1.280000pt}}{0.000000pt}%
\pgfpathmoveto{\pgfqpoint{1.208464in}{0.330514in}}%
\pgfpathlineto{\pgfqpoint{1.208464in}{3.350000in}}%
\pgfusepath{stroke}%
\end{pgfscope}%
\begin{pgfscope}%
\definecolor{textcolor}{rgb}{0.150000,0.150000,0.150000}%
\pgfsetstrokecolor{textcolor}%
\pgfsetfillcolor{textcolor}%
\pgftext[x=1.208464in,y=0.252737in,,top]{\color{textcolor}\rmfamily\fontsize{8.330000}{9.996000}\selectfont \(\displaystyle -2\)}%
\end{pgfscope}%
\begin{pgfscope}%
\pgfpathrectangle{\pgfqpoint{0.437657in}{0.330514in}}{\pgfqpoint{3.912343in}{3.019486in}}%
\pgfusepath{clip}%
\pgfsetbuttcap%
\pgfsetroundjoin%
\pgfsetlinewidth{0.803000pt}%
\definecolor{currentstroke}{rgb}{0.800000,0.800000,0.800000}%
\pgfsetstrokecolor{currentstroke}%
\pgfsetdash{{2.960000pt}{1.280000pt}}{0.000000pt}%
\pgfpathmoveto{\pgfqpoint{1.758517in}{0.330514in}}%
\pgfpathlineto{\pgfqpoint{1.758517in}{3.350000in}}%
\pgfusepath{stroke}%
\end{pgfscope}%
\begin{pgfscope}%
\definecolor{textcolor}{rgb}{0.150000,0.150000,0.150000}%
\pgfsetstrokecolor{textcolor}%
\pgfsetfillcolor{textcolor}%
\pgftext[x=1.758517in,y=0.252737in,,top]{\color{textcolor}\rmfamily\fontsize{8.330000}{9.996000}\selectfont \(\displaystyle -1\)}%
\end{pgfscope}%
\begin{pgfscope}%
\pgfpathrectangle{\pgfqpoint{0.437657in}{0.330514in}}{\pgfqpoint{3.912343in}{3.019486in}}%
\pgfusepath{clip}%
\pgfsetbuttcap%
\pgfsetroundjoin%
\pgfsetlinewidth{0.803000pt}%
\definecolor{currentstroke}{rgb}{0.800000,0.800000,0.800000}%
\pgfsetstrokecolor{currentstroke}%
\pgfsetdash{{2.960000pt}{1.280000pt}}{0.000000pt}%
\pgfpathmoveto{\pgfqpoint{2.308570in}{0.330514in}}%
\pgfpathlineto{\pgfqpoint{2.308570in}{3.350000in}}%
\pgfusepath{stroke}%
\end{pgfscope}%
\begin{pgfscope}%
\definecolor{textcolor}{rgb}{0.150000,0.150000,0.150000}%
\pgfsetstrokecolor{textcolor}%
\pgfsetfillcolor{textcolor}%
\pgftext[x=2.308570in,y=0.252737in,,top]{\color{textcolor}\rmfamily\fontsize{8.330000}{9.996000}\selectfont \(\displaystyle 0\)}%
\end{pgfscope}%
\begin{pgfscope}%
\pgfpathrectangle{\pgfqpoint{0.437657in}{0.330514in}}{\pgfqpoint{3.912343in}{3.019486in}}%
\pgfusepath{clip}%
\pgfsetbuttcap%
\pgfsetroundjoin%
\pgfsetlinewidth{0.803000pt}%
\definecolor{currentstroke}{rgb}{0.800000,0.800000,0.800000}%
\pgfsetstrokecolor{currentstroke}%
\pgfsetdash{{2.960000pt}{1.280000pt}}{0.000000pt}%
\pgfpathmoveto{\pgfqpoint{2.858624in}{0.330514in}}%
\pgfpathlineto{\pgfqpoint{2.858624in}{3.350000in}}%
\pgfusepath{stroke}%
\end{pgfscope}%
\begin{pgfscope}%
\definecolor{textcolor}{rgb}{0.150000,0.150000,0.150000}%
\pgfsetstrokecolor{textcolor}%
\pgfsetfillcolor{textcolor}%
\pgftext[x=2.858624in,y=0.252737in,,top]{\color{textcolor}\rmfamily\fontsize{8.330000}{9.996000}\selectfont \(\displaystyle 1\)}%
\end{pgfscope}%
\begin{pgfscope}%
\pgfpathrectangle{\pgfqpoint{0.437657in}{0.330514in}}{\pgfqpoint{3.912343in}{3.019486in}}%
\pgfusepath{clip}%
\pgfsetbuttcap%
\pgfsetroundjoin%
\pgfsetlinewidth{0.803000pt}%
\definecolor{currentstroke}{rgb}{0.800000,0.800000,0.800000}%
\pgfsetstrokecolor{currentstroke}%
\pgfsetdash{{2.960000pt}{1.280000pt}}{0.000000pt}%
\pgfpathmoveto{\pgfqpoint{3.408677in}{0.330514in}}%
\pgfpathlineto{\pgfqpoint{3.408677in}{3.350000in}}%
\pgfusepath{stroke}%
\end{pgfscope}%
\begin{pgfscope}%
\definecolor{textcolor}{rgb}{0.150000,0.150000,0.150000}%
\pgfsetstrokecolor{textcolor}%
\pgfsetfillcolor{textcolor}%
\pgftext[x=3.408677in,y=0.252737in,,top]{\color{textcolor}\rmfamily\fontsize{8.330000}{9.996000}\selectfont \(\displaystyle 2\)}%
\end{pgfscope}%
\begin{pgfscope}%
\pgfpathrectangle{\pgfqpoint{0.437657in}{0.330514in}}{\pgfqpoint{3.912343in}{3.019486in}}%
\pgfusepath{clip}%
\pgfsetbuttcap%
\pgfsetroundjoin%
\pgfsetlinewidth{0.803000pt}%
\definecolor{currentstroke}{rgb}{0.800000,0.800000,0.800000}%
\pgfsetstrokecolor{currentstroke}%
\pgfsetdash{{2.960000pt}{1.280000pt}}{0.000000pt}%
\pgfpathmoveto{\pgfqpoint{3.958730in}{0.330514in}}%
\pgfpathlineto{\pgfqpoint{3.958730in}{3.350000in}}%
\pgfusepath{stroke}%
\end{pgfscope}%
\begin{pgfscope}%
\definecolor{textcolor}{rgb}{0.150000,0.150000,0.150000}%
\pgfsetstrokecolor{textcolor}%
\pgfsetfillcolor{textcolor}%
\pgftext[x=3.958730in,y=0.252737in,,top]{\color{textcolor}\rmfamily\fontsize{8.330000}{9.996000}\selectfont \(\displaystyle 3\)}%
\end{pgfscope}%
\begin{pgfscope}%
\pgfpathrectangle{\pgfqpoint{0.437657in}{0.330514in}}{\pgfqpoint{3.912343in}{3.019486in}}%
\pgfusepath{clip}%
\pgfsetbuttcap%
\pgfsetroundjoin%
\pgfsetlinewidth{0.803000pt}%
\definecolor{currentstroke}{rgb}{0.800000,0.800000,0.800000}%
\pgfsetstrokecolor{currentstroke}%
\pgfsetdash{{2.960000pt}{1.280000pt}}{0.000000pt}%
\pgfpathmoveto{\pgfqpoint{0.437657in}{0.645723in}}%
\pgfpathlineto{\pgfqpoint{4.350000in}{0.645723in}}%
\pgfusepath{stroke}%
\end{pgfscope}%
\begin{pgfscope}%
\definecolor{textcolor}{rgb}{0.150000,0.150000,0.150000}%
\pgfsetstrokecolor{textcolor}%
\pgfsetfillcolor{textcolor}%
\pgftext[x=0.150000in,y=0.605577in,left,base]{\color{textcolor}\rmfamily\fontsize{8.330000}{9.996000}\selectfont \(\displaystyle -30\)}%
\end{pgfscope}%
\begin{pgfscope}%
\pgfpathrectangle{\pgfqpoint{0.437657in}{0.330514in}}{\pgfqpoint{3.912343in}{3.019486in}}%
\pgfusepath{clip}%
\pgfsetbuttcap%
\pgfsetroundjoin%
\pgfsetlinewidth{0.803000pt}%
\definecolor{currentstroke}{rgb}{0.800000,0.800000,0.800000}%
\pgfsetstrokecolor{currentstroke}%
\pgfsetdash{{2.960000pt}{1.280000pt}}{0.000000pt}%
\pgfpathmoveto{\pgfqpoint{0.437657in}{1.051062in}}%
\pgfpathlineto{\pgfqpoint{4.350000in}{1.051062in}}%
\pgfusepath{stroke}%
\end{pgfscope}%
\begin{pgfscope}%
\definecolor{textcolor}{rgb}{0.150000,0.150000,0.150000}%
\pgfsetstrokecolor{textcolor}%
\pgfsetfillcolor{textcolor}%
\pgftext[x=0.150000in,y=1.010916in,left,base]{\color{textcolor}\rmfamily\fontsize{8.330000}{9.996000}\selectfont \(\displaystyle -20\)}%
\end{pgfscope}%
\begin{pgfscope}%
\pgfpathrectangle{\pgfqpoint{0.437657in}{0.330514in}}{\pgfqpoint{3.912343in}{3.019486in}}%
\pgfusepath{clip}%
\pgfsetbuttcap%
\pgfsetroundjoin%
\pgfsetlinewidth{0.803000pt}%
\definecolor{currentstroke}{rgb}{0.800000,0.800000,0.800000}%
\pgfsetstrokecolor{currentstroke}%
\pgfsetdash{{2.960000pt}{1.280000pt}}{0.000000pt}%
\pgfpathmoveto{\pgfqpoint{0.437657in}{1.456401in}}%
\pgfpathlineto{\pgfqpoint{4.350000in}{1.456401in}}%
\pgfusepath{stroke}%
\end{pgfscope}%
\begin{pgfscope}%
\definecolor{textcolor}{rgb}{0.150000,0.150000,0.150000}%
\pgfsetstrokecolor{textcolor}%
\pgfsetfillcolor{textcolor}%
\pgftext[x=0.150000in,y=1.416255in,left,base]{\color{textcolor}\rmfamily\fontsize{8.330000}{9.996000}\selectfont \(\displaystyle -10\)}%
\end{pgfscope}%
\begin{pgfscope}%
\pgfpathrectangle{\pgfqpoint{0.437657in}{0.330514in}}{\pgfqpoint{3.912343in}{3.019486in}}%
\pgfusepath{clip}%
\pgfsetbuttcap%
\pgfsetroundjoin%
\pgfsetlinewidth{0.803000pt}%
\definecolor{currentstroke}{rgb}{0.800000,0.800000,0.800000}%
\pgfsetstrokecolor{currentstroke}%
\pgfsetdash{{2.960000pt}{1.280000pt}}{0.000000pt}%
\pgfpathmoveto{\pgfqpoint{0.437657in}{1.861740in}}%
\pgfpathlineto{\pgfqpoint{4.350000in}{1.861740in}}%
\pgfusepath{stroke}%
\end{pgfscope}%
\begin{pgfscope}%
\definecolor{textcolor}{rgb}{0.150000,0.150000,0.150000}%
\pgfsetstrokecolor{textcolor}%
\pgfsetfillcolor{textcolor}%
\pgftext[x=0.300851in,y=1.821594in,left,base]{\color{textcolor}\rmfamily\fontsize{8.330000}{9.996000}\selectfont \(\displaystyle 0\)}%
\end{pgfscope}%
\begin{pgfscope}%
\pgfpathrectangle{\pgfqpoint{0.437657in}{0.330514in}}{\pgfqpoint{3.912343in}{3.019486in}}%
\pgfusepath{clip}%
\pgfsetbuttcap%
\pgfsetroundjoin%
\pgfsetlinewidth{0.803000pt}%
\definecolor{currentstroke}{rgb}{0.800000,0.800000,0.800000}%
\pgfsetstrokecolor{currentstroke}%
\pgfsetdash{{2.960000pt}{1.280000pt}}{0.000000pt}%
\pgfpathmoveto{\pgfqpoint{0.437657in}{2.267079in}}%
\pgfpathlineto{\pgfqpoint{4.350000in}{2.267079in}}%
\pgfusepath{stroke}%
\end{pgfscope}%
\begin{pgfscope}%
\definecolor{textcolor}{rgb}{0.150000,0.150000,0.150000}%
\pgfsetstrokecolor{textcolor}%
\pgfsetfillcolor{textcolor}%
\pgftext[x=0.241822in,y=2.226933in,left,base]{\color{textcolor}\rmfamily\fontsize{8.330000}{9.996000}\selectfont \(\displaystyle 10\)}%
\end{pgfscope}%
\begin{pgfscope}%
\pgfpathrectangle{\pgfqpoint{0.437657in}{0.330514in}}{\pgfqpoint{3.912343in}{3.019486in}}%
\pgfusepath{clip}%
\pgfsetbuttcap%
\pgfsetroundjoin%
\pgfsetlinewidth{0.803000pt}%
\definecolor{currentstroke}{rgb}{0.800000,0.800000,0.800000}%
\pgfsetstrokecolor{currentstroke}%
\pgfsetdash{{2.960000pt}{1.280000pt}}{0.000000pt}%
\pgfpathmoveto{\pgfqpoint{0.437657in}{2.672418in}}%
\pgfpathlineto{\pgfqpoint{4.350000in}{2.672418in}}%
\pgfusepath{stroke}%
\end{pgfscope}%
\begin{pgfscope}%
\definecolor{textcolor}{rgb}{0.150000,0.150000,0.150000}%
\pgfsetstrokecolor{textcolor}%
\pgfsetfillcolor{textcolor}%
\pgftext[x=0.241822in,y=2.632272in,left,base]{\color{textcolor}\rmfamily\fontsize{8.330000}{9.996000}\selectfont \(\displaystyle 20\)}%
\end{pgfscope}%
\begin{pgfscope}%
\pgfpathrectangle{\pgfqpoint{0.437657in}{0.330514in}}{\pgfqpoint{3.912343in}{3.019486in}}%
\pgfusepath{clip}%
\pgfsetbuttcap%
\pgfsetroundjoin%
\pgfsetlinewidth{0.803000pt}%
\definecolor{currentstroke}{rgb}{0.800000,0.800000,0.800000}%
\pgfsetstrokecolor{currentstroke}%
\pgfsetdash{{2.960000pt}{1.280000pt}}{0.000000pt}%
\pgfpathmoveto{\pgfqpoint{0.437657in}{3.077757in}}%
\pgfpathlineto{\pgfqpoint{4.350000in}{3.077757in}}%
\pgfusepath{stroke}%
\end{pgfscope}%
\begin{pgfscope}%
\definecolor{textcolor}{rgb}{0.150000,0.150000,0.150000}%
\pgfsetstrokecolor{textcolor}%
\pgfsetfillcolor{textcolor}%
\pgftext[x=0.241822in,y=3.037611in,left,base]{\color{textcolor}\rmfamily\fontsize{8.330000}{9.996000}\selectfont \(\displaystyle 30\)}%
\end{pgfscope}%
\begin{pgfscope}%
\pgfpathrectangle{\pgfqpoint{0.437657in}{0.330514in}}{\pgfqpoint{3.912343in}{3.019486in}}%
\pgfusepath{clip}%
\pgfsetbuttcap%
\pgfsetroundjoin%
\definecolor{currentfill}{rgb}{0.282353,0.470588,0.811765}%
\pgfsetfillcolor{currentfill}%
\pgfsetlinewidth{0.240900pt}%
\definecolor{currentstroke}{rgb}{0.282353,0.470588,0.811765}%
\pgfsetstrokecolor{currentstroke}%
\pgfsetdash{}{0pt}%
\pgfpathmoveto{\pgfqpoint{0.619907in}{0.429296in}}%
\pgfpathcurveto{\pgfqpoint{0.630221in}{0.429296in}}{\pgfqpoint{0.640113in}{0.433393in}}{\pgfqpoint{0.647406in}{0.440686in}}%
\pgfpathcurveto{\pgfqpoint{0.654699in}{0.447979in}}{\pgfqpoint{0.658796in}{0.457871in}}{\pgfqpoint{0.658796in}{0.468185in}}%
\pgfpathcurveto{\pgfqpoint{0.658796in}{0.478498in}}{\pgfqpoint{0.654699in}{0.488391in}}{\pgfqpoint{0.647406in}{0.495683in}}%
\pgfpathcurveto{\pgfqpoint{0.640113in}{0.502976in}}{\pgfqpoint{0.630221in}{0.507074in}}{\pgfqpoint{0.619907in}{0.507074in}}%
\pgfpathcurveto{\pgfqpoint{0.609594in}{0.507074in}}{\pgfqpoint{0.599701in}{0.502976in}}{\pgfqpoint{0.592409in}{0.495683in}}%
\pgfpathcurveto{\pgfqpoint{0.585116in}{0.488391in}}{\pgfqpoint{0.581018in}{0.478498in}}{\pgfqpoint{0.581018in}{0.468185in}}%
\pgfpathcurveto{\pgfqpoint{0.581018in}{0.457871in}}{\pgfqpoint{0.585116in}{0.447979in}}{\pgfqpoint{0.592409in}{0.440686in}}%
\pgfpathcurveto{\pgfqpoint{0.599701in}{0.433393in}}{\pgfqpoint{0.609594in}{0.429296in}}{\pgfqpoint{0.619907in}{0.429296in}}%
\pgfpathclose%
\pgfusepath{stroke,fill}%
\end{pgfscope}%
\begin{pgfscope}%
\pgfpathrectangle{\pgfqpoint{0.437657in}{0.330514in}}{\pgfqpoint{3.912343in}{3.019486in}}%
\pgfusepath{clip}%
\pgfsetbuttcap%
\pgfsetroundjoin%
\definecolor{currentfill}{rgb}{0.282353,0.470588,0.811765}%
\pgfsetfillcolor{currentfill}%
\pgfsetlinewidth{0.240900pt}%
\definecolor{currentstroke}{rgb}{0.282353,0.470588,0.811765}%
\pgfsetstrokecolor{currentstroke}%
\pgfsetdash{}{0pt}%
\pgfpathmoveto{\pgfqpoint{0.823427in}{0.689524in}}%
\pgfpathcurveto{\pgfqpoint{0.833740in}{0.689524in}}{\pgfqpoint{0.843633in}{0.693621in}}{\pgfqpoint{0.850926in}{0.700914in}}%
\pgfpathcurveto{\pgfqpoint{0.858218in}{0.708207in}}{\pgfqpoint{0.862316in}{0.718099in}}{\pgfqpoint{0.862316in}{0.728412in}}%
\pgfpathcurveto{\pgfqpoint{0.862316in}{0.738726in}}{\pgfqpoint{0.858218in}{0.748618in}}{\pgfqpoint{0.850926in}{0.755911in}}%
\pgfpathcurveto{\pgfqpoint{0.843633in}{0.763204in}}{\pgfqpoint{0.833740in}{0.767301in}}{\pgfqpoint{0.823427in}{0.767301in}}%
\pgfpathcurveto{\pgfqpoint{0.813114in}{0.767301in}}{\pgfqpoint{0.803221in}{0.763204in}}{\pgfqpoint{0.795928in}{0.755911in}}%
\pgfpathcurveto{\pgfqpoint{0.788636in}{0.748618in}}{\pgfqpoint{0.784538in}{0.738726in}}{\pgfqpoint{0.784538in}{0.728412in}}%
\pgfpathcurveto{\pgfqpoint{0.784538in}{0.718099in}}{\pgfqpoint{0.788636in}{0.708207in}}{\pgfqpoint{0.795928in}{0.700914in}}%
\pgfpathcurveto{\pgfqpoint{0.803221in}{0.693621in}}{\pgfqpoint{0.813114in}{0.689524in}}{\pgfqpoint{0.823427in}{0.689524in}}%
\pgfpathclose%
\pgfusepath{stroke,fill}%
\end{pgfscope}%
\begin{pgfscope}%
\pgfpathrectangle{\pgfqpoint{0.437657in}{0.330514in}}{\pgfqpoint{3.912343in}{3.019486in}}%
\pgfusepath{clip}%
\pgfsetbuttcap%
\pgfsetroundjoin%
\definecolor{currentfill}{rgb}{0.282353,0.470588,0.811765}%
\pgfsetfillcolor{currentfill}%
\pgfsetlinewidth{0.240900pt}%
\definecolor{currentstroke}{rgb}{0.282353,0.470588,0.811765}%
\pgfsetstrokecolor{currentstroke}%
\pgfsetdash{}{0pt}%
\pgfpathmoveto{\pgfqpoint{1.378981in}{1.181200in}}%
\pgfpathcurveto{\pgfqpoint{1.389294in}{1.181200in}}{\pgfqpoint{1.399187in}{1.185297in}}{\pgfqpoint{1.406479in}{1.192590in}}%
\pgfpathcurveto{\pgfqpoint{1.413772in}{1.199883in}}{\pgfqpoint{1.417870in}{1.209775in}}{\pgfqpoint{1.417870in}{1.220089in}}%
\pgfpathcurveto{\pgfqpoint{1.417870in}{1.230402in}}{\pgfqpoint{1.413772in}{1.240294in}}{\pgfqpoint{1.406479in}{1.247587in}}%
\pgfpathcurveto{\pgfqpoint{1.399187in}{1.254880in}}{\pgfqpoint{1.389294in}{1.258977in}}{\pgfqpoint{1.378981in}{1.258977in}}%
\pgfpathcurveto{\pgfqpoint{1.368667in}{1.258977in}}{\pgfqpoint{1.358775in}{1.254880in}}{\pgfqpoint{1.351482in}{1.247587in}}%
\pgfpathcurveto{\pgfqpoint{1.344189in}{1.240294in}}{\pgfqpoint{1.340092in}{1.230402in}}{\pgfqpoint{1.340092in}{1.220089in}}%
\pgfpathcurveto{\pgfqpoint{1.340092in}{1.209775in}}{\pgfqpoint{1.344189in}{1.199883in}}{\pgfqpoint{1.351482in}{1.192590in}}%
\pgfpathcurveto{\pgfqpoint{1.358775in}{1.185297in}}{\pgfqpoint{1.368667in}{1.181200in}}{\pgfqpoint{1.378981in}{1.181200in}}%
\pgfpathclose%
\pgfusepath{stroke,fill}%
\end{pgfscope}%
\begin{pgfscope}%
\pgfpathrectangle{\pgfqpoint{0.437657in}{0.330514in}}{\pgfqpoint{3.912343in}{3.019486in}}%
\pgfusepath{clip}%
\pgfsetbuttcap%
\pgfsetroundjoin%
\definecolor{currentfill}{rgb}{0.282353,0.470588,0.811765}%
\pgfsetfillcolor{currentfill}%
\pgfsetlinewidth{0.240900pt}%
\definecolor{currentstroke}{rgb}{0.282353,0.470588,0.811765}%
\pgfsetstrokecolor{currentstroke}%
\pgfsetdash{}{0pt}%
\pgfpathmoveto{\pgfqpoint{1.499992in}{1.385490in}}%
\pgfpathcurveto{\pgfqpoint{1.510306in}{1.385490in}}{\pgfqpoint{1.520198in}{1.389588in}}{\pgfqpoint{1.527491in}{1.396881in}}%
\pgfpathcurveto{\pgfqpoint{1.534784in}{1.404174in}}{\pgfqpoint{1.538881in}{1.414066in}}{\pgfqpoint{1.538881in}{1.424379in}}%
\pgfpathcurveto{\pgfqpoint{1.538881in}{1.434693in}}{\pgfqpoint{1.534784in}{1.444585in}}{\pgfqpoint{1.527491in}{1.451878in}}%
\pgfpathcurveto{\pgfqpoint{1.520198in}{1.459171in}}{\pgfqpoint{1.510306in}{1.463268in}}{\pgfqpoint{1.499992in}{1.463268in}}%
\pgfpathcurveto{\pgfqpoint{1.489679in}{1.463268in}}{\pgfqpoint{1.479786in}{1.459171in}}{\pgfqpoint{1.472494in}{1.451878in}}%
\pgfpathcurveto{\pgfqpoint{1.465201in}{1.444585in}}{\pgfqpoint{1.461103in}{1.434693in}}{\pgfqpoint{1.461103in}{1.424379in}}%
\pgfpathcurveto{\pgfqpoint{1.461103in}{1.414066in}}{\pgfqpoint{1.465201in}{1.404174in}}{\pgfqpoint{1.472494in}{1.396881in}}%
\pgfpathcurveto{\pgfqpoint{1.479786in}{1.389588in}}{\pgfqpoint{1.489679in}{1.385490in}}{\pgfqpoint{1.499992in}{1.385490in}}%
\pgfpathclose%
\pgfusepath{stroke,fill}%
\end{pgfscope}%
\begin{pgfscope}%
\pgfpathrectangle{\pgfqpoint{0.437657in}{0.330514in}}{\pgfqpoint{3.912343in}{3.019486in}}%
\pgfusepath{clip}%
\pgfsetbuttcap%
\pgfsetroundjoin%
\definecolor{currentfill}{rgb}{0.282353,0.470588,0.811765}%
\pgfsetfillcolor{currentfill}%
\pgfsetlinewidth{0.240900pt}%
\definecolor{currentstroke}{rgb}{0.282353,0.470588,0.811765}%
\pgfsetstrokecolor{currentstroke}%
\pgfsetdash{}{0pt}%
\pgfpathmoveto{\pgfqpoint{1.967537in}{1.501012in}}%
\pgfpathcurveto{\pgfqpoint{1.977851in}{1.501012in}}{\pgfqpoint{1.987743in}{1.505110in}}{\pgfqpoint{1.995036in}{1.512402in}}%
\pgfpathcurveto{\pgfqpoint{2.002329in}{1.519695in}}{\pgfqpoint{2.006426in}{1.529588in}}{\pgfqpoint{2.006426in}{1.539901in}}%
\pgfpathcurveto{\pgfqpoint{2.006426in}{1.550214in}}{\pgfqpoint{2.002329in}{1.560107in}}{\pgfqpoint{1.995036in}{1.567400in}}%
\pgfpathcurveto{\pgfqpoint{1.987743in}{1.574692in}}{\pgfqpoint{1.977851in}{1.578790in}}{\pgfqpoint{1.967537in}{1.578790in}}%
\pgfpathcurveto{\pgfqpoint{1.957224in}{1.578790in}}{\pgfqpoint{1.947332in}{1.574692in}}{\pgfqpoint{1.940039in}{1.567400in}}%
\pgfpathcurveto{\pgfqpoint{1.932746in}{1.560107in}}{\pgfqpoint{1.928649in}{1.550214in}}{\pgfqpoint{1.928649in}{1.539901in}}%
\pgfpathcurveto{\pgfqpoint{1.928649in}{1.529588in}}{\pgfqpoint{1.932746in}{1.519695in}}{\pgfqpoint{1.940039in}{1.512402in}}%
\pgfpathcurveto{\pgfqpoint{1.947332in}{1.505110in}}{\pgfqpoint{1.957224in}{1.501012in}}{\pgfqpoint{1.967537in}{1.501012in}}%
\pgfpathclose%
\pgfusepath{stroke,fill}%
\end{pgfscope}%
\begin{pgfscope}%
\pgfpathrectangle{\pgfqpoint{0.437657in}{0.330514in}}{\pgfqpoint{3.912343in}{3.019486in}}%
\pgfusepath{clip}%
\pgfsetbuttcap%
\pgfsetroundjoin%
\definecolor{currentfill}{rgb}{0.282353,0.470588,0.811765}%
\pgfsetfillcolor{currentfill}%
\pgfsetlinewidth{0.240900pt}%
\definecolor{currentstroke}{rgb}{0.282353,0.470588,0.811765}%
\pgfsetstrokecolor{currentstroke}%
\pgfsetdash{}{0pt}%
\pgfpathmoveto{\pgfqpoint{2.286568in}{1.820825in}}%
\pgfpathcurveto{\pgfqpoint{2.296882in}{1.820825in}}{\pgfqpoint{2.306774in}{1.824922in}}{\pgfqpoint{2.314067in}{1.832215in}}%
\pgfpathcurveto{\pgfqpoint{2.321360in}{1.839508in}}{\pgfqpoint{2.325457in}{1.849400in}}{\pgfqpoint{2.325457in}{1.859713in}}%
\pgfpathcurveto{\pgfqpoint{2.325457in}{1.870027in}}{\pgfqpoint{2.321360in}{1.879919in}}{\pgfqpoint{2.314067in}{1.887212in}}%
\pgfpathcurveto{\pgfqpoint{2.306774in}{1.894505in}}{\pgfqpoint{2.296882in}{1.898602in}}{\pgfqpoint{2.286568in}{1.898602in}}%
\pgfpathcurveto{\pgfqpoint{2.276255in}{1.898602in}}{\pgfqpoint{2.266362in}{1.894505in}}{\pgfqpoint{2.259070in}{1.887212in}}%
\pgfpathcurveto{\pgfqpoint{2.251777in}{1.879919in}}{\pgfqpoint{2.247679in}{1.870027in}}{\pgfqpoint{2.247679in}{1.859713in}}%
\pgfpathcurveto{\pgfqpoint{2.247679in}{1.849400in}}{\pgfqpoint{2.251777in}{1.839508in}}{\pgfqpoint{2.259070in}{1.832215in}}%
\pgfpathcurveto{\pgfqpoint{2.266362in}{1.824922in}}{\pgfqpoint{2.276255in}{1.820825in}}{\pgfqpoint{2.286568in}{1.820825in}}%
\pgfpathclose%
\pgfusepath{stroke,fill}%
\end{pgfscope}%
\begin{pgfscope}%
\pgfpathrectangle{\pgfqpoint{0.437657in}{0.330514in}}{\pgfqpoint{3.912343in}{3.019486in}}%
\pgfusepath{clip}%
\pgfsetbuttcap%
\pgfsetroundjoin%
\definecolor{currentfill}{rgb}{0.282353,0.470588,0.811765}%
\pgfsetfillcolor{currentfill}%
\pgfsetlinewidth{0.240900pt}%
\definecolor{currentstroke}{rgb}{0.282353,0.470588,0.811765}%
\pgfsetstrokecolor{currentstroke}%
\pgfsetdash{}{0pt}%
\pgfpathmoveto{\pgfqpoint{2.704609in}{2.124018in}}%
\pgfpathcurveto{\pgfqpoint{2.714922in}{2.124018in}}{\pgfqpoint{2.724815in}{2.128116in}}{\pgfqpoint{2.732107in}{2.135408in}}%
\pgfpathcurveto{\pgfqpoint{2.739400in}{2.142701in}}{\pgfqpoint{2.743498in}{2.152594in}}{\pgfqpoint{2.743498in}{2.162907in}}%
\pgfpathcurveto{\pgfqpoint{2.743498in}{2.173220in}}{\pgfqpoint{2.739400in}{2.183113in}}{\pgfqpoint{2.732107in}{2.190406in}}%
\pgfpathcurveto{\pgfqpoint{2.724815in}{2.197698in}}{\pgfqpoint{2.714922in}{2.201796in}}{\pgfqpoint{2.704609in}{2.201796in}}%
\pgfpathcurveto{\pgfqpoint{2.694295in}{2.201796in}}{\pgfqpoint{2.684403in}{2.197698in}}{\pgfqpoint{2.677110in}{2.190406in}}%
\pgfpathcurveto{\pgfqpoint{2.669817in}{2.183113in}}{\pgfqpoint{2.665720in}{2.173220in}}{\pgfqpoint{2.665720in}{2.162907in}}%
\pgfpathcurveto{\pgfqpoint{2.665720in}{2.152594in}}{\pgfqpoint{2.669817in}{2.142701in}}{\pgfqpoint{2.677110in}{2.135408in}}%
\pgfpathcurveto{\pgfqpoint{2.684403in}{2.128116in}}{\pgfqpoint{2.694295in}{2.124018in}}{\pgfqpoint{2.704609in}{2.124018in}}%
\pgfpathclose%
\pgfusepath{stroke,fill}%
\end{pgfscope}%
\begin{pgfscope}%
\pgfpathrectangle{\pgfqpoint{0.437657in}{0.330514in}}{\pgfqpoint{3.912343in}{3.019486in}}%
\pgfusepath{clip}%
\pgfsetbuttcap%
\pgfsetroundjoin%
\definecolor{currentfill}{rgb}{0.282353,0.470588,0.811765}%
\pgfsetfillcolor{currentfill}%
\pgfsetlinewidth{0.240900pt}%
\definecolor{currentstroke}{rgb}{0.282353,0.470588,0.811765}%
\pgfsetstrokecolor{currentstroke}%
\pgfsetdash{}{0pt}%
\pgfpathmoveto{\pgfqpoint{2.996137in}{2.364789in}}%
\pgfpathcurveto{\pgfqpoint{3.006450in}{2.364789in}}{\pgfqpoint{3.016343in}{2.368887in}}{\pgfqpoint{3.023635in}{2.376180in}}%
\pgfpathcurveto{\pgfqpoint{3.030928in}{2.383472in}}{\pgfqpoint{3.035026in}{2.393365in}}{\pgfqpoint{3.035026in}{2.403678in}}%
\pgfpathcurveto{\pgfqpoint{3.035026in}{2.413992in}}{\pgfqpoint{3.030928in}{2.423884in}}{\pgfqpoint{3.023635in}{2.431177in}}%
\pgfpathcurveto{\pgfqpoint{3.016343in}{2.438470in}}{\pgfqpoint{3.006450in}{2.442567in}}{\pgfqpoint{2.996137in}{2.442567in}}%
\pgfpathcurveto{\pgfqpoint{2.985823in}{2.442567in}}{\pgfqpoint{2.975931in}{2.438470in}}{\pgfqpoint{2.968638in}{2.431177in}}%
\pgfpathcurveto{\pgfqpoint{2.961346in}{2.423884in}}{\pgfqpoint{2.957248in}{2.413992in}}{\pgfqpoint{2.957248in}{2.403678in}}%
\pgfpathcurveto{\pgfqpoint{2.957248in}{2.393365in}}{\pgfqpoint{2.961346in}{2.383472in}}{\pgfqpoint{2.968638in}{2.376180in}}%
\pgfpathcurveto{\pgfqpoint{2.975931in}{2.368887in}}{\pgfqpoint{2.985823in}{2.364789in}}{\pgfqpoint{2.996137in}{2.364789in}}%
\pgfpathclose%
\pgfusepath{stroke,fill}%
\end{pgfscope}%
\begin{pgfscope}%
\pgfpathrectangle{\pgfqpoint{0.437657in}{0.330514in}}{\pgfqpoint{3.912343in}{3.019486in}}%
\pgfusepath{clip}%
\pgfsetbuttcap%
\pgfsetroundjoin%
\definecolor{currentfill}{rgb}{0.282353,0.470588,0.811765}%
\pgfsetfillcolor{currentfill}%
\pgfsetlinewidth{0.240900pt}%
\definecolor{currentstroke}{rgb}{0.282353,0.470588,0.811765}%
\pgfsetstrokecolor{currentstroke}%
\pgfsetdash{}{0pt}%
\pgfpathmoveto{\pgfqpoint{3.601195in}{2.696762in}}%
\pgfpathcurveto{\pgfqpoint{3.611509in}{2.696762in}}{\pgfqpoint{3.621401in}{2.700860in}}{\pgfqpoint{3.628694in}{2.708152in}}%
\pgfpathcurveto{\pgfqpoint{3.635987in}{2.715445in}}{\pgfqpoint{3.640084in}{2.725337in}}{\pgfqpoint{3.640084in}{2.735651in}}%
\pgfpathcurveto{\pgfqpoint{3.640084in}{2.745964in}}{\pgfqpoint{3.635987in}{2.755857in}}{\pgfqpoint{3.628694in}{2.763150in}}%
\pgfpathcurveto{\pgfqpoint{3.621401in}{2.770442in}}{\pgfqpoint{3.611509in}{2.774540in}}{\pgfqpoint{3.601195in}{2.774540in}}%
\pgfpathcurveto{\pgfqpoint{3.590882in}{2.774540in}}{\pgfqpoint{3.580989in}{2.770442in}}{\pgfqpoint{3.573697in}{2.763150in}}%
\pgfpathcurveto{\pgfqpoint{3.566404in}{2.755857in}}{\pgfqpoint{3.562306in}{2.745964in}}{\pgfqpoint{3.562306in}{2.735651in}}%
\pgfpathcurveto{\pgfqpoint{3.562306in}{2.725337in}}{\pgfqpoint{3.566404in}{2.715445in}}{\pgfqpoint{3.573697in}{2.708152in}}%
\pgfpathcurveto{\pgfqpoint{3.580989in}{2.700860in}}{\pgfqpoint{3.590882in}{2.696762in}}{\pgfqpoint{3.601195in}{2.696762in}}%
\pgfpathclose%
\pgfusepath{stroke,fill}%
\end{pgfscope}%
\begin{pgfscope}%
\pgfpathrectangle{\pgfqpoint{0.437657in}{0.330514in}}{\pgfqpoint{3.912343in}{3.019486in}}%
\pgfusepath{clip}%
\pgfsetbuttcap%
\pgfsetroundjoin%
\definecolor{currentfill}{rgb}{0.282353,0.470588,0.811765}%
\pgfsetfillcolor{currentfill}%
\pgfsetlinewidth{0.240900pt}%
\definecolor{currentstroke}{rgb}{0.282353,0.470588,0.811765}%
\pgfsetstrokecolor{currentstroke}%
\pgfsetdash{}{0pt}%
\pgfpathmoveto{\pgfqpoint{3.672702in}{3.093184in}}%
\pgfpathcurveto{\pgfqpoint{3.683016in}{3.093184in}}{\pgfqpoint{3.692908in}{3.097281in}}{\pgfqpoint{3.700201in}{3.104574in}}%
\pgfpathcurveto{\pgfqpoint{3.707494in}{3.111867in}}{\pgfqpoint{3.711591in}{3.121759in}}{\pgfqpoint{3.711591in}{3.132072in}}%
\pgfpathcurveto{\pgfqpoint{3.711591in}{3.142386in}}{\pgfqpoint{3.707494in}{3.152278in}}{\pgfqpoint{3.700201in}{3.159571in}}%
\pgfpathcurveto{\pgfqpoint{3.692908in}{3.166864in}}{\pgfqpoint{3.683016in}{3.170961in}}{\pgfqpoint{3.672702in}{3.170961in}}%
\pgfpathcurveto{\pgfqpoint{3.662389in}{3.170961in}}{\pgfqpoint{3.652496in}{3.166864in}}{\pgfqpoint{3.645204in}{3.159571in}}%
\pgfpathcurveto{\pgfqpoint{3.637911in}{3.152278in}}{\pgfqpoint{3.633813in}{3.142386in}}{\pgfqpoint{3.633813in}{3.132072in}}%
\pgfpathcurveto{\pgfqpoint{3.633813in}{3.121759in}}{\pgfqpoint{3.637911in}{3.111867in}}{\pgfqpoint{3.645204in}{3.104574in}}%
\pgfpathcurveto{\pgfqpoint{3.652496in}{3.097281in}}{\pgfqpoint{3.662389in}{3.093184in}}{\pgfqpoint{3.672702in}{3.093184in}}%
\pgfpathclose%
\pgfusepath{stroke,fill}%
\end{pgfscope}%
\begin{pgfscope}%
\pgfpathrectangle{\pgfqpoint{0.437657in}{0.330514in}}{\pgfqpoint{3.912343in}{3.019486in}}%
\pgfusepath{clip}%
\pgfsetbuttcap%
\pgfsetroundjoin%
\definecolor{currentfill}{rgb}{0.282353,0.470588,0.811765}%
\pgfsetfillcolor{currentfill}%
\pgfsetlinewidth{0.240900pt}%
\definecolor{currentstroke}{rgb}{0.282353,0.470588,0.811765}%
\pgfsetstrokecolor{currentstroke}%
\pgfsetdash{}{0pt}%
\pgfpathmoveto{\pgfqpoint{4.167750in}{3.173441in}}%
\pgfpathcurveto{\pgfqpoint{4.178064in}{3.173441in}}{\pgfqpoint{4.187956in}{3.177538in}}{\pgfqpoint{4.195249in}{3.184831in}}%
\pgfpathcurveto{\pgfqpoint{4.202541in}{3.192124in}}{\pgfqpoint{4.206639in}{3.202016in}}{\pgfqpoint{4.206639in}{3.212330in}}%
\pgfpathcurveto{\pgfqpoint{4.206639in}{3.222643in}}{\pgfqpoint{4.202541in}{3.232535in}}{\pgfqpoint{4.195249in}{3.239828in}}%
\pgfpathcurveto{\pgfqpoint{4.187956in}{3.247121in}}{\pgfqpoint{4.178064in}{3.251218in}}{\pgfqpoint{4.167750in}{3.251218in}}%
\pgfpathcurveto{\pgfqpoint{4.157437in}{3.251218in}}{\pgfqpoint{4.147544in}{3.247121in}}{\pgfqpoint{4.140251in}{3.239828in}}%
\pgfpathcurveto{\pgfqpoint{4.132959in}{3.232535in}}{\pgfqpoint{4.128861in}{3.222643in}}{\pgfqpoint{4.128861in}{3.212330in}}%
\pgfpathcurveto{\pgfqpoint{4.128861in}{3.202016in}}{\pgfqpoint{4.132959in}{3.192124in}}{\pgfqpoint{4.140251in}{3.184831in}}%
\pgfpathcurveto{\pgfqpoint{4.147544in}{3.177538in}}{\pgfqpoint{4.157437in}{3.173441in}}{\pgfqpoint{4.167750in}{3.173441in}}%
\pgfpathclose%
\pgfusepath{stroke,fill}%
\end{pgfscope}%
\begin{pgfscope}%
\pgfsetrectcap%
\pgfsetmiterjoin%
\pgfsetlinewidth{1.003750pt}%
\definecolor{currentstroke}{rgb}{0.400000,0.400000,0.400000}%
\pgfsetstrokecolor{currentstroke}%
\pgfsetdash{}{0pt}%
\pgfpathmoveto{\pgfqpoint{0.437657in}{0.330514in}}%
\pgfpathlineto{\pgfqpoint{0.437657in}{3.350000in}}%
\pgfusepath{stroke}%
\end{pgfscope}%
\begin{pgfscope}%
\pgfsetrectcap%
\pgfsetmiterjoin%
\pgfsetlinewidth{1.003750pt}%
\definecolor{currentstroke}{rgb}{0.400000,0.400000,0.400000}%
\pgfsetstrokecolor{currentstroke}%
\pgfsetdash{}{0pt}%
\pgfpathmoveto{\pgfqpoint{0.437657in}{0.330514in}}%
\pgfpathlineto{\pgfqpoint{4.350000in}{0.330514in}}%
\pgfusepath{stroke}%
\end{pgfscope}%
\end{pgfpicture}%
\makeatother%
\endgroup%


        \caption{Gráfico de dispersão dos pontos}
        \label{fig:reta:dados}
    \end{figure}


\subsection{Texto dos Eixos e do Título}

    Em um gŕafico como este, é necessário colocar texto no título e nos rótulos (\textit{label}) dos eixos.

    \begin{listing}[H]
        \caption{Montagem dos textos do gráfico}
        \label{code:reta:textos}

        \pyinclude[firstline=19, lastline=25]{recursos/reta/reta.py}
    \end{listing}


\subsection{Resultado}

    \begin{figure}[htbp]
        \centering
        %% Creator: Matplotlib, PGF backend
%%
%% To include the figure in your LaTeX document, write
%%   \input{<filename>.pgf}
%%
%% Make sure the required packages are loaded in your preamble
%%   \usepackage{pgf}
%%
%% Figures using additional raster images can only be included by \input if
%% they are in the same directory as the main LaTeX file. For loading figures
%% from other directories you can use the `import` package
%%   \usepackage{import}
%% and then include the figures with
%%   \import{<path to file>}{<filename>.pgf}
%%
%% Matplotlib used the following preamble
%%   
%%       \usepackage[portuguese]{babel}
%%       \usepackage[T1]{fontenc}
%%       \usepackage[utf8]{inputenc}
%%   \usepackage{fontspec}
%%
\begingroup%
\makeatletter%
\begin{pgfpicture}%
\pgfpathrectangle{\pgfpointorigin}{\pgfqpoint{4.500000in}{3.500000in}}%
\pgfusepath{use as bounding box, clip}%
\begin{pgfscope}%
\pgfsetbuttcap%
\pgfsetmiterjoin%
\definecolor{currentfill}{rgb}{1.000000,1.000000,1.000000}%
\pgfsetfillcolor{currentfill}%
\pgfsetlinewidth{0.000000pt}%
\definecolor{currentstroke}{rgb}{1.000000,1.000000,1.000000}%
\pgfsetstrokecolor{currentstroke}%
\pgfsetdash{}{0pt}%
\pgfpathmoveto{\pgfqpoint{0.000000in}{0.000000in}}%
\pgfpathlineto{\pgfqpoint{4.500000in}{0.000000in}}%
\pgfpathlineto{\pgfqpoint{4.500000in}{3.500000in}}%
\pgfpathlineto{\pgfqpoint{0.000000in}{3.500000in}}%
\pgfpathclose%
\pgfusepath{fill}%
\end{pgfscope}%
\begin{pgfscope}%
\pgfsetbuttcap%
\pgfsetmiterjoin%
\definecolor{currentfill}{rgb}{1.000000,1.000000,1.000000}%
\pgfsetfillcolor{currentfill}%
\pgfsetlinewidth{0.000000pt}%
\definecolor{currentstroke}{rgb}{0.000000,0.000000,0.000000}%
\pgfsetstrokecolor{currentstroke}%
\pgfsetstrokeopacity{0.000000}%
\pgfsetdash{}{0pt}%
\pgfpathmoveto{\pgfqpoint{0.632102in}{0.524958in}}%
\pgfpathlineto{\pgfqpoint{4.350000in}{0.524958in}}%
\pgfpathlineto{\pgfqpoint{4.350000in}{3.149333in}}%
\pgfpathlineto{\pgfqpoint{0.632102in}{3.149333in}}%
\pgfpathclose%
\pgfusepath{fill}%
\end{pgfscope}%
\begin{pgfscope}%
\pgfpathrectangle{\pgfqpoint{0.632102in}{0.524958in}}{\pgfqpoint{3.717898in}{2.624375in}}%
\pgfusepath{clip}%
\pgfsetbuttcap%
\pgfsetroundjoin%
\pgfsetlinewidth{0.803000pt}%
\definecolor{currentstroke}{rgb}{0.800000,0.800000,0.800000}%
\pgfsetstrokecolor{currentstroke}%
\pgfsetdash{{2.960000pt}{1.280000pt}}{0.000000pt}%
\pgfpathmoveto{\pgfqpoint{0.841884in}{0.524958in}}%
\pgfpathlineto{\pgfqpoint{0.841884in}{3.149333in}}%
\pgfusepath{stroke}%
\end{pgfscope}%
\begin{pgfscope}%
\definecolor{textcolor}{rgb}{0.150000,0.150000,0.150000}%
\pgfsetstrokecolor{textcolor}%
\pgfsetfillcolor{textcolor}%
\pgftext[x=0.841884in,y=0.447181in,,top]{\color{textcolor}\rmfamily\fontsize{8.330000}{9.996000}\selectfont \(\displaystyle -3\)}%
\end{pgfscope}%
\begin{pgfscope}%
\pgfpathrectangle{\pgfqpoint{0.632102in}{0.524958in}}{\pgfqpoint{3.717898in}{2.624375in}}%
\pgfusepath{clip}%
\pgfsetbuttcap%
\pgfsetroundjoin%
\pgfsetlinewidth{0.803000pt}%
\definecolor{currentstroke}{rgb}{0.800000,0.800000,0.800000}%
\pgfsetstrokecolor{currentstroke}%
\pgfsetdash{{2.960000pt}{1.280000pt}}{0.000000pt}%
\pgfpathmoveto{\pgfqpoint{1.364599in}{0.524958in}}%
\pgfpathlineto{\pgfqpoint{1.364599in}{3.149333in}}%
\pgfusepath{stroke}%
\end{pgfscope}%
\begin{pgfscope}%
\definecolor{textcolor}{rgb}{0.150000,0.150000,0.150000}%
\pgfsetstrokecolor{textcolor}%
\pgfsetfillcolor{textcolor}%
\pgftext[x=1.364599in,y=0.447181in,,top]{\color{textcolor}\rmfamily\fontsize{8.330000}{9.996000}\selectfont \(\displaystyle -2\)}%
\end{pgfscope}%
\begin{pgfscope}%
\pgfpathrectangle{\pgfqpoint{0.632102in}{0.524958in}}{\pgfqpoint{3.717898in}{2.624375in}}%
\pgfusepath{clip}%
\pgfsetbuttcap%
\pgfsetroundjoin%
\pgfsetlinewidth{0.803000pt}%
\definecolor{currentstroke}{rgb}{0.800000,0.800000,0.800000}%
\pgfsetstrokecolor{currentstroke}%
\pgfsetdash{{2.960000pt}{1.280000pt}}{0.000000pt}%
\pgfpathmoveto{\pgfqpoint{1.887314in}{0.524958in}}%
\pgfpathlineto{\pgfqpoint{1.887314in}{3.149333in}}%
\pgfusepath{stroke}%
\end{pgfscope}%
\begin{pgfscope}%
\definecolor{textcolor}{rgb}{0.150000,0.150000,0.150000}%
\pgfsetstrokecolor{textcolor}%
\pgfsetfillcolor{textcolor}%
\pgftext[x=1.887314in,y=0.447181in,,top]{\color{textcolor}\rmfamily\fontsize{8.330000}{9.996000}\selectfont \(\displaystyle -1\)}%
\end{pgfscope}%
\begin{pgfscope}%
\pgfpathrectangle{\pgfqpoint{0.632102in}{0.524958in}}{\pgfqpoint{3.717898in}{2.624375in}}%
\pgfusepath{clip}%
\pgfsetbuttcap%
\pgfsetroundjoin%
\pgfsetlinewidth{0.803000pt}%
\definecolor{currentstroke}{rgb}{0.800000,0.800000,0.800000}%
\pgfsetstrokecolor{currentstroke}%
\pgfsetdash{{2.960000pt}{1.280000pt}}{0.000000pt}%
\pgfpathmoveto{\pgfqpoint{2.410030in}{0.524958in}}%
\pgfpathlineto{\pgfqpoint{2.410030in}{3.149333in}}%
\pgfusepath{stroke}%
\end{pgfscope}%
\begin{pgfscope}%
\definecolor{textcolor}{rgb}{0.150000,0.150000,0.150000}%
\pgfsetstrokecolor{textcolor}%
\pgfsetfillcolor{textcolor}%
\pgftext[x=2.410030in,y=0.447181in,,top]{\color{textcolor}\rmfamily\fontsize{8.330000}{9.996000}\selectfont \(\displaystyle 0\)}%
\end{pgfscope}%
\begin{pgfscope}%
\pgfpathrectangle{\pgfqpoint{0.632102in}{0.524958in}}{\pgfqpoint{3.717898in}{2.624375in}}%
\pgfusepath{clip}%
\pgfsetbuttcap%
\pgfsetroundjoin%
\pgfsetlinewidth{0.803000pt}%
\definecolor{currentstroke}{rgb}{0.800000,0.800000,0.800000}%
\pgfsetstrokecolor{currentstroke}%
\pgfsetdash{{2.960000pt}{1.280000pt}}{0.000000pt}%
\pgfpathmoveto{\pgfqpoint{2.932745in}{0.524958in}}%
\pgfpathlineto{\pgfqpoint{2.932745in}{3.149333in}}%
\pgfusepath{stroke}%
\end{pgfscope}%
\begin{pgfscope}%
\definecolor{textcolor}{rgb}{0.150000,0.150000,0.150000}%
\pgfsetstrokecolor{textcolor}%
\pgfsetfillcolor{textcolor}%
\pgftext[x=2.932745in,y=0.447181in,,top]{\color{textcolor}\rmfamily\fontsize{8.330000}{9.996000}\selectfont \(\displaystyle 1\)}%
\end{pgfscope}%
\begin{pgfscope}%
\pgfpathrectangle{\pgfqpoint{0.632102in}{0.524958in}}{\pgfqpoint{3.717898in}{2.624375in}}%
\pgfusepath{clip}%
\pgfsetbuttcap%
\pgfsetroundjoin%
\pgfsetlinewidth{0.803000pt}%
\definecolor{currentstroke}{rgb}{0.800000,0.800000,0.800000}%
\pgfsetstrokecolor{currentstroke}%
\pgfsetdash{{2.960000pt}{1.280000pt}}{0.000000pt}%
\pgfpathmoveto{\pgfqpoint{3.455461in}{0.524958in}}%
\pgfpathlineto{\pgfqpoint{3.455461in}{3.149333in}}%
\pgfusepath{stroke}%
\end{pgfscope}%
\begin{pgfscope}%
\definecolor{textcolor}{rgb}{0.150000,0.150000,0.150000}%
\pgfsetstrokecolor{textcolor}%
\pgfsetfillcolor{textcolor}%
\pgftext[x=3.455461in,y=0.447181in,,top]{\color{textcolor}\rmfamily\fontsize{8.330000}{9.996000}\selectfont \(\displaystyle 2\)}%
\end{pgfscope}%
\begin{pgfscope}%
\pgfpathrectangle{\pgfqpoint{0.632102in}{0.524958in}}{\pgfqpoint{3.717898in}{2.624375in}}%
\pgfusepath{clip}%
\pgfsetbuttcap%
\pgfsetroundjoin%
\pgfsetlinewidth{0.803000pt}%
\definecolor{currentstroke}{rgb}{0.800000,0.800000,0.800000}%
\pgfsetstrokecolor{currentstroke}%
\pgfsetdash{{2.960000pt}{1.280000pt}}{0.000000pt}%
\pgfpathmoveto{\pgfqpoint{3.978176in}{0.524958in}}%
\pgfpathlineto{\pgfqpoint{3.978176in}{3.149333in}}%
\pgfusepath{stroke}%
\end{pgfscope}%
\begin{pgfscope}%
\definecolor{textcolor}{rgb}{0.150000,0.150000,0.150000}%
\pgfsetstrokecolor{textcolor}%
\pgfsetfillcolor{textcolor}%
\pgftext[x=3.978176in,y=0.447181in,,top]{\color{textcolor}\rmfamily\fontsize{8.330000}{9.996000}\selectfont \(\displaystyle 3\)}%
\end{pgfscope}%
\begin{pgfscope}%
\definecolor{textcolor}{rgb}{0.000000,0.000000,0.000000}%
\pgfsetstrokecolor{textcolor}%
\pgfsetfillcolor{textcolor}%
\pgftext[x=2.491051in,y=0.288889in,,top]{\color{textcolor}\rmfamily\fontsize{10.000000}{12.000000}\selectfont Tensão [V]}%
\end{pgfscope}%
\begin{pgfscope}%
\pgfpathrectangle{\pgfqpoint{0.632102in}{0.524958in}}{\pgfqpoint{3.717898in}{2.624375in}}%
\pgfusepath{clip}%
\pgfsetbuttcap%
\pgfsetroundjoin%
\pgfsetlinewidth{0.803000pt}%
\definecolor{currentstroke}{rgb}{0.800000,0.800000,0.800000}%
\pgfsetstrokecolor{currentstroke}%
\pgfsetdash{{2.960000pt}{1.280000pt}}{0.000000pt}%
\pgfpathmoveto{\pgfqpoint{0.632102in}{0.798921in}}%
\pgfpathlineto{\pgfqpoint{4.350000in}{0.798921in}}%
\pgfusepath{stroke}%
\end{pgfscope}%
\begin{pgfscope}%
\definecolor{textcolor}{rgb}{0.150000,0.150000,0.150000}%
\pgfsetstrokecolor{textcolor}%
\pgfsetfillcolor{textcolor}%
\pgftext[x=0.344444in,y=0.758775in,left,base]{\color{textcolor}\rmfamily\fontsize{8.330000}{9.996000}\selectfont \(\displaystyle -30\)}%
\end{pgfscope}%
\begin{pgfscope}%
\pgfpathrectangle{\pgfqpoint{0.632102in}{0.524958in}}{\pgfqpoint{3.717898in}{2.624375in}}%
\pgfusepath{clip}%
\pgfsetbuttcap%
\pgfsetroundjoin%
\pgfsetlinewidth{0.803000pt}%
\definecolor{currentstroke}{rgb}{0.800000,0.800000,0.800000}%
\pgfsetstrokecolor{currentstroke}%
\pgfsetdash{{2.960000pt}{1.280000pt}}{0.000000pt}%
\pgfpathmoveto{\pgfqpoint{0.632102in}{1.151220in}}%
\pgfpathlineto{\pgfqpoint{4.350000in}{1.151220in}}%
\pgfusepath{stroke}%
\end{pgfscope}%
\begin{pgfscope}%
\definecolor{textcolor}{rgb}{0.150000,0.150000,0.150000}%
\pgfsetstrokecolor{textcolor}%
\pgfsetfillcolor{textcolor}%
\pgftext[x=0.344444in,y=1.111074in,left,base]{\color{textcolor}\rmfamily\fontsize{8.330000}{9.996000}\selectfont \(\displaystyle -20\)}%
\end{pgfscope}%
\begin{pgfscope}%
\pgfpathrectangle{\pgfqpoint{0.632102in}{0.524958in}}{\pgfqpoint{3.717898in}{2.624375in}}%
\pgfusepath{clip}%
\pgfsetbuttcap%
\pgfsetroundjoin%
\pgfsetlinewidth{0.803000pt}%
\definecolor{currentstroke}{rgb}{0.800000,0.800000,0.800000}%
\pgfsetstrokecolor{currentstroke}%
\pgfsetdash{{2.960000pt}{1.280000pt}}{0.000000pt}%
\pgfpathmoveto{\pgfqpoint{0.632102in}{1.503519in}}%
\pgfpathlineto{\pgfqpoint{4.350000in}{1.503519in}}%
\pgfusepath{stroke}%
\end{pgfscope}%
\begin{pgfscope}%
\definecolor{textcolor}{rgb}{0.150000,0.150000,0.150000}%
\pgfsetstrokecolor{textcolor}%
\pgfsetfillcolor{textcolor}%
\pgftext[x=0.344444in,y=1.463373in,left,base]{\color{textcolor}\rmfamily\fontsize{8.330000}{9.996000}\selectfont \(\displaystyle -10\)}%
\end{pgfscope}%
\begin{pgfscope}%
\pgfpathrectangle{\pgfqpoint{0.632102in}{0.524958in}}{\pgfqpoint{3.717898in}{2.624375in}}%
\pgfusepath{clip}%
\pgfsetbuttcap%
\pgfsetroundjoin%
\pgfsetlinewidth{0.803000pt}%
\definecolor{currentstroke}{rgb}{0.800000,0.800000,0.800000}%
\pgfsetstrokecolor{currentstroke}%
\pgfsetdash{{2.960000pt}{1.280000pt}}{0.000000pt}%
\pgfpathmoveto{\pgfqpoint{0.632102in}{1.855818in}}%
\pgfpathlineto{\pgfqpoint{4.350000in}{1.855818in}}%
\pgfusepath{stroke}%
\end{pgfscope}%
\begin{pgfscope}%
\definecolor{textcolor}{rgb}{0.150000,0.150000,0.150000}%
\pgfsetstrokecolor{textcolor}%
\pgfsetfillcolor{textcolor}%
\pgftext[x=0.495295in,y=1.815672in,left,base]{\color{textcolor}\rmfamily\fontsize{8.330000}{9.996000}\selectfont \(\displaystyle 0\)}%
\end{pgfscope}%
\begin{pgfscope}%
\pgfpathrectangle{\pgfqpoint{0.632102in}{0.524958in}}{\pgfqpoint{3.717898in}{2.624375in}}%
\pgfusepath{clip}%
\pgfsetbuttcap%
\pgfsetroundjoin%
\pgfsetlinewidth{0.803000pt}%
\definecolor{currentstroke}{rgb}{0.800000,0.800000,0.800000}%
\pgfsetstrokecolor{currentstroke}%
\pgfsetdash{{2.960000pt}{1.280000pt}}{0.000000pt}%
\pgfpathmoveto{\pgfqpoint{0.632102in}{2.208117in}}%
\pgfpathlineto{\pgfqpoint{4.350000in}{2.208117in}}%
\pgfusepath{stroke}%
\end{pgfscope}%
\begin{pgfscope}%
\definecolor{textcolor}{rgb}{0.150000,0.150000,0.150000}%
\pgfsetstrokecolor{textcolor}%
\pgfsetfillcolor{textcolor}%
\pgftext[x=0.436267in,y=2.167971in,left,base]{\color{textcolor}\rmfamily\fontsize{8.330000}{9.996000}\selectfont \(\displaystyle 10\)}%
\end{pgfscope}%
\begin{pgfscope}%
\pgfpathrectangle{\pgfqpoint{0.632102in}{0.524958in}}{\pgfqpoint{3.717898in}{2.624375in}}%
\pgfusepath{clip}%
\pgfsetbuttcap%
\pgfsetroundjoin%
\pgfsetlinewidth{0.803000pt}%
\definecolor{currentstroke}{rgb}{0.800000,0.800000,0.800000}%
\pgfsetstrokecolor{currentstroke}%
\pgfsetdash{{2.960000pt}{1.280000pt}}{0.000000pt}%
\pgfpathmoveto{\pgfqpoint{0.632102in}{2.560415in}}%
\pgfpathlineto{\pgfqpoint{4.350000in}{2.560415in}}%
\pgfusepath{stroke}%
\end{pgfscope}%
\begin{pgfscope}%
\definecolor{textcolor}{rgb}{0.150000,0.150000,0.150000}%
\pgfsetstrokecolor{textcolor}%
\pgfsetfillcolor{textcolor}%
\pgftext[x=0.436267in,y=2.520269in,left,base]{\color{textcolor}\rmfamily\fontsize{8.330000}{9.996000}\selectfont \(\displaystyle 20\)}%
\end{pgfscope}%
\begin{pgfscope}%
\pgfpathrectangle{\pgfqpoint{0.632102in}{0.524958in}}{\pgfqpoint{3.717898in}{2.624375in}}%
\pgfusepath{clip}%
\pgfsetbuttcap%
\pgfsetroundjoin%
\pgfsetlinewidth{0.803000pt}%
\definecolor{currentstroke}{rgb}{0.800000,0.800000,0.800000}%
\pgfsetstrokecolor{currentstroke}%
\pgfsetdash{{2.960000pt}{1.280000pt}}{0.000000pt}%
\pgfpathmoveto{\pgfqpoint{0.632102in}{2.912714in}}%
\pgfpathlineto{\pgfqpoint{4.350000in}{2.912714in}}%
\pgfusepath{stroke}%
\end{pgfscope}%
\begin{pgfscope}%
\definecolor{textcolor}{rgb}{0.150000,0.150000,0.150000}%
\pgfsetstrokecolor{textcolor}%
\pgfsetfillcolor{textcolor}%
\pgftext[x=0.436267in,y=2.872568in,left,base]{\color{textcolor}\rmfamily\fontsize{8.330000}{9.996000}\selectfont \(\displaystyle 30\)}%
\end{pgfscope}%
\begin{pgfscope}%
\definecolor{textcolor}{rgb}{0.000000,0.000000,0.000000}%
\pgfsetstrokecolor{textcolor}%
\pgfsetfillcolor{textcolor}%
\pgftext[x=0.288889in,y=1.837146in,,bottom,rotate=90.000000]{\color{textcolor}\rmfamily\fontsize{10.000000}{12.000000}\selectfont Corrente [mA]}%
\end{pgfscope}%
\begin{pgfscope}%
\pgfpathrectangle{\pgfqpoint{0.632102in}{0.524958in}}{\pgfqpoint{3.717898in}{2.624375in}}%
\pgfusepath{clip}%
\pgfsetbuttcap%
\pgfsetroundjoin%
\definecolor{currentfill}{rgb}{0.282353,0.470588,0.811765}%
\pgfsetfillcolor{currentfill}%
\pgfsetlinewidth{0.240900pt}%
\definecolor{currentstroke}{rgb}{0.282353,0.470588,0.811765}%
\pgfsetstrokecolor{currentstroke}%
\pgfsetdash{}{0pt}%
\pgfpathmoveto{\pgfqpoint{0.805294in}{0.605725in}}%
\pgfpathcurveto{\pgfqpoint{0.815607in}{0.605725in}}{\pgfqpoint{0.825500in}{0.609823in}}{\pgfqpoint{0.832792in}{0.617116in}}%
\pgfpathcurveto{\pgfqpoint{0.840085in}{0.624408in}}{\pgfqpoint{0.844183in}{0.634301in}}{\pgfqpoint{0.844183in}{0.644614in}}%
\pgfpathcurveto{\pgfqpoint{0.844183in}{0.654928in}}{\pgfqpoint{0.840085in}{0.664820in}}{\pgfqpoint{0.832792in}{0.672113in}}%
\pgfpathcurveto{\pgfqpoint{0.825500in}{0.679406in}}{\pgfqpoint{0.815607in}{0.683503in}}{\pgfqpoint{0.805294in}{0.683503in}}%
\pgfpathcurveto{\pgfqpoint{0.794980in}{0.683503in}}{\pgfqpoint{0.785088in}{0.679406in}}{\pgfqpoint{0.777795in}{0.672113in}}%
\pgfpathcurveto{\pgfqpoint{0.770502in}{0.664820in}}{\pgfqpoint{0.766405in}{0.654928in}}{\pgfqpoint{0.766405in}{0.644614in}}%
\pgfpathcurveto{\pgfqpoint{0.766405in}{0.634301in}}{\pgfqpoint{0.770502in}{0.624408in}}{\pgfqpoint{0.777795in}{0.617116in}}%
\pgfpathcurveto{\pgfqpoint{0.785088in}{0.609823in}}{\pgfqpoint{0.794980in}{0.605725in}}{\pgfqpoint{0.805294in}{0.605725in}}%
\pgfpathclose%
\pgfusepath{stroke,fill}%
\end{pgfscope}%
\begin{pgfscope}%
\pgfpathrectangle{\pgfqpoint{0.632102in}{0.524958in}}{\pgfqpoint{3.717898in}{2.624375in}}%
\pgfusepath{clip}%
\pgfsetbuttcap%
\pgfsetroundjoin%
\definecolor{currentfill}{rgb}{0.282353,0.470588,0.811765}%
\pgfsetfillcolor{currentfill}%
\pgfsetlinewidth{0.240900pt}%
\definecolor{currentstroke}{rgb}{0.282353,0.470588,0.811765}%
\pgfsetstrokecolor{currentstroke}%
\pgfsetdash{}{0pt}%
\pgfpathmoveto{\pgfqpoint{0.998698in}{0.831901in}}%
\pgfpathcurveto{\pgfqpoint{1.009012in}{0.831901in}}{\pgfqpoint{1.018904in}{0.835999in}}{\pgfqpoint{1.026197in}{0.843292in}}%
\pgfpathcurveto{\pgfqpoint{1.033490in}{0.850584in}}{\pgfqpoint{1.037587in}{0.860477in}}{\pgfqpoint{1.037587in}{0.870790in}}%
\pgfpathcurveto{\pgfqpoint{1.037587in}{0.881104in}}{\pgfqpoint{1.033490in}{0.890996in}}{\pgfqpoint{1.026197in}{0.898289in}}%
\pgfpathcurveto{\pgfqpoint{1.018904in}{0.905581in}}{\pgfqpoint{1.009012in}{0.909679in}}{\pgfqpoint{0.998698in}{0.909679in}}%
\pgfpathcurveto{\pgfqpoint{0.988385in}{0.909679in}}{\pgfqpoint{0.978492in}{0.905581in}}{\pgfqpoint{0.971200in}{0.898289in}}%
\pgfpathcurveto{\pgfqpoint{0.963907in}{0.890996in}}{\pgfqpoint{0.959809in}{0.881104in}}{\pgfqpoint{0.959809in}{0.870790in}}%
\pgfpathcurveto{\pgfqpoint{0.959809in}{0.860477in}}{\pgfqpoint{0.963907in}{0.850584in}}{\pgfqpoint{0.971200in}{0.843292in}}%
\pgfpathcurveto{\pgfqpoint{0.978492in}{0.835999in}}{\pgfqpoint{0.988385in}{0.831901in}}{\pgfqpoint{0.998698in}{0.831901in}}%
\pgfpathclose%
\pgfusepath{stroke,fill}%
\end{pgfscope}%
\begin{pgfscope}%
\pgfpathrectangle{\pgfqpoint{0.632102in}{0.524958in}}{\pgfqpoint{3.717898in}{2.624375in}}%
\pgfusepath{clip}%
\pgfsetbuttcap%
\pgfsetroundjoin%
\definecolor{currentfill}{rgb}{0.282353,0.470588,0.811765}%
\pgfsetfillcolor{currentfill}%
\pgfsetlinewidth{0.240900pt}%
\definecolor{currentstroke}{rgb}{0.282353,0.470588,0.811765}%
\pgfsetstrokecolor{currentstroke}%
\pgfsetdash{}{0pt}%
\pgfpathmoveto{\pgfqpoint{1.526641in}{1.259240in}}%
\pgfpathcurveto{\pgfqpoint{1.536954in}{1.259240in}}{\pgfqpoint{1.546847in}{1.263337in}}{\pgfqpoint{1.554139in}{1.270630in}}%
\pgfpathcurveto{\pgfqpoint{1.561432in}{1.277923in}}{\pgfqpoint{1.565530in}{1.287815in}}{\pgfqpoint{1.565530in}{1.298129in}}%
\pgfpathcurveto{\pgfqpoint{1.565530in}{1.308442in}}{\pgfqpoint{1.561432in}{1.318335in}}{\pgfqpoint{1.554139in}{1.325627in}}%
\pgfpathcurveto{\pgfqpoint{1.546847in}{1.332920in}}{\pgfqpoint{1.536954in}{1.337018in}}{\pgfqpoint{1.526641in}{1.337018in}}%
\pgfpathcurveto{\pgfqpoint{1.516327in}{1.337018in}}{\pgfqpoint{1.506435in}{1.332920in}}{\pgfqpoint{1.499142in}{1.325627in}}%
\pgfpathcurveto{\pgfqpoint{1.491850in}{1.318335in}}{\pgfqpoint{1.487752in}{1.308442in}}{\pgfqpoint{1.487752in}{1.298129in}}%
\pgfpathcurveto{\pgfqpoint{1.487752in}{1.287815in}}{\pgfqpoint{1.491850in}{1.277923in}}{\pgfqpoint{1.499142in}{1.270630in}}%
\pgfpathcurveto{\pgfqpoint{1.506435in}{1.263337in}}{\pgfqpoint{1.516327in}{1.259240in}}{\pgfqpoint{1.526641in}{1.259240in}}%
\pgfpathclose%
\pgfusepath{stroke,fill}%
\end{pgfscope}%
\begin{pgfscope}%
\pgfpathrectangle{\pgfqpoint{0.632102in}{0.524958in}}{\pgfqpoint{3.717898in}{2.624375in}}%
\pgfusepath{clip}%
\pgfsetbuttcap%
\pgfsetroundjoin%
\definecolor{currentfill}{rgb}{0.282353,0.470588,0.811765}%
\pgfsetfillcolor{currentfill}%
\pgfsetlinewidth{0.240900pt}%
\definecolor{currentstroke}{rgb}{0.282353,0.470588,0.811765}%
\pgfsetstrokecolor{currentstroke}%
\pgfsetdash{}{0pt}%
\pgfpathmoveto{\pgfqpoint{1.641638in}{1.436798in}}%
\pgfpathcurveto{\pgfqpoint{1.651952in}{1.436798in}}{\pgfqpoint{1.661844in}{1.440896in}}{\pgfqpoint{1.669137in}{1.448189in}}%
\pgfpathcurveto{\pgfqpoint{1.676430in}{1.455481in}}{\pgfqpoint{1.680527in}{1.465374in}}{\pgfqpoint{1.680527in}{1.475687in}}%
\pgfpathcurveto{\pgfqpoint{1.680527in}{1.486001in}}{\pgfqpoint{1.676430in}{1.495893in}}{\pgfqpoint{1.669137in}{1.503186in}}%
\pgfpathcurveto{\pgfqpoint{1.661844in}{1.510479in}}{\pgfqpoint{1.651952in}{1.514576in}}{\pgfqpoint{1.641638in}{1.514576in}}%
\pgfpathcurveto{\pgfqpoint{1.631325in}{1.514576in}}{\pgfqpoint{1.621432in}{1.510479in}}{\pgfqpoint{1.614140in}{1.503186in}}%
\pgfpathcurveto{\pgfqpoint{1.606847in}{1.495893in}}{\pgfqpoint{1.602749in}{1.486001in}}{\pgfqpoint{1.602749in}{1.475687in}}%
\pgfpathcurveto{\pgfqpoint{1.602749in}{1.465374in}}{\pgfqpoint{1.606847in}{1.455481in}}{\pgfqpoint{1.614140in}{1.448189in}}%
\pgfpathcurveto{\pgfqpoint{1.621432in}{1.440896in}}{\pgfqpoint{1.631325in}{1.436798in}}{\pgfqpoint{1.641638in}{1.436798in}}%
\pgfpathclose%
\pgfusepath{stroke,fill}%
\end{pgfscope}%
\begin{pgfscope}%
\pgfpathrectangle{\pgfqpoint{0.632102in}{0.524958in}}{\pgfqpoint{3.717898in}{2.624375in}}%
\pgfusepath{clip}%
\pgfsetbuttcap%
\pgfsetroundjoin%
\definecolor{currentfill}{rgb}{0.282353,0.470588,0.811765}%
\pgfsetfillcolor{currentfill}%
\pgfsetlinewidth{0.240900pt}%
\definecolor{currentstroke}{rgb}{0.282353,0.470588,0.811765}%
\pgfsetstrokecolor{currentstroke}%
\pgfsetdash{}{0pt}%
\pgfpathmoveto{\pgfqpoint{2.085946in}{1.537204in}}%
\pgfpathcurveto{\pgfqpoint{2.096260in}{1.537204in}}{\pgfqpoint{2.106152in}{1.541301in}}{\pgfqpoint{2.113445in}{1.548594in}}%
\pgfpathcurveto{\pgfqpoint{2.120738in}{1.555887in}}{\pgfqpoint{2.124835in}{1.565779in}}{\pgfqpoint{2.124835in}{1.576092in}}%
\pgfpathcurveto{\pgfqpoint{2.124835in}{1.586406in}}{\pgfqpoint{2.120738in}{1.596298in}}{\pgfqpoint{2.113445in}{1.603591in}}%
\pgfpathcurveto{\pgfqpoint{2.106152in}{1.610884in}}{\pgfqpoint{2.096260in}{1.614981in}}{\pgfqpoint{2.085946in}{1.614981in}}%
\pgfpathcurveto{\pgfqpoint{2.075633in}{1.614981in}}{\pgfqpoint{2.065740in}{1.610884in}}{\pgfqpoint{2.058448in}{1.603591in}}%
\pgfpathcurveto{\pgfqpoint{2.051155in}{1.596298in}}{\pgfqpoint{2.047057in}{1.586406in}}{\pgfqpoint{2.047057in}{1.576092in}}%
\pgfpathcurveto{\pgfqpoint{2.047057in}{1.565779in}}{\pgfqpoint{2.051155in}{1.555887in}}{\pgfqpoint{2.058448in}{1.548594in}}%
\pgfpathcurveto{\pgfqpoint{2.065740in}{1.541301in}}{\pgfqpoint{2.075633in}{1.537204in}}{\pgfqpoint{2.085946in}{1.537204in}}%
\pgfpathclose%
\pgfusepath{stroke,fill}%
\end{pgfscope}%
\begin{pgfscope}%
\pgfpathrectangle{\pgfqpoint{0.632102in}{0.524958in}}{\pgfqpoint{3.717898in}{2.624375in}}%
\pgfusepath{clip}%
\pgfsetbuttcap%
\pgfsetroundjoin%
\definecolor{currentfill}{rgb}{0.282353,0.470588,0.811765}%
\pgfsetfillcolor{currentfill}%
\pgfsetlinewidth{0.240900pt}%
\definecolor{currentstroke}{rgb}{0.282353,0.470588,0.811765}%
\pgfsetstrokecolor{currentstroke}%
\pgfsetdash{}{0pt}%
\pgfpathmoveto{\pgfqpoint{2.389121in}{1.815167in}}%
\pgfpathcurveto{\pgfqpoint{2.399435in}{1.815167in}}{\pgfqpoint{2.409327in}{1.819265in}}{\pgfqpoint{2.416620in}{1.826558in}}%
\pgfpathcurveto{\pgfqpoint{2.423913in}{1.833850in}}{\pgfqpoint{2.428010in}{1.843743in}}{\pgfqpoint{2.428010in}{1.854056in}}%
\pgfpathcurveto{\pgfqpoint{2.428010in}{1.864370in}}{\pgfqpoint{2.423913in}{1.874262in}}{\pgfqpoint{2.416620in}{1.881555in}}%
\pgfpathcurveto{\pgfqpoint{2.409327in}{1.888848in}}{\pgfqpoint{2.399435in}{1.892945in}}{\pgfqpoint{2.389121in}{1.892945in}}%
\pgfpathcurveto{\pgfqpoint{2.378808in}{1.892945in}}{\pgfqpoint{2.368915in}{1.888848in}}{\pgfqpoint{2.361623in}{1.881555in}}%
\pgfpathcurveto{\pgfqpoint{2.354330in}{1.874262in}}{\pgfqpoint{2.350232in}{1.864370in}}{\pgfqpoint{2.350232in}{1.854056in}}%
\pgfpathcurveto{\pgfqpoint{2.350232in}{1.843743in}}{\pgfqpoint{2.354330in}{1.833850in}}{\pgfqpoint{2.361623in}{1.826558in}}%
\pgfpathcurveto{\pgfqpoint{2.368915in}{1.819265in}}{\pgfqpoint{2.378808in}{1.815167in}}{\pgfqpoint{2.389121in}{1.815167in}}%
\pgfpathclose%
\pgfusepath{stroke,fill}%
\end{pgfscope}%
\begin{pgfscope}%
\pgfpathrectangle{\pgfqpoint{0.632102in}{0.524958in}}{\pgfqpoint{3.717898in}{2.624375in}}%
\pgfusepath{clip}%
\pgfsetbuttcap%
\pgfsetroundjoin%
\definecolor{currentfill}{rgb}{0.282353,0.470588,0.811765}%
\pgfsetfillcolor{currentfill}%
\pgfsetlinewidth{0.240900pt}%
\definecolor{currentstroke}{rgb}{0.282353,0.470588,0.811765}%
\pgfsetstrokecolor{currentstroke}%
\pgfsetdash{}{0pt}%
\pgfpathmoveto{\pgfqpoint{2.786385in}{2.078687in}}%
\pgfpathcurveto{\pgfqpoint{2.796698in}{2.078687in}}{\pgfqpoint{2.806591in}{2.082784in}}{\pgfqpoint{2.813884in}{2.090077in}}%
\pgfpathcurveto{\pgfqpoint{2.821176in}{2.097370in}}{\pgfqpoint{2.825274in}{2.107262in}}{\pgfqpoint{2.825274in}{2.117576in}}%
\pgfpathcurveto{\pgfqpoint{2.825274in}{2.127889in}}{\pgfqpoint{2.821176in}{2.137782in}}{\pgfqpoint{2.813884in}{2.145074in}}%
\pgfpathcurveto{\pgfqpoint{2.806591in}{2.152367in}}{\pgfqpoint{2.796698in}{2.156465in}}{\pgfqpoint{2.786385in}{2.156465in}}%
\pgfpathcurveto{\pgfqpoint{2.776072in}{2.156465in}}{\pgfqpoint{2.766179in}{2.152367in}}{\pgfqpoint{2.758886in}{2.145074in}}%
\pgfpathcurveto{\pgfqpoint{2.751594in}{2.137782in}}{\pgfqpoint{2.747496in}{2.127889in}}{\pgfqpoint{2.747496in}{2.117576in}}%
\pgfpathcurveto{\pgfqpoint{2.747496in}{2.107262in}}{\pgfqpoint{2.751594in}{2.097370in}}{\pgfqpoint{2.758886in}{2.090077in}}%
\pgfpathcurveto{\pgfqpoint{2.766179in}{2.082784in}}{\pgfqpoint{2.776072in}{2.078687in}}{\pgfqpoint{2.786385in}{2.078687in}}%
\pgfpathclose%
\pgfusepath{stroke,fill}%
\end{pgfscope}%
\begin{pgfscope}%
\pgfpathrectangle{\pgfqpoint{0.632102in}{0.524958in}}{\pgfqpoint{3.717898in}{2.624375in}}%
\pgfusepath{clip}%
\pgfsetbuttcap%
\pgfsetroundjoin%
\definecolor{currentfill}{rgb}{0.282353,0.470588,0.811765}%
\pgfsetfillcolor{currentfill}%
\pgfsetlinewidth{0.240900pt}%
\definecolor{currentstroke}{rgb}{0.282353,0.470588,0.811765}%
\pgfsetstrokecolor{currentstroke}%
\pgfsetdash{}{0pt}%
\pgfpathmoveto{\pgfqpoint{3.063424in}{2.287952in}}%
\pgfpathcurveto{\pgfqpoint{3.073738in}{2.287952in}}{\pgfqpoint{3.083630in}{2.292050in}}{\pgfqpoint{3.090923in}{2.299343in}}%
\pgfpathcurveto{\pgfqpoint{3.098215in}{2.306635in}}{\pgfqpoint{3.102313in}{2.316528in}}{\pgfqpoint{3.102313in}{2.326841in}}%
\pgfpathcurveto{\pgfqpoint{3.102313in}{2.337155in}}{\pgfqpoint{3.098215in}{2.347047in}}{\pgfqpoint{3.090923in}{2.354340in}}%
\pgfpathcurveto{\pgfqpoint{3.083630in}{2.361633in}}{\pgfqpoint{3.073738in}{2.365730in}}{\pgfqpoint{3.063424in}{2.365730in}}%
\pgfpathcurveto{\pgfqpoint{3.053111in}{2.365730in}}{\pgfqpoint{3.043218in}{2.361633in}}{\pgfqpoint{3.035926in}{2.354340in}}%
\pgfpathcurveto{\pgfqpoint{3.028633in}{2.347047in}}{\pgfqpoint{3.024535in}{2.337155in}}{\pgfqpoint{3.024535in}{2.326841in}}%
\pgfpathcurveto{\pgfqpoint{3.024535in}{2.316528in}}{\pgfqpoint{3.028633in}{2.306635in}}{\pgfqpoint{3.035926in}{2.299343in}}%
\pgfpathcurveto{\pgfqpoint{3.043218in}{2.292050in}}{\pgfqpoint{3.053111in}{2.287952in}}{\pgfqpoint{3.063424in}{2.287952in}}%
\pgfpathclose%
\pgfusepath{stroke,fill}%
\end{pgfscope}%
\begin{pgfscope}%
\pgfpathrectangle{\pgfqpoint{0.632102in}{0.524958in}}{\pgfqpoint{3.717898in}{2.624375in}}%
\pgfusepath{clip}%
\pgfsetbuttcap%
\pgfsetroundjoin%
\definecolor{currentfill}{rgb}{0.282353,0.470588,0.811765}%
\pgfsetfillcolor{currentfill}%
\pgfsetlinewidth{0.240900pt}%
\definecolor{currentstroke}{rgb}{0.282353,0.470588,0.811765}%
\pgfsetstrokecolor{currentstroke}%
\pgfsetdash{}{0pt}%
\pgfpathmoveto{\pgfqpoint{3.638411in}{2.576485in}}%
\pgfpathcurveto{\pgfqpoint{3.648725in}{2.576485in}}{\pgfqpoint{3.658617in}{2.580583in}}{\pgfqpoint{3.665910in}{2.587875in}}%
\pgfpathcurveto{\pgfqpoint{3.673202in}{2.595168in}}{\pgfqpoint{3.677300in}{2.605061in}}{\pgfqpoint{3.677300in}{2.615374in}}%
\pgfpathcurveto{\pgfqpoint{3.677300in}{2.625688in}}{\pgfqpoint{3.673202in}{2.635580in}}{\pgfqpoint{3.665910in}{2.642873in}}%
\pgfpathcurveto{\pgfqpoint{3.658617in}{2.650165in}}{\pgfqpoint{3.648725in}{2.654263in}}{\pgfqpoint{3.638411in}{2.654263in}}%
\pgfpathcurveto{\pgfqpoint{3.628098in}{2.654263in}}{\pgfqpoint{3.618205in}{2.650165in}}{\pgfqpoint{3.610912in}{2.642873in}}%
\pgfpathcurveto{\pgfqpoint{3.603620in}{2.635580in}}{\pgfqpoint{3.599522in}{2.625688in}}{\pgfqpoint{3.599522in}{2.615374in}}%
\pgfpathcurveto{\pgfqpoint{3.599522in}{2.605061in}}{\pgfqpoint{3.603620in}{2.595168in}}{\pgfqpoint{3.610912in}{2.587875in}}%
\pgfpathcurveto{\pgfqpoint{3.618205in}{2.580583in}}{\pgfqpoint{3.628098in}{2.576485in}}{\pgfqpoint{3.638411in}{2.576485in}}%
\pgfpathclose%
\pgfusepath{stroke,fill}%
\end{pgfscope}%
\begin{pgfscope}%
\pgfpathrectangle{\pgfqpoint{0.632102in}{0.524958in}}{\pgfqpoint{3.717898in}{2.624375in}}%
\pgfusepath{clip}%
\pgfsetbuttcap%
\pgfsetroundjoin%
\definecolor{currentfill}{rgb}{0.282353,0.470588,0.811765}%
\pgfsetfillcolor{currentfill}%
\pgfsetlinewidth{0.240900pt}%
\definecolor{currentstroke}{rgb}{0.282353,0.470588,0.811765}%
\pgfsetstrokecolor{currentstroke}%
\pgfsetdash{}{0pt}%
\pgfpathmoveto{\pgfqpoint{3.706364in}{2.921033in}}%
\pgfpathcurveto{\pgfqpoint{3.716678in}{2.921033in}}{\pgfqpoint{3.726570in}{2.925131in}}{\pgfqpoint{3.733863in}{2.932424in}}%
\pgfpathcurveto{\pgfqpoint{3.741155in}{2.939716in}}{\pgfqpoint{3.745253in}{2.949609in}}{\pgfqpoint{3.745253in}{2.959922in}}%
\pgfpathcurveto{\pgfqpoint{3.745253in}{2.970236in}}{\pgfqpoint{3.741155in}{2.980128in}}{\pgfqpoint{3.733863in}{2.987421in}}%
\pgfpathcurveto{\pgfqpoint{3.726570in}{2.994714in}}{\pgfqpoint{3.716678in}{2.998811in}}{\pgfqpoint{3.706364in}{2.998811in}}%
\pgfpathcurveto{\pgfqpoint{3.696051in}{2.998811in}}{\pgfqpoint{3.686158in}{2.994714in}}{\pgfqpoint{3.678865in}{2.987421in}}%
\pgfpathcurveto{\pgfqpoint{3.671573in}{2.980128in}}{\pgfqpoint{3.667475in}{2.970236in}}{\pgfqpoint{3.667475in}{2.959922in}}%
\pgfpathcurveto{\pgfqpoint{3.667475in}{2.949609in}}{\pgfqpoint{3.671573in}{2.939716in}}{\pgfqpoint{3.678865in}{2.932424in}}%
\pgfpathcurveto{\pgfqpoint{3.686158in}{2.925131in}}{\pgfqpoint{3.696051in}{2.921033in}}{\pgfqpoint{3.706364in}{2.921033in}}%
\pgfpathclose%
\pgfusepath{stroke,fill}%
\end{pgfscope}%
\begin{pgfscope}%
\pgfpathrectangle{\pgfqpoint{0.632102in}{0.524958in}}{\pgfqpoint{3.717898in}{2.624375in}}%
\pgfusepath{clip}%
\pgfsetbuttcap%
\pgfsetroundjoin%
\definecolor{currentfill}{rgb}{0.282353,0.470588,0.811765}%
\pgfsetfillcolor{currentfill}%
\pgfsetlinewidth{0.240900pt}%
\definecolor{currentstroke}{rgb}{0.282353,0.470588,0.811765}%
\pgfsetstrokecolor{currentstroke}%
\pgfsetdash{}{0pt}%
\pgfpathmoveto{\pgfqpoint{4.176808in}{2.990789in}}%
\pgfpathcurveto{\pgfqpoint{4.187121in}{2.990789in}}{\pgfqpoint{4.197014in}{2.994886in}}{\pgfqpoint{4.204307in}{3.002179in}}%
\pgfpathcurveto{\pgfqpoint{4.211599in}{3.009472in}}{\pgfqpoint{4.215697in}{3.019364in}}{\pgfqpoint{4.215697in}{3.029678in}}%
\pgfpathcurveto{\pgfqpoint{4.215697in}{3.039991in}}{\pgfqpoint{4.211599in}{3.049883in}}{\pgfqpoint{4.204307in}{3.057176in}}%
\pgfpathcurveto{\pgfqpoint{4.197014in}{3.064469in}}{\pgfqpoint{4.187121in}{3.068566in}}{\pgfqpoint{4.176808in}{3.068566in}}%
\pgfpathcurveto{\pgfqpoint{4.166494in}{3.068566in}}{\pgfqpoint{4.156602in}{3.064469in}}{\pgfqpoint{4.149309in}{3.057176in}}%
\pgfpathcurveto{\pgfqpoint{4.142017in}{3.049883in}}{\pgfqpoint{4.137919in}{3.039991in}}{\pgfqpoint{4.137919in}{3.029678in}}%
\pgfpathcurveto{\pgfqpoint{4.137919in}{3.019364in}}{\pgfqpoint{4.142017in}{3.009472in}}{\pgfqpoint{4.149309in}{3.002179in}}%
\pgfpathcurveto{\pgfqpoint{4.156602in}{2.994886in}}{\pgfqpoint{4.166494in}{2.990789in}}{\pgfqpoint{4.176808in}{2.990789in}}%
\pgfpathclose%
\pgfusepath{stroke,fill}%
\end{pgfscope}%
\begin{pgfscope}%
\pgfsetrectcap%
\pgfsetmiterjoin%
\pgfsetlinewidth{1.003750pt}%
\definecolor{currentstroke}{rgb}{0.400000,0.400000,0.400000}%
\pgfsetstrokecolor{currentstroke}%
\pgfsetdash{}{0pt}%
\pgfpathmoveto{\pgfqpoint{0.632102in}{0.524958in}}%
\pgfpathlineto{\pgfqpoint{0.632102in}{3.149333in}}%
\pgfusepath{stroke}%
\end{pgfscope}%
\begin{pgfscope}%
\pgfsetrectcap%
\pgfsetmiterjoin%
\pgfsetlinewidth{1.003750pt}%
\definecolor{currentstroke}{rgb}{0.400000,0.400000,0.400000}%
\pgfsetstrokecolor{currentstroke}%
\pgfsetdash{}{0pt}%
\pgfpathmoveto{\pgfqpoint{0.632102in}{0.524958in}}%
\pgfpathlineto{\pgfqpoint{4.350000in}{0.524958in}}%
\pgfusepath{stroke}%
\end{pgfscope}%
\begin{pgfscope}%
\definecolor{textcolor}{rgb}{0.000000,0.000000,0.000000}%
\pgfsetstrokecolor{textcolor}%
\pgfsetfillcolor{textcolor}%
\pgftext[x=2.491051in,y=3.232667in,,base]{\color{textcolor}\rmfamily\fontsize{12.000000}{14.400000}\selectfont Relação da Corrente pela Tensão em um Resistor}%
\end{pgfscope}%
\end{pgfpicture}%
\makeatother%
\endgroup%


        \caption{Exemplo de um gráfico com pontos para os dados}
        \label{fig:reta:resultado}
    \end{figure}
